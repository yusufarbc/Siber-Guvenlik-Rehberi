\chapter{GÜVENLİK OPERASYONLARI VE SOC/NOC YÖNETİMİ}

\section*{Giriş}
Güvenlik operasyonları ve SOC/NOC yönetimi, organizasyonların siber tehditlere karşı sürekli izleme ve müdahale yeteneklerini sağlayan kritik operasyonel disiplinlerdir. Bu bölümde SOC/NOC mimarilerini, güvenlik operasyonları yönetimini ve SIEM platformlarını detaylı olarak inceleyeceğiz.

\section{Güvenlik Operasyonları Merkezi (SOC) Mimarisi}

Bir SOC'nin temel mimari bileşenlerini, operasyonel modellerini ve iç işleyişini derinlemesine inceleyen bu bölüm, siber savunma yeteneklerinin omurgasını oluşturan yapıları detaylandırmaktadır.

\subsection{SOC Organizasyonel Modelleri: Kurum İçi, Dış Kaynak ve Hibrit}

Her kuruluş, kendine özgü kaynak, risk ve kontrol gereksinimlerine göre en uygun SOC modelini seçmelidir. Bu karar, yalnızca teknik bir tercih değil, aynı zamanda stratejik bir iş kararıdır.

\textbf{Kurum İçi (In-house) SOC}

Kurum içi SOC, bir kuruluşun kendi personeli ve altyapısı ile yönettiği modeldir. Bu yaklaşımın temel gücü, güvenlik süreçlerinin doğrudan kontrol edilebilir ve kuruluşun özel ihtiyaçlarına göre özelleştirilebilir olmasıdır. Dahili ekipler, altyapıya, verilere ve iş hedeflerine dair derinlemesine bir bilgi birikimi geliştirirler, bu da daha hassas tehdit algılama ve daha hızlı yanıt süreleri sağlar. Hassas verilerin şirket içinde kalması, özellikle yasal düzenlemelerle uyum açısından kritik bir avantajdır.

Ancak bu model, yüksek başlangıç ve sürekli işletme maliyetleri gerektirir. Uzman personel işe alma, ileri düzey güvenlik araçlarına yatırım yapma ve sürekli eğitim masrafları önemli bir finansal yük oluşturur. Sınırlı personel sayısı, özellikle 7/24 izleme gereken durumlarda analistlerde tükenmişlik (burnout) riskini artırabilir. Ek olarak, kurum içi ekiplerin bilgi birikimi kendi ortamlarıyla sınırlı kalabilir, bu da daha geniş tehdit ortamına dair kör noktalar oluşturabilir.

\textbf{Dış Kaynak (Outsourced) SOC}

Dış kaynak SOC hizmeti, üçüncü taraf bir sağlayıcı (MSSP) tarafından sunulur. Bu model, özellikle maliyetleri düşürme ve uzmanlığa hızlı erişim sağlama açısından caziptir. Dış kaynak SOC'ler, genellikle 7/24 kesintisiz izleme yeteneğine sahiptir ve farklı sektörlerden topladıkları geniş tehdit istihbaratı veritabanından faydalanırlar. Bu, kuruluşun kendi başına erişemeyeceği geniş bir yetenek ve bilgi havuzuna erişimini sağlar.

Bu modelin başlıca dezavantajları, kontrol ve esneklik kaybıdır. Hizmet kalitesi sağlayıcıdan sağlayıcıya büyük farklılık gösterebilir ve hizmetin kapsamı sözleşme koşullarıyla sınırlıdır. Dış kaynak ekipler, bir kuruluşun iş kültürünü veya özel altyapısını tam olarak anlayamayabilir, bu da hizmette boşluklara yol açabilir. Uzun vadede, belirli bir sağlayıcının araçlarına ve süreçlerine bağımlılık, başka bir modele geçişi maliyetli ve karmaşık hale getirebilir.

\textbf{Hibrit SOC}

Hibrit model, kurum içi ve dış kaynak yaklaşımlarının bir kombinasyonudur. Bu modelde, kuruluş kritik ve hassas operasyonları içeride tutarken, rutin izleme gibi daha az kritik görevleri dışarıdan sağlar. Bu yaklaşım, bütçeyi dengelemeye, ölçeklenebilirliği artırmaya ve yetenekleri en iyi şekilde kullanmaya olanak tanır.

Bu modelin en büyük zorluğu, yönetim karmaşıklığıdır. Farklı ekipler ve sözleşmeler arasında koordinasyon sağlamak, özel yönetim becerileri gerektirir. Kurum içi ve dış kaynak ekipler arasında etkili bir iletişim kurmak zor olabilir ve birden fazla sağlayıcıyı yönetmek, başlangıçta öngörülemeyen gizli maliyetler yaratabilir. Verilerin birden fazla yerde saklanması, yönetimi ve güvenliği daha karmaşık hale getirir. Her modelin kendine özgü güçlü ve zayıf yönleri, SOC model seçimini yalnızca bir teknoloji kararı olmaktan çıkarıp, risk, kontrol ve finansal çeviklik arasında bir denge kurmayı gerektiren, sürekli bir optimizasyon süreci haline getirmektedir.

\begin{longtable}{|p{3cm}|p{3cm}|p{3cm}|p{3cm}|}
\hline
\textbf{Kriter} & \textbf{Kurum İçi (In-house)} & \textbf{Dış Kaynak (Outsourced)} & \textbf{Hibrit} \\
\hline
\textbf{Tanım} & Kendi personeli ve altyapısıyla yönetilir. & Üçüncü taraf bir sağlayıcı (MSSP) tarafından sunulur. & Kritik fonksiyonlar kurum içinde tutulur, rutin görevler dış kaynak olarak kullanılır. \\
\hline
\textbf{Maliyet} & Yüksek başlangıç ve işletme giderleri. & Genellikle daha düşük maliyetli, abonelik tabanlı. & Dengeleyici, ancak gizli yönetim maliyetleri olabilir. \\
\hline
\textbf{Kontrol Düzeyi} & Maksimum kontrol ve kişiselleştirme. & Kısıtlı kontrol, sözleşme şartlarına bağlı. & Kritik fonksiyonlarda yüksek, diğerlerinde kısıtlı kontrol. \\
\hline
\textbf{Uzmanlık Erişimi} & Sınırlı, kendi personel yetkinliğine bağlı. & Geniş yetenek ve tehdit istihbaratı havuzuna erişim. & Her iki dünyanın avantajı: Hem iç yetenek hem dış uzmanlık. \\
\hline
\textbf{Ölçeklenebilirlik} & Sınırlı ve maliyetlidir. & Kolay ve hızlı ölçeklenebilir. & Kaynakları hem içeriden hem dışarıdan ekleyerek kolay ölçeklenir. \\
\hline
\textbf{İletişim Zorluğu} & Kolay, aynı çatı altında. & Zorlu, farklı kuruluşlar ve süreçler arası. & Karmaşık, iki ekip arasında sürekli koordinasyon gerektirir. \\
\hline
\end{longtable}

\subsection{SOC Teknoloji Yığını ve Araç Entegrasyonu}

Bir SOC'nin operasyonel yeteneklerini destekleyen teknoloji yığını, basit bir araç koleksiyonundan ziyade, her bir bileşenin birbiriyle entegre olması gereken karmaşık bir ekosistemdir. Bu entegrasyon, bir SOC'nin "kör noktaları ve boşlukları gidermek" için hayati önem taşır.

\textbf{Temel Araçlar:}

\begin{itemize}
    \item \textbf{SIEM (Security Information and Event Management):} Bir SOC'nin temel taşıdır. Tüm log ve olay verilerini kuruluşun çeşitli kaynaklarından (sunucular, ağ cihazları, uygulamalar, güvenlik araçları) toplar, normalleştirir ve korelasyon analizi yaparak potansiyel güvenlik olaylarını tespit eder.
    \item \textbf{SOAR (Security Orchestration, Automation and Response):} SIEM'den gelen uyarıları ele alarak güvenlik araçlarını entegre eder ve otomatik iş akışları (playbook'lar) oluşturur. Bu, tekrarlayan görevleri otomatikleştirir ve olay yanıtını hızlandırır.
    \item \textbf{EDR/XDR (Endpoint Detection and Response/Extended Detection and Response):} Uç noktalarda (bilgisayarlar, sunucular) tehditleri algılar, analiz eder ve yanıt verir. EDR/XDR sistemlerinden gelen telemetri verileri, SIEM'e gönderilerek olayların daha geniş bir bağlamda analiz edilmesini sağlar.
    \item \textbf{TIP (Threat Intelligence Platform):} Saldırganların taktikleri, teknikleri, araçları ve göstergeleri hakkında dış tehdit istihbaratı beslemelerini yönetir ve SIEM ve SOAR gibi araçlara entegre ederek uyarıların zenginleştirilmesini sağlar.
\end{itemize}

Bu araçlar arasındaki kusursuz veri akışı, bir SOC'nin etkinliğini belirleyen ana faktördür. Bir SIEM, tek başına bir veri depolama aracı olarak kalırken, diğer güvenlik çözümleriyle entegre olduğunda, her birinin değerini katlayan güçlü bir istihbarat motoruna dönüşür. Tek bir çözümün yetersiz olduğu göz önüne alındığında, bir aracın değeri, sadece kendi yeteneklerinden değil, aynı zamanda diğer araçlarla ne kadar iyi işbirliği yapabildiğinden kaynaklanmaktadır. Bu durum, teknoloji yığınını yönetmeyi basit bir satın alma işleminden daha karmaşık bir ekosistem mühendisliği problemine dönüştürmektedir.

\subsection{SOC Rolleri ve Sorumlulukları (L1, L2, L3 Analistleri)}

Bir SOC, operasyonel verimliliği ve insan sermayesi yönetimini optimize etmek için katmanlı bir yapıda çalışır. Bu katmanlı model, iş akışlarını düzenler ve uzmanlık alanlarının en verimli şekilde kullanılmasını sağlar.

\textbf{L1 (Tier 1) Analisti}

L1 analistleri, bir SOC'nin ilk müdahale ekibidir. Başlıca sorumlulukları, gelen güvenlik uyarılarını önceliklendirmek (triage), hatalı pozitifleri (false positives) filtrelemek ve standart prosedürlerle çözülebilecek basit olayları ele almaktır. Bu filtreleme süreci, L2 ve L3 analistlerini, sık sık görülen uyarı hacminin yarattığı "alarm yorgunluğu"ndan korur. L1 analistleri, bir olay kendi kapsamlarını aştığında, daha derinlemesine bir araştırma için olayı ve toplanan tüm bilgileri L2 analistlerine iletirler.

\textbf{L2 (Tier 2) Analisti}

L2 analistleri, L1'den eskalasyonla gelen olayları derinlemesine incelemekten sorumludur. Bu görevler, kapsamlı log analizi, adli bilişim (forensics) incelemeleri ve tehditlerin nasıl sınırlandırılıp onarılacağına dair detaylı stratejiler geliştirmeyi içerir. Ayrıca, yeni tehditleri tespit etmek için özel korelasyon kuralları ve tespit mantığı oluşturma yeteneğine de sahiptirler.

\textbf{L3 (Tier 3) Analisti}

L3 analistleri, en yüksek uzmanlık seviyesini temsil ederler. En karmaşık güvenlik olaylarıyla (malware tersine mühendisliği, ağ adli bilişimi) ilgilenirler. Olay müdahalesinin ötesinde, tehdit avcılığı (threat hunting) metodolojileri oluşturur, şirket çapında güvenlik stratejileri tasarlar ve L1/L2 analistlerine mentorluk yaparak organizasyonel bilginin kurum içinde kalmasını sağlarlar. Bu katmanlama, SOC'nin sadece var olan tehditlere reaktif olarak yanıt vermekle kalmayıp, aynı zamanda geleceğe yönelik olarak savunma yeteneklerini proaktif bir şekilde güçlendirmesini sağlayan stratejik bir yapıdır.

\begin{longtable}{|p{3cm}|p{3cm}|p{3cm}|p{3cm}|}
\hline
\textbf{Rol} & \textbf{Ana Sorumluluklar} & \textbf{Gerekli Beceriler} & \textbf{Ana Araçlar} \\
\hline
\textbf{L1 Analisti} & Uyarı triyajı, hatalı pozitif filtrelemesi, olay eskalasyonu, izleme araçlarının ayarlanması. & Olay yönetim bilgisi, temel log analizi, güvenlik araçları bilgisi. & SIEM, EDR, Ticketing System. \\
\hline
\textbf{L2 Analisti} & Derinlemesine olay analizi, adli bilişim, içerik oluşturma (korelasyon kuralı geliştirme). & İleri seviye log analizi, adli bilişim, tehdit istihbaratı entegrasyonu. & SIEM, SOAR, EDR/XDR, TIP. \\
\hline
\textbf{L3 Analisti} & Uzman olay müdahalesi, malware analizi, tehdit avcılığı, güvenlik stratejisi geliştirme, mentorluk. & Tersine mühendislik, ağ ve sistem adli bilişimi, güvenlik mimarisi tasarımı. & SIEM, SOAR, EDR/XDR, TIP, Özel araçlar. \\
\hline
\end{longtable}

\subsection{7/24 Operasyon Yönetimi ve Vardiya Planlaması}

Kesintisiz güvenlik izlemesi, modern bir SOC için hayati önem taşır. Ancak, 7/24 operasyonun yönetimi, analistlerde hızlı tükenmeye yol açabilecek en büyük zorluklardan biridir. Yüksek personel devir hızı, ekip performansını ve kurumsal bilgiyi olumsuz etkiler.

Bu zorluğun üstesinden gelmek için akıllı vardiya planlaması ve otomasyon stratejileri benimsenmelidir. Yaygın vardiya modelleri arasında 8 veya 12 saatlik vardiyalar bulunur. Popüler 12 saatlik modellerden biri, analistlere dört gün çalışma ve ardından dört gün tatil imkanı sunan "4-on, 4-off" modelidir. DuPont ve Pitman gibi daha karmaşık rotasyon programları, personel yorgunluğunu azaltmak ve sürekli uyanıklığı sağlamak için tasarlanmıştır.

Vardiya planlaması sadece bir çizelge sorunu değildir; aynı zamanda insan sermayesi yönetimi sorunudur. İş yükünün vardiyalar arasında adil bir şekilde dağıtılması ve tekrarlayan görevlerin otomasyona devredilmesi, tükenmişliği önlemenin anahtarıdır. Otomasyon, özellikle basit log analizleri ve triyaj gibi görevleri üstlenerek analistlerin iş yükünü hafifletir ve daha karmaşık, stratejik görevlere odaklanmalarını sağlar. Vardiya değişimlerinde, kritik bilginin ve bağlamın bir ekipten diğerine kesintisiz bir şekilde aktarılması için kapsamlı devir prosedürlerinin oluşturulması gereklidir. Bu süreçler, olay yanıtında kritik bilgi kaybını önler ve tutarlı bir performans sağlar.

\subsection{SOC Olgunluk Modelleri ve Yetenek Değerlendirmesi}

Bir SOC'nin ne kadar etkili çalıştığını değerlendirmek için, operasyonel olgunluğunu ölçen modeller kullanılır. SOC-CMM (SOC Capability Maturity Model) gibi modeller, bir SOC'nin mevcut durumunu değerlendirmek ve onu reaktif bir "itfaiyeci" rolünden proaktif, stratejik bir güvenlik fonksiyonuna dönüştürmek için bir yol haritası sunar.

Bu yetenek değerlendirme süreçleri, mevcut güvenlik kontrollerindeki boşlukların belirlenmesiyle başlar. L3 analistleri gibi üst düzey uzmanlar, mevcut güvenlik duruşunu analiz eder, tehdit avcılığı metodolojileri geliştirir ve operasyonel süreçlerin nasıl iyileştirileceğine dair stratejik tavsiyelerde bulunur. Bu değerlendirmeler, organizasyonların sadece anlık tehditlere yanıt vermekle kalmayıp, aynı zamanda sürekli öğrenme ve iyileştirme döngüsünü kurumsallaştırmasını sağlar. Bir olgunluk modeli, soyut iyileştirme hedeflerini somut, ölçülebilir ve uygulanabilir bir sürece dönüştüren yapısal bir çerçevedir. Bu, SOC'nin sürekli gelişen tehdit ortamına uyum sağlaması ve savunma yeteneklerini sürekli olarak güçlendirmesi için hayati bir mekanizmadır.

\section{Ağ Operasyon Merkezi (NOC) ve SOC Entegrasyonu}

Bu bölüm, NOC ve SOC arasındaki temel ayrımı, işbirliği mekanizmalarını ve iki fonksiyonun birleşimi olan Birleşik Operasyon Merkezi (UOC) modelini incelemektedir. Bu entegrasyon, operasyonel verimliliği artırırken güvenlik duruşunu güçlendirmek için kritik öneme sahiptir.

\subsection{NOC İşlevselliği ve Ağ İzleme Yetenekleri}

Bir Ağ Operasyon Merkezi (NOC), ağ sistemlerini 7/24 izleyen ve yöneten merkezi bir birimdir. NOC, bir kuruluşun tüm ağ altyapısının kalbi olarak görev yapar ve modern iş operasyonlarının hayati teknolojik omurgasını oluşturur.

\textbf{Temel NOC İşlevleri:}
\begin{itemize}
    \item \textbf{Ağ Performans İzleme:} Bant genişliği kullanımı, gecikme (latency), paket kaybı ve jitter gibi kritik ağ metriklerinin sürekli izlenmesi
    \item \textbf{Altyapı Sağlığı Yönetimi:} Router, switch, firewall ve sunucuların çalışma durumunun real-time takibi
    \item \textbf{Kapasite Planlama:} Ağ trafiği trendlerini analiz ederek gelecekteki kapasite ihtiyaçlarını öngörme
    \item \textbf{Konfigürasyon Yönetimi:} Ağ cihazlarının ayar değişikliklerini takip etme ve versiyon kontrolü
    \item \textbf{Sorun Giderme ve Root Cause Analysis:} Ağ kesintilerinin kök nedenlerini belirleme ve çözme
    \item \textbf{Yama ve Güncelleme Yönetimi:} Sistem güncellemelerinin planlanması ve uygulanması
    \item \textbf{Backup ve Felaket Kurtarma:} Veri yedekleme operasyonları ve iş sürekliliği planlarının yürütülmesi
\end{itemize}

\textbf{İleri Seviye NOC Yetenekleri:}

NOC, basit ağ izlemesinin ötesinde karmaşık ağ optimizasyon ve otomasyon yetenekleri sunar. **Intent-Based Networking (IBN)** gibi modern teknolojiler, NOC'nin ağ politikalarını otomatik olarak uygulamasına ve ağın kendini yapılandırmasına olanak tanır. **Network Digital Twins** kavramı ile NOC, fiziksel ağın sanal bir modelini oluşturarak değişikliklerin etkilerini test edebilir. Ayrıca **Predictive Analytics** kullanarak potansiyel ağ sorunlarını önceden tahmin eder ve proaktif bakım planları geliştirir.

NOC'nin asıl amacı, iş sürekliliğini ve müşteri memnuniyetini sağlamaktır. Sağlıklı bir ağ altyapısı, güvenlik operasyonlarının temelini oluşturur. Ancak, ağdaki bir anormallik (örneğin, alışılmadık trafik hacmi veya bir sunucu hatası) operasyonel bir sorundan (NOC'nin alanı) veya siber bir tehditten (SOC'nin alanı) kaynaklanabilir. Bu belirsizlik, iki merkezin işbirliğinin neden vazgeçilmez olduğunu ortaya koymaktadır.

\subsection{Failover ve Yedeklilik Mekanizmaları}

NOC operasyonlarının kritik başarı faktörlerinden biri, sistem arızalarında hızlı failover ve yedeklilik (redundancy) sağlama yeteneğidir. Modern iş dünyasında kesintinin maliyeti çok yüksek olduğundan, NOC'nin çoklu katmanlarda yedeklilik stratejileri geliştirmesi gerekir.

\textbf{Failover Türleri ve Uygulamaları:}
\begin{itemize}
    \item \textbf{Aktif-Pasif Failover:} Birincil sistem çalışırken yedek sistem bekleme modunda bulunur. Arıza durumunda yedek sistem devreye girer.
    \item \textbf{Aktif-Aktif Failover:} Her iki sistem de aynı anda çalışır ve yük paylaşımı yapar. Bir sistemin arızalanması durumunda diğeri tüm yükü üstlenir.
    \item \textbf{Load Balancing ile Failover:} Trafiği birden fazla sunucu arasında dağıtan ve arızalı sunucuları otomatik olarak devre dışı bırakan mekanizma.
\end{itemize}

\textbf{Kritik Altyapı Bileşenleri için Yedeklilik:}
- **Güç Kaynakları:** UPS sistemleri ve jeneratörler ile elektrik kesintilerine karşı koruma
- **Ağ Bağlantıları:** Birden fazla ISP ve farklı rotalar üzerinden internet erişimi
- **Veri Merkezleri:** Coğrafi olarak dağıtılmış veri merkezleri ile felaket kurtarma
- **Kritik Cihazlar:** Router, switch ve firewall'ların yedek örneklerinin hazır bulunması

\textbf{Failover Test Prosedürleri:}
NOC, düzenli olarak failover testleri gerçekleştirmeli ve Recovery Time Objective (RTO) ile Recovery Point Objective (RPO) hedeflerini doğrulamalıdır. Bu testler, gerçek bir felaket anında sistemlerin beklendiği gibi çalışacağını garanti eder.

\subsection{Network Monitoring ve Performance Management}

Etkili NOC operasyonları, ağ performansının proaktif izlenmesi ve yönetilmesi üzerine kuruludur. Modern ağ izleme sistemleri, basit ping testlerinin ötesine geçerek derinlemesine analitik yetenekler sunar.

\textbf{Kritik İzleme Metrikleri:}
\begin{itemize}
    \item \textbf{Throughput (İş Hacmi):} Belirli bir zaman diliminde ağdan geçen veri miktarı
    \item \textbf{Latency (Gecikme):} Veri paketinin kaynak ile hedef arasında seyahat süresi  
    \item \textbf{Packet Loss (Paket Kaybı):} İletim sırasında kaybolan veri paketlerinin yüzdesi
    \item \textbf{Jitter:} Gecikme sürelerindeki değişkenlik, özellikle real-time uygulamalar için kritik
    \item \textbf{Bandwidth Utilization:} Kullanılabilir bant genişliğinin ne kadarının aktif olarak kullanıldığı
    \item \textbf{Error Rate:} Ağ cihazlarında meydana gelen hata oranları
\end{itemize}

\textbf{İleri Seviye Monitoring Teknolojileri:}
- **Deep Packet Inspection (DPI):** Ağ trafiğinin içeriğini analiz ederek uygulama bazında performans izleme
- **Network Flow Analysis:** NetFlow, sFlow ve IPFIX protokolleri ile trafik akış analizi
- **Synthetic Monitoring:** Yapay test trafiği oluşturarak ağ performansını sürekli test etme
- **Real User Monitoring (RUM):** Gerçek kullanıcı deneyimini izleyerek performans metriklerini toplama

\subsection{Capacity Planning ve Ölçeklenebilirlik}

NOC'nin stratejik sorumluluklarından biri, gelecekteki kapasite ihtiyaçlarını öngörmek ve ağın büyümeyi destekleyecek şekilde ölçeklenmesini sağlamaktır.

\textbf{Kapasite Planlama Metodolojisi:}
\begin{enumerate}
    \item \textbf{Baseline Belirleme:} Mevcut ağ kullanımının detaylı analizi ve normal operasyon parametrelerinin belirlenmesi
    \item \textbf{Trend Analizi:} Geçmiş verileri kullanarak büyüme trendlerinin matematiksel modellemesi
    \item \textbf{İş Büyüme Projeksiyonları:} İş birimlerinin büyüme planları ile ağ kapasitesi gereksinimlerinin ilişkilendirilmesi
    \item \textbf{Peak Load Analizi:} En yoğun kullanım dönemlerinin analizi ve aşırı yük senaryolarının planlanması
    \item \textbf{Teknoloji Roadmap:} Yeni teknolojilerin (5G, IoT, Cloud) ağ kapasitesi üzerindeki etkilerinin değerlendirilmesi
\end{enumerate}

\textbf{Ölçeklenebilirlik Stratejileri:}
- **Horizontal Scaling:** Daha fazla cihaz ekleyerek kapasiteyi artırma
- **Vertical Scaling:** Mevcut cihazların performansını artırma
- **Software-Defined Networking (SDN):** Yazılım tabanlı ağ yönetimi ile dinamik ölçeklendirme
- **Network Function Virtualization (NFV):** Ağ fonksiyonlarının sanallaştırılması ile maliyet etkin ölçekleme

\subsection{SOC-NOC İşbirliği ve Bilgi Paylaşımı}

Geleneksel olarak, NOC ve SOC ekipleri, farklı hedeflere sahip ayrı silolar olarak çalışır. Ancak, her iki ekibin de nihai amacı hizmetlerin sürekli kullanılabilirliğini sağlamaktır. Bu nedenle, bu siloları kırmak, operasyonel verimsizliği ve güvenlik risklerini azaltmak için kritik öneme sahiptir.

\textbf{Entegre İşbirliği Modelleri:}
\begin{itemize}
    \item \textbf{Cross-Functional Teams:} NOC ve SOC personelinin ortak projeler ve olay müdahalelerinde birlikte çalışması
    \item \textbf{Shared Dashboards:} Her iki ekibin erişebileceği ortak izleme panelleri ve alarm sistemleri
    \item \textbf{Joint Incident Response:} Güvenlik ve operasyonel olayların birleşik müdahale prosedürleri ile ele alınması
    \item \textbf{Knowledge Sharing:} Düzenli bilgi paylaşımı toplantıları ve çapraz eğitim programları
\end{itemize}

İşbirliği, ortak veri kaynakları, entegre araçlar ve standartlaştırılmış süreçler aracılığıyla gerçekleştirilir. Örneğin, bir NOC tarafından tespit edilen anormal trafik desenleri, bir siber tehdide işaret edebileceğinden derhal SOC ekibine aktarılmalıdır. Bu işbirliği, operasyonel ve güvenlik sorunlarının hızla ilişkilendirilmesini sağlar. Veri çoğaltmasını en aza indirerek ve bilgi paylaşımını artırarak, bu entegrasyonlar olay yanıt sürelerini hızlandırır ve maliyet etkinliğini artırır. NOC'nin gördüğü bir ağ performansı sorunu, bir SOC'nin araştırdığı veri sızdırma girişimiyle ilişkili olabilir.

\subsection{Change Management ve Konfigürasyon Kontrolü}

NOC operasyonlarının kritik bir bileşeni, ağ altyapısındaki değişikliklerin kontrollü ve güvenli bir şekilde yönetilmesidir. Yanlış yapılandırılan bir değişiklik, tüm ağı çökertebilir ve işletmeye milyonlarca dolar zarara neden olabilir.

\textbf{Change Management Süreç Adımları:}
\begin{enumerate}
    \item \textbf{Change Request (RFC):} Değişiklik talebinin resmi olarak dokumentasyonu ve gerekçelendirilmesi
    \item \textbf{Impact Assessment:} Değişikliğin sistem üzerindeki potansiyel etkilerinin analizi
    \item \textbf{Change Advisory Board (CAB):} Teknik ve iş temsilcilerinin katıldığı değişiklik onay komitesi
    \item \textbf{Testing:} Lab ortamında değişikliğin test edilmesi ve validation
    \item \textbf{Implementation:} Üretim ortamına kontrollü değişiklik uygulaması
    \item \textbf{Post-Implementation Review:} Değişikliğin başarısının değerlendirilmesi ve rollback planı
\end{enumerate}

\textbf{Konfigürasyon Yönetimi Best Practice'leri:}
- **Version Control:** Tüm network cihaz konfigürasyonlarının versiyonlanması ve merkezi depoda saklanması
- **Automated Backup:** Konfigürasyon değişikliklerinden önce otomatik yedekleme
- **Configuration Drift Detection:** Beklenmedik konfigürasyon değişikliklerinin otomatik tespiti
- **Standard Templates:** Standart konfigürasyon şablonları ile tutarlılık sağlama
- **Emergency Change Procedures:** Acil durumlar için hızlandırılmış değişiklik prosedürleri

\textbf{Change Management Araçları:**
Modern NOC'ler, change management süreçlerini destekleyen ITIL tabanlı araçlar kullanır. ServiceNow, Remedy ve Jira Service Management gibi platformlar, değişiklik yaşam döngüsünü otomatikleştirir ve audit trail sağlar.

\subsection{NOC Automation ve Orchestration}

Modern NOC operasyonları, manuel süreçlerin otomasyonuyla verimliliği artırır ve insan hatalarını azaltır. Otomasyon, rutin görevlerden karmaşık iş akışlarına kadar geniş bir spektrumda uygulanır.

\textbf{Temel Otomasyon Alanları:}
\begin{itemize}
    \item \textbf{Network Discovery:} Yeni cihazların otomatik keşfi ve envantere eklenmesi
    \item \textbf{Configuration Management:} Konfigürasyonların otomatik dağıtımı ve güncellenmesi
    \item \textbf{Patch Management:} Güvenlik yamalarının otomatik test edilmesi ve uygulanması
    \item \textbf{Backup Automation:} Düzenli veri yedekleme işlemlerinin otomatikleştirilmesi
    \item \textbf{Health Checks:} Sistem sağlık kontrollerinin otomatik gerçekleştirilmesi
    \item \textbf{Incident Response:} Belirli alarm türleri için otomatik müdahale prosedürleri
\end{itemize}

\textbf{Network Orchestration Teknolojileri:}
- **Ansible Network Automation:** Ağ cihazlarının yapılandırması için playbook tabanlı otomasyon
- **Terraform Network Modules:** Infrastructure as Code yaklaşımıyla ağ altyapısı yönetimi
- **Python Network Scripts:** Özel otomasyon görevleri için Python tabanlı scriptler
- **NETCONF/YANG:** Standartlaştırılmış ağ yönetimi protokolleri
- **REST API Integration:** Ağ cihazlarının API'leri üzerinden otomatik yönetimi

\textbf{Self-Healing Networks:**
Gelişmiş NOC otomasyon sistemleri, belirli sorunları otomatik olarak tespit edip düzeltebilen "self-healing" yetenekleri geliştirir. Bu sistemler, alarm aldıktan sonra otomatik olarak tanılama yapabilir, geçici çözümler uygulayabilir ve gerekirse failover işlemlerini başlatabilir.

\subsection{Birleşik Operasyon Merkezi (UOC) Modelleri}

UOC, NOC ve SOC işlevlerini birleştirerek, tüm iş operasyonları için tek bir merkezi komuta ve kontrol noktası oluşturan bir yaklaşımdır. Bu model, mühendislik, operasyon, güvenlik, performans ve finansal verileri tek bir "cam bölme" (single pane of glass) görünümünde birleştirir.

\textbf{UOC'nin Temel Bileşenleri:}
\begin{itemize}
    \item \textbf{Unified Dashboard:} NOC, SOC ve diğer operasyonel verilerin tek ekranda görüntülenmesi
    \item \textbf{Cross-Functional Analytics:} Güvenlik, performans ve operasyonel metriklerin korele edilmesi
    \item \textbf{Integrated Incident Management:} Tüm olay türleri için birleşik müdahale süreçleri
    \item \textbf{Shared Knowledge Base:} Ortak prosedürler ve çözüm kütüphanesi
    \item \textbf{Joint Training Programs:} Çapraz fonksiyonel eğitim ve sertifikasyon programları
\end{itemize}

Bu birleşik yaklaşım, veri ve iletişim silolarını kırar ve daha iyi işbirliği, daha hızlı karar alma ve artan operasyonel verimlilik sağlar. UOC, özellikle enerji, su yönetimi ve akıllı şehirler gibi endüstrilerde, operasyonel teknolojiler (OT) ile bilgi teknolojileri (IT) arasındaki verileri entegre ederek uçtan uca görünürlük sağlar. Bu, güvenlik ve operasyonel kararların, işletmenin daha geniş hedefleriyle uyumlu hale getirilmesine olanak tanıyan, fonksiyon merkezli bir modelden (NOC, SOC) iş çıktısı merkezli bir modele stratejik bir geçişi temsil eder.

\subsection{Hizmet Seviyesi Anlaşmaları (SLA) ve Performans Metrikleri}

Hizmet Seviyesi Anlaşmaları (SLA), bir hizmet sağlayıcı ile müşteri arasındaki hizmet kalitesini tanımlayan resmi sözleşmelerdir. Bu anlaşmalar, performansı somut ve ölçülebilir hale getirmek için çeşitli metrikleri içerir.

\textbf{Kritik Metrikler:}

\begin{itemize}
    \item \textbf{Ortalama Tespit Süresi (MTTD - Mean Time to Detect):} Bir tehdidin oluştuğu an ile SOC tarafından tespit edildiği an arasında geçen ortalama süreyi ölçer. Düşük bir MTTD, saldırganların sistemde kalış süresinin azaldığı anlamına gelir, bu da potansiyel hasarı sınırlar.
    \item \textbf{Ortalama Yanıt Süresi (MTTR - Mean Time to Respond):} Bir uyarının tespit edilmesinden sonra, analistin olaya müdahale etmek için geçirdiği ortalama süreyi ölçer. Bu, genellikle olayın önceliğine bağlı olarak değişir.
    \item \textbf{Ortalama Çözüm Süresi (MTTR - Mean Time to Resolution):} Bir olayın başlangıcından itibaren tamamen çözülmesine ve normal operasyonlara dönülmesine kadar geçen toplam süreyi ölçer.
    \item \textbf{Hatalı Pozitif Oranı (FPR):} Bir güvenlik uyarısının yanlışlıkla bir tehdit olarak etiketlenme yüzdesini gösterir. Yüksek bir FPR, analist yorgunluğuna yol açar ve verimliliği düşürür.
\end{itemize}

Bu metrikler, SOC ve NOC'nin teknik performansını somutlaştırarak, siber güvenlik yatırımlarının iş üzerindeki etkisini ve getirisini kanıtlamaya olanak tanır. Düşük bir MTTR, doğrudan operasyonel riski ve olası finansal kayıpları azaltır. Bu nedenle, metrikler sadece bir raporlama aracı değil, aynı zamanda sürekli iyileştirme için temel bir yol göstericidir.

\subsection{Eskalasyon Prosedürleri ve Olay Devir Süreçleri}

Etkili bir olay yanıtı, her olayın önceliğine ve niteliğine göre doğru ekibe, doğru zamanda ve doğru bilgilerle aktarılmasını gerektirir. İyi tanımlanmış eskalasyon prosedürleri, olay yanıt sürecinde bir olayın "kaybolmasını" önleyen, yanıt süresini minimize eden ve doğru uzmanlığın doğru zamanda devreye girmesini sağlayan kritik bir operasyonel kontrol mekanizmasıdır.

Süreç, L1 analistleri tarafından yapılan olay triyajıyla başlar. Olay, aciliyetine ve karmaşıklığına göre önceliklendirilir. Eğer olay, L1'in yetki alanını aşarsa, önceden belirlenmiş protokollere göre L2'ye veya duruma göre diğer birimlere (örneğin, üst yönetim, hukuk veya insan kaynakları) eskalasyon yapılır. Bu süreçte, tüm bağlamsal bilgilerin (loglar, analiz notları, ilgili sistemler) eksiksiz bir şekilde aktarılması büyük önem taşır.

NOC ile SOC arasındaki devir süreçleri de bu prosedürlerin bir alt kümesidir. Örneğin, bir NOC, ağda bir anormallik tespit ettiğinde ve bunun potansiyel bir siber saldırı olduğunu belirlediğinde, olayı resmi bir devir süreciyle SOC'ye aktarır. Bu tür formalize edilmiş prosedürlerin olmaması, kaotik bir durumda bile koordineli bir yanıtı engelleyerek olay yanıtında kritik gecikmelere yol açabilir.

\section{SIEM Platform Yönetimi ve Log Analizi}

Bir SIEM platformunun yönetimi ve log analizi, modern bir SOC'nin temel yeteneklerini oluşturur. Bu bölüm, bir SIEM'in mimarisini, veri işleme süreçlerini ve tehdit tespiti için nasıl kullanıldığını ele almaktadır.

\subsection{SIEM Mimari Tasarımı ve Ölçeklenebilirlik}

SIEM mimarisi, bir SOC'nin gelecekteki büyüme ve veri hacmiyle başa çıkma yeteneğini doğrudan belirler. Temel mimari bileşenler arasında veri toplama, normalizasyon, korelasyon motoru, uyarı ve raporlama, log yönetimi ve saklama yer alır.

Modern bir SIEM'in en büyük zorluğu, yüksek hacimli, hızlı ve çeşitli güvenlik verilerini yönetmektir. Geleneksel (on-premise) SIEM mimarileri, bu tür büyük veri hacimlerini yönetmede ve maliyet etkin bir şekilde saklamada zorluklar yaşayabilir. Bu zorlukların üstesinden gelmek için, SIEM'ler dağıtık mimarilere (horizontal scaling) ve bulut tabanlı çözümlere yönelmektedir. Bulut tabanlı SIEM'ler, otomatik ölçeklendirme ve esneklik sunarak, bir kuruluşun finansal riskini sermaye harcamasından (CAPEX) operasyonel harcamaya (OPEX) dönüştürür. Dağıtık bir mimari, veri işlemeyi ve depolamayı birden fazla düğüme dağıtarak yüksek performansın sürdürülmesini sağlar.

\subsection{Log Toplama, Normalizasyon ve Zenginleştirme}

SIEM'in analitik yeteneklerinin temelini, işlenen verinin kalitesi oluşturur. Bu kalite, log toplama, normalizasyon ve zenginleştirme süreçleriyle sağlanır.

\begin{itemize}
    \item \textbf{Log Toplama:} Bu ilk adımda, kuruluşun tüm altyapısından (sunucular, uygulamalar, ağ cihazları, güvenlik cihazları gibi) log verileri merkezi SIEM platformuna aktarılır.
    \item \textbf{Normalizasyon:} Farklı kaynaklardan gelen ham log verileri, farklı format ve şemalara sahiptir. Normalizasyon, bu dağınık veriyi SIEM'in kolayca analiz edebileceği tutarlı ve standart bir formata dönüştürür. Bu süreç, veri arama ve analizini hızlandırır, aynı zamanda depolama maliyetlerini optimize eder.
    \item \textbf{Zenginleştirme:} Normalleştirilmiş log verisine ek bağlamsal bilgi (kullanıcı kimliği, coğrafi konum, varlık bilgisi, tehdit istihbaratı) eklenmesidir. Örneğin, başarısız bir oturum açma girişiminin loguna, kaynağın bilinen kötü niyetli bir IP adresi olup olmadığı veya coğrafi konumunun anormal olup olmadığı bilgisi eklenir. Bu bağlamsal bilgi, analistlerin bir olayın gerçek bir tehdit olup olmadığını hızla belirlemesine yardımcı olur, böylece hatalı pozitifleri ve alarm yorgunluğunu azaltır.
\end{itemize}

\subsection{Korelasyon Kuralı Geliştirme ve Ayarlama}

Korelasyon kuralları, bir SIEM'in zekasını oluşturan mantık parçalarıdır. Tek başına zararsız görünen birden fazla olayı mantıksal olarak birleştirerek, karmaşık tehditleri (örneğin, brute-force saldırısı, ayrıcalık yükseltme) tespit etmek için kullanılır. Kurallar, olaylar arasında zaman pencereleri, belirli bir olay dizisi ve mantıksal koşullar (AND/OR) temel alınarak oluşturulur.

Kural geliştirme, bir organizasyonun kendine özgü "normal" trafiğini belirlemeyi ve eşik değerlerini buna göre sürekli olarak ayarlamayı gerektiren bir süreçtir. Eğer bir kural çok geniş tutulursa, normal aktiviteler yanlışlıkla uyarıları tetikleyebilir ve hatalı pozitiflere yol açabilir. Bu durum, "alarm yorgunluğunun" en yaygın kaynaklarından biridir. Bu nedenle, kural ayarlama (tuning), sürekli bir operasyonel görevdir. SIEM'in etkinliğini korumak için, ağda yeni cihazlar eklendikçe veya yazılım güncellemeleri yapıldıkça kuralların gözden geçirilmesi ve ayarlanması gerekir.

\subsection{Use Case Geliştirme ve Tespit Mühendisliği}

Tespit mühendisliği, reaktif bir "uyarı yanıtlayıcı" rolünden, stratejik bir "savunma oluşturucu" rolüne geçişin kurumsal mekanizmasıdır. Bu, tehdit istihbaratını (MITRE ATT\&CK gibi) kullanarak, bir organizasyon için en ilgili tehditlere karşı proaktif olarak tespitler oluşturma sürecidir.

Bu sürecin yaşam döngüsü aşağıdaki adımları içerir:
\begin{enumerate}
    \item \textbf{Tehdit Modelleme:} Kuruluşun tehdit profilini anlamak ve MITRE ATT\&CK çerçevesini kullanarak hangi tehdit aktörlerinin ve TTP'lerin en alakalı olduğunu belirlemekle başlar.
    \item \textbf{Veri Gereksinimleri:} Belirlenen tehditleri tespit etmek için hangi log ve telemetri verilerinin gerekli olduğunu saptamak.
    \item \textbf{Mantık Geliştirme:} SIEM'de veya diğer araçlarda, tespit mantığını kural veya makine öğrenimi modeli olarak yazmak.
    \item \textbf{Test ve Doğrulama:} Oluşturulan kuralın bir test ortamında, kontrollü saldırı senaryolarıyla (örneğin, kırmızı takım-mavi takım işbirliğiyle) çalıştığının doğrulanması.
    \item \textbf{Dağıtım ve İyileştirme:} Kuralı canlı ortama almak ve performansını izleyerek sürekli olarak ayarlamaktır.
\end{enumerate}

Bu döngü, bir SOC'nin rastgele uyarıları ele almaktan, iş hedefleri ve tehdit modeline dayalı olarak kasıtlı bir şekilde savunma yetenekleri oluşturmaya geçtiğini gösterir.

\subsection{SIEM Performans Optimizasyonu ve Depolama Yönetimi}

SIEM'in teknik performansı (sorgu hızı, veri alımı), doğrudan operasyonel performansı ve olay yanıt süresini etkiler. Bu, altyapı yönetiminin, SOC'nin başarısı için bir ön koşul olduğu anlamına gelir. Performans optimizasyonu, log normalizasyonu ve veri toplama süreçlerinin iyileştirilmesiyle başlar. Verilerin tutarlı bir formata dönüştürülmesi ve gereksiz alanların kaldırılması, sorgu performansını artırır ve depolama gereksinimlerini azaltır.

Depolama yönetimi, SIEM'in yüksek veri hacmiyle başa çıkabilmesi için hayati önem taşır. Veri yaşam döngüsü yönetimi stratejileri, sıcak, soğuk veya arşiv depolama katmanlarını kullanarak verilerin maliyet etkin bir şekilde saklanmasını sağlar. Eğer bir analist, basit bir sorgunun çalışması için dakikalarca beklemek zorunda kalırsa, bu operasyonel verimliliği düşürür ve olay çözümünü geciktirir.

\section{Güvenlik Orkestrasyonu, Otomasyonu ve Yanıtı (SOAR)}

SOAR, bir SOC'nin verimliliğini, hızını ve tutarlılığını dönüştüren, güvenlik operasyonlarını merkezileştiren ve otomatikleştiren bir platformdur.

\subsection{SOAR Platform Seçimi ve Uygulama}

SOAR, güvenlik orkestrasyonu, otomasyon ve yanıt yeteneklerini birleştirir. En büyük değeri, analistlerin tekrarlayan, manuel görevlerden kurtulmasını sağlaması, olay yanıt sürelerini (MTTR) kısaltması ve playbook'lar aracılığıyla yanıt süreçlerini standartlaştırmasıdır. SOAR'ın getirdiği verimlilik, analistlerin daha karmaşık ve stratejik görevlere (örneğin, proaktif tehdit avcılığı) odaklanmasını sağlar.

SOAR platformu seçimi, bir kuruluşun mevcut güvenlik araçlarıyla (SIEM, EDR vb.) kusursuz bir şekilde entegre olabilme yeteneğine bağlıdır. Platformun API yönetimini ve diğer araçlarla entegrasyonu ne kadar iyi desteklediği, SOAR'ın potansiyelini doğrudan belirler. SOAR, SIEM ile başlayan "görünürlük" yolculuğunu, "eyleme geçirilebilirlik" aşamasına taşıyan bir köprüdür.

\subsection{Playbook Geliştirme ve Otomatize İş Akışları}

Playbook'lar, belirli bir tehdit türüne (örneğin, kimlik avı) karşı izlenecek adımları tanımlayan, otomatikleştirilmiş iş akışlarıdır. Geliştirme süreci, en sık ve tekrarlayan görevlerin belirlenmesiyle başlar. Playbook'lar, saldırının tipine göre veri zenginleştirme, zararlı göstergeleri engelleme ve etkilenen uç noktaları izole etme gibi adımları içerir.

\textbf{Pratik Senaryo: Otomatik Kimlik Avı Yanıt Playbook'u}

\begin{enumerate}
    \item \textbf{Tetkik:} Bir çalışan, şüpheli bir e-postayı güvenlik ekibine bildirir veya bir güvenlik aracı kimlik avı girişimi tespit eder. Bu olay, playbook'u otomatik olarak tetikler.
    \item \textbf{Veri Ayıklama:} Playbook, e-postadan göstergeleri (URL, IP, dosya karması) otomatik olarak ayıklar.
    \item \textbf{Zenginleştirme:} Ayıklanan göstergeler, VirusTotal gibi üçüncü taraf tehdit istihbaratı araçlarıyla çapraz kontrol edilir ve e-postanın sahte olup olmadığı doğrulanır.
    \item \textbf{Otomatik Engelleme:} Eğer URL veya IP adresi kötü niyetli olarak doğrulanırsa, playbook ilgili güvenlik duvarı veya DNS ayarlarında bu göstergeleri otomatik olarak engeller.
    \item \textbf{Bildirim ve Vaka Oluşturma:} Playbook, olayı analiz etmesi için bir analist için otomatik olarak bir vaka yönetim sisteminde bir ticket oluşturur ve topladığı tüm bağlamsal bilgileri (veri zenginleştirmesi, otomatik eylemler) bu ticketa ekler.
\end{enumerate}

Bu süreç, manuel olarak saatler sürebilecek bir görevi saniyeler içinde tamamlayarak analistlerin iş yükünü büyük ölçüde azaltır ve yanıt süresini kısaltır.

\subsection{Güvenlik Aracı Entegrasyonu ve API Yönetimi}

SOAR'ın gerçek değeri, diğer güvenlik araçlarıyla entegrasyon yeteneğine bağlıdır. SIEM, EDR, güvenlik duvarları ve IAM gibi araçların SOAR ile entegrasyonu, playbook'ların bu araçlar üzerinde eylemler gerçekleştirmesini sağlar. Bu entegrasyonların temelini API'lar oluşturur.

API'lar, güvenlik araçları arasındaki iletişim için kritik bir altyapıdır. API güvenliği (kimlik doğrulama, yetkilendirme) ve yaşam döngüsü yönetimi, bu entegrasyonun güvenilirliği için hayati önem taşır. Güvenlik operasyonlarında API yönetimi, yalnızca işlevselliği sağlamakla kalmaz, aynı zamanda API'ları siber saldırılara karşı koruyarak veri güvenliğini de sağlar.

\subsection{Vaka Yönetimi ve Ticket Sistemi Entegrasyonu}

SOAR, olayları otomatik olarak ele alırken, insan ekiplerle senkronize çalışabilmek için vaka yönetimi ve ticket sistemleri (örneğin, Jira, ServiceNow) ile entegre olur. Bu entegrasyon, otomatikleştirilmiş yanıt süreçlerinin ilerlemesini takip etmek ve analistlerin gözetimini sağlamak için kritik öneme sahiptir.

Playbook'lar, bir olay tetiklendiğinde otomatik olarak bir ticket oluşturabilir ve bu ticketa bağlamsal bilgileri (loglar, tehdit göstergeleri, otomatik eylemler) ekleyebilir. Bu, analistlerin manuel olarak veri toplama ve raporlama ihtiyacını ortadan kaldırır. Bu entegrasyon, otomatikleştirilmiş yanıt ile insan gözetimi ve raporlaması arasındaki döngüyü tamamlar, böylece olayların tek bir yerden takibini ve anahtar metriklerin (MTTR) kolayca ölçülmesini sağlar.

\subsection{Otomasyonun ROI Hesaplanması ve Süreç Optimizasyonu}

SOAR yatırımlarının geri dönüşü (ROI), basit bir maliyet tasarrufu hesaplamasının ötesindedir. Gerçek değeri, insan sermayesinin optimizasyonu ve reaktif bir iş gücünün proaktif bir ekibe dönüştürülmesi yeteneğidir.

SOAR'ın ROI'si, manuel ve otomatik süreçler arasında kaydedilen zamana dayalı olarak hesaplanabilir. Ancak, otomasyonun asıl değeri, manuel iş yükünün azalmasının yanı sıra, analistlerin daha karmaşık ve stratejik görevlere (örneğin, tehdit avcılığı) odaklanmasını sağlamasıdır. Otomasyonla boşalan zaman, operasyonel işlerden tehdit avcılığı ve güvenlik mimarisi geliştirmeleri gibi daha değerli işlere yönlendirilebilir. Bu, SOAR'ın yalnızca para tasarrufu sağlamadığını, aynı zamanda mevcut analistlerin değerini ve becerilerini artırdığını gösterir.

\section{Tehdit Tespit Mühendisliği ve Tehdit Avcılığı Operasyonları}

Bu bölüm, modern bir SOC'nin en proaktif iki fonksiyonunu, tespit mühendisliğini ve tehdit avcılığını derinlemesine ele almaktadır.

\subsection{Tespit Kullanım Senaryosu Geliştirme Yaşam Döngüsü}

Tehdit tespit mühendisliği, tehdit aktörlerinin taktiklerini, tekniklerini ve prosedürlerini (TTP'ler) anlamak ve bunlara karşı etkili savunmalar oluşturmak için disiplinli bir yaklaşımdır. Bu süreç, reaktif bir "uyarı yanıtı" kültüründen, aktif olarak "tehditleri tasarlayıp engelleme" kültürüne geçişi temsil eder.

Yaşam döngüsü şu adımları içerir:
\begin{enumerate}
    \item \textbf{Tehdit Modelleme:} Kuruluş için en alakalı tehditlerin MITRE ATT\&CK çerçevesi kullanılarak belirlenmesi.
    \item \textbf{Veri Kaynağı Gereksinimleri:} Tespiti gerçekleştirmek için hangi log ve telemetri verilerinin gerekli olduğunun saptanması.
    \item \textbf{Mantık Geliştirme:} SIEM'de veya diğer araçlarda, tespit mantığını kural veya makine öğrenimi modeli olarak yazmak.
    \item \textbf{Test ve Doğrulama:} Kuralın bir test ortamında, kontrollü saldırı senaryolarıyla (örneğin, kırmızı takım-mavi takım işbirliğiyle) çalıştığının doğrulanması.
    \item \textbf{Ayarlama ve Sürekli İyileştirme:} Hatalı pozitifleri azaltmak ve yeni tehditlere uyum sağlamak için kuralın sürekli ayarlanması.
\end{enumerate}

Bu döngü, bir SOC'nin sürekli gelişen tehdit ortamına uyum sağlamasını ve reaktif yeteneklerini proaktif stratejilerle desteklemesini sağlar.

\subsection{Davranışsal Analiz ve Anomali Tespit Kuralları}

Davranışsal analiz, kullanıcı ve varlıkların "normal" davranışlarını öğrenen ve bu normal profillerden sapmaları (anomalileri) tespit ederek tehditleri ortaya çıkaran bir teknolojidir. Geleneksel kural tabanlı sistemler, yalnızca bilinen imzaları veya kalıpları arar. Davranışsal analiz ise, bilinen imzası olmayan veya henüz bilinmeyen tehditleri (sıfırıncı gün saldırıları, iç tehditler) tespit etme yeteneği sunar.

Çalışma mekanizması, istatistiksel analiz ve makine öğrenimi algoritmalarına dayanır. Örneğin, bir çalışanın normalde erişmediği dosyalara erişmesi veya alışılmadık saatlerde oturum açması gibi davranışsal anormallikler, sistem tarafından hemen işaretlenir. Ancak, bu teknolojinin en büyük zorluğu "normal"in tanımını yapmak ve hatalı pozitif riskini yönetmektir. Bu, sistemin sürekli olarak ince ayar yapılmasını ve evrilen tehditlere adapte olmasını gerektirir.

\subsection{Tehdit Avcılığı Programı Uygulaması}

Tehdit avcılığı, otomatik araçlar tarafından gözden kaçırılan veya bilinmeyen tehditleri proaktif olarak arayan, insan odaklı bir faaliyettir. Amacı, reaktif olay yanıtını tamamlayarak, kuruluşu daha siber dirençli hale getirmektir. Tehdit avcılığı, bir SOC'nin mevcut tespit yeteneklerini test eden ve doğrulayan bir kalite güvence (QA) süreci olarak da hizmet eder.

Bir tehdit avcılığı programı şu temel adımları içerir:
\begin{enumerate}
    \item \textbf{Hipotez Geliştirme:} Bir araştırma sorusu veya hipoteziyle başlama.
    \item \textbf{Veri Toplama:} Hipotezi test etmek için gerekli logları ve telemetri verilerini toplama.
    \item \textbf{Soruşturma ve Analiz:} Toplanan veri üzerinde desen ve anormallik arama.
    \item \textbf{Doğrulama ve Yanıt:} Tehdidin varlığını doğrulama ve olay yanıt ekibine eskalasyon.
\end{enumerate}

\subsection{Avlanma Hipotezi Geliştirme ve Doğrulama}

Tehdit avcılığının en önemli adımı, araştırmayı yönlendiren, spesifik ve test edilebilir bir hipotez oluşturmaktır. Hipotez, tehdit istihbaratına, önceki olaylara veya SIEM/UEBA tarafından belirlenen anormal bir aktiviteye dayanabilir.

Bir hipotezi detaylandırmak için ABLE (Actor, Behavior, Location, Evidence - Aktör, Davranış, Konum, Kanıt) çerçevesi kullanılabilir. Örneğin, "Bir tehdit aktörü, DNS tünelleme kullanarak hassas finansal verileri sızdırıyor olabilir" gibi spesifik bir hipotez, tehdit avcısını doğru veri kaynaklarına ve belirli bir hedefe yönlendirerek boşa harcanan çabayı azaltır. Bu, sınırlı SOC kaynaklarının en etkili şekilde kullanılmasını sağlar.

\subsection{Tespit Kapsamı Değerlendirmesi ve Boşluk Analizi}

Tespit kapsama değerlendirmesi, bir SOC'nin mevcut güvenlik kontrollerinin, MITRE ATT\&CK gibi bir çerçeveye göre ne kadar etkili olduğunu değerlendirme sürecidir. Bu süreç, log kaynağı yapılandırma hataları, bozuk log toplayıcılar veya yetersiz kural setleri nedeniyle oluşan tespit boşluklarını belirlemeye yardımcı olur.

Bu analizin amacı \%100 kapsama ulaşmak değil, kuruluşun tehdit modeline göre en kritik tehditleri önceliklendirmektir. Bu süreç, soyut kapsama verilerini somut metriklere dönüştürür ve gelecekteki güvenlik yatırımlarını riske dayalı verilere göre gerekçelendirmek için kullanılır.

\section{SOC Performans Yönetimi ve Sürekli İyileştirme}

Bir SOC'nin performansı, operasyonel verimliliğini nasıl ölçtüğüne ve sürekli olarak nasıl artırdığına bağlıdır.

\subsection{SOC Metrik Geliştirme: Verimlilik, Etkililik, Kalite}

SOC metrikleri, operasyonel hedeflere ne kadar ulaşıldığını gösteren nicel ölçümlerdir.

\begin{longtable}{|p{2.5cm}|p{3cm}|p{3cm}|p{3.5cm}|}
\hline
\textbf{Metrik} & \textbf{Tanım} & \textbf{Hesaplama Formülü} & \textbf{Hedef Değer Aralığı} \\
\hline
\textbf{MTTD} & Bir tehdidin oluştuğu andan tespit edildiği ana kadar geçen ortalama süre. & (Uyarı Oluşturma Zamanı) - (Etkinlik Başlangıç Zamanı) & 30 dakika - 4 saat \\
\hline
\textbf{MTTR} & Tehdide müdahale edilmesi için geçen ortalama süre. & (Kontrol Altına Alma Zamanı) - (Tespit Zamanı) & 2 saat - 4 saat \\
\hline
\textbf{MTTI} & Bir uyarının doğrulanması ve incelenmesi için geçen ortalama süre. & (Araştırma Başlangıç Zamanı) - (Uyarı Zamanı) & Değişken \\
\hline
\textbf{FPR} & Yanlışlıkla tehdit olarak etiketlenen uyarıların yüzdesi. & (Hatalı Pozitif Sayısı) / (Toplam Uyarı Sayısı) & Kritik: $<$\%25, Yüksek: $<$\%50 \\
\hline
\end{longtable}

Bu metrikler, SOC'nin kendi iç sağlığını ölçmesi ve iş hedeflerine nasıl katkıda bulunduğunu kanıtlaması için temel araçlardır. Yüksek bir MTTD, saldırganların sistemde kalış süresinin arttığını ve dolayısıyla iş riskinin yüksek olduğunu gösterir. Bu metrikler, sadece raporlama aracı olarak değil, aynı zamanda kök neden analizi ve iyileştirme için de kullanılır.

\subsection{Alarm Yorgunluğu Azaltma ve Hatalı Pozitif Yönetimi}

Alarm yorgunluğu, çok sayıda anlamsız uyarı nedeniyle analistlerin gerçek tehditleri gözden kaçırması ve tükenmesidir. Bu durum, analist moralini düşürür ve doğrudan bir güvenlik ihlali riskine yol açar.

Bu sorunu çözmek için aşağıdaki teknikler uygulanabilir:
\begin{itemize}
    \item \textbf{Net Tanımlama:} Yalnızca acil eylem gerektiren uyarıları gerçek olarak kabul etmek ve diğerlerini raporlara kaydetmek.
    \item \textbf{Kural Temizleme:} Kullanılmayan veya gereksiz varsayılan kuralları devre dışı bırakmak.
    \item \textbf{Ortama Özel Ayarlama:} Kural eşik değerlerini organizasyonun "normal" trafiğine göre ayarlamak. Bu, bir ağın temelini (baseline) belirlemeyi gerektirir.
    \item \textbf{Bağlam Kullanımı:} Varlık bilgisi ve tehdit istihbaratı gibi bağlamsal verilerle uyarıları zenginleştirmek. Bu, bir SQL saldırısı uyarısının, hedef sunucunun SQL çalıştırmadığı bilgisiyle otomatik olarak hatalı pozitif olarak belirlenmesini sağlayabilir.
\end{itemize}

\subsection{Analist Eğitimi ve Beceri Geliştirme Programları}

Siber tehdit ortamı sürekli evrildiği için analistlerin becerileri de sürekli güncellenmelidir. Analist eğitimi ve beceri geliştirme, bir SOC'nin uzun vadeli etkinliği için bir ek fayda değil, zorunlu bir operasyonel gerekliliktir.

Programlar, Microsoft Learn gibi platformlardan yapılandırılmış öğrenme yolları, L3 analistlerinin L1 ve L2 analistlerine mentorluk yapması ve proaktif tehdit avcılığı gibi uygulamalı alıştırmaları içermelidir. Teknoloji ne kadar gelişirse gelişsin, insan unsuru, başarılı bir SOC'nin en önemli faktörü olmaya devam etmektedir.

\subsection{SOC Araç Konsolidasyonu ve Teknoloji Yol Haritası}

Kuruluşların ortalama 60-75 güvenlik aracı kullandığı, bu durumun "araç karmaşası"na (tool sprawl) ve veri silolarına yol açtığı belirtilir. Bu karmaşa, yönetim zorluğunu ve lisanslama maliyetlerini artırır ve kör noktalar oluşturur.

Araç konsolidasyonu, bu soruna stratejik bir yanıttır. Araçları birleştirerek, yönetim karmaşıklığı azaltılır ve "tek bir cam bölmede" görünürlük artırılır. Bu süreç, reaktif, duruma göre güvenlik ürünü satın alma yaklaşımını bırakıp, daha uyumlu ve entegre bir güvenlik mimarisi oluşturmayı hedefler.

\subsection{Yönetilen Güvenlik Hizmeti Sağlayıcı (MSSP) Değerlendirmesi}

Dış kaynak kullanımına karar verildiğinde, bir MSSP'yi değerlendirme, bir hizmeti satın almaktan daha fazlasıdır; kontrol, esneklik ve maliyet arasında dikkatli bir denge kurmayı gerektiren stratejik bir ortaklık kararıdır.

Değerlendirme, sağlayıcının sunduğu SLA'ları, tehdit avcılığı ve adli bilişim gibi hizmetlerin derinliğini, kuruluşun iş modelini ve sektörünü ne kadar anladığını içermelidir. Bir sağlayıcının hizmetinin sözleşme koşullarıyla sınırlı olması, beklenmedik tehditlere karşı esnekliği sınırlayabilir. Bu nedenle, değerlendirme sürecinin, sağlayıcının teknolojik yeteneklerinin ötesinde, olası operasyonel riskleri de kapsaması önemlidir.

\section{Güvenlik Operasyonlarında Yükselen Teknolojiler}

Bu bölüm, güvenlik operasyonlarının geleceğini şekillendiren yeni ve gelişmekte olan teknolojileri incelemektedir.

\subsection{SOC Operasyonlarında Yapay Zeka (AI)/Makine Öğrenimi (ML) Entegrasyonu}

AI ve ML, SOC için dönüştürücü bir güçtür. Bu teknolojiler, geleneksel kural tabanlı sistemlerin atlayabileceği ince anomalileri ve desenleri tespit ederek tehdit algılamayı önemli ölçüde geliştirir. Aynı zamanda, rutin görevleri (uyarı triyajı gibi) otomatikleştirerek analistlerin iş yükünü hafifletir. AI/ML'nin en önemli özelliklerinden biri, sürekli öğrenme ve adapte olma yeteneğidir. Bu, algoritmaların tehdit ortamı geliştikçe kendilerini sürekli olarak güncellemesini sağlar. Ayrıca, hatalı pozitifleri filtrelemek için geçmiş olaylardan öğrenerek alarm yorgunluğunu azaltır.

\subsection{Kullanıcı ve Varlık Davranış Analizi (UEBA) Uygulaması}

UEBA, kullanıcıların, sunucuların ve diğer varlıkların normal davranışlarını profillendirerek, bunlardan sapmaları tespit eder. Bu teknoloji, kural tabanlı sistemlerin zayıf olduğu bir alanda, yani bir iç tehdidin veya ele geçirilmiş bir hesabın normal görünen davranışlarını tespit etmede kritik bir rol oynar. UEBA, iç tehditlerin, hesap ele geçirmelerinin ve geleneksel araçlar tarafından gözden kaçırılan gelişmiş kalıcı tehditlerin (APT'ler) tespiti için kullanılır.

\subsection{Bulut-Yerel Güvenlik Operasyonları}

Bulut-yerel mimariler (mikro hizmetler, konteynerler), güvenliği ağa dayalı bir yaklaşımdan, kimlik ve iş yükü odaklı bir yaklaşıma kaydırır. Bu değişim, SOC'nin bulut ortamlarındaki dinamik ve kısa ömürlü varlıkları izlemesini gerektirir. Bu, Kimlik ve Erişim Yönetimi (IAM), İş Yükü Güvenliği ve sürekli izleme gibi yeni odak alanları yaratır. Geleneksel güvenlik, bir duvarı korumaya odaklanırken, bulut güvenliği, duvarın içinde sürekli olarak her işlemi doğrulamaya odaklanır.

\subsection{DevSecOps'un Güvenlik Operasyonları ile Entegrasyonu}

DevSecOps, güvenliği yazılım geliştirme yaşam döngüsünün her aşamasına (geliştirme, test, dağıtım) entegre eden kültürel ve pratik bir yaklaşımdır. Bu entegrasyon, geliştirme, operasyon ve güvenlik ekipleri arasındaki siloları yıkar. Güvenlik açıklarını, bir ürün SOC'ye gelmeden önce geliştirme aşamasında bulup düzeltir. Bu proaktif yaklaşım, tehditleri algılamak ve bunlara yanıt vermek yerine, tehditleri oluştukları yerde önleyerek SOC'nin iş yükünü en temelden azaltan en güçlü stratejilerden biridir.

\subsection{Sıfır Güven (Zero Trust) Mimarisi ve SOC Operasyonlarına Etkisi}

Sıfır Güven, "Asla güvenme, her zaman doğrula" ilkesine dayanan bir güvenlik modelidir. Ağın içindeki hiçbir kullanıcıya veya cihaza varsayılan olarak güvenilmez. Bu mimari, SOC'nin görevini daha zorlu ve karmaşık hale getirirken, aynı zamanda iç tehditlere karşı daha dirençli olmasını sağlar.

Sıfır Güven, SOC'nin odak noktasını "ağ sınırını savunmak"tan, "içerideki her işlemi sürekli olarak izlemek ve doğrulamak"a kaydırır. Bu yaklaşım, her işlem loglandığı ve izlendiği için veri hacminde katlanarak artışa neden olur. Bu durum, UEBA ve Gelişmiş Tehdit Tespiti (AI/ML) gibi teknolojilerin, anormal iç hareketleri tespit etmek için temel gereksinim haline geldiğini gösterir. SOC, artık sadece dış tehditlerle değil, aynı zamanda içeriden gelen tehditlerle ve yanal hareketlerle de daha fazla mücadele etmek zorunda kalır. Sıfır Güven felsefesi, kural tabanlı sistemlerin her olası iç tehdit senaryosunu kapsayamayacağı gerçeğinden hareketle, davranışsal analize dayalı teknolojilerin önemini artırır.

\end{itemize}