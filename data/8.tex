\chapter{SİBER TEHDİT İSTİHBARATI VE TEHDİT AVCILIĞI}

\section*{Giriş}
Siber tehdit istihbaratı ve tehdit avcılığı, organizasyonların proaktif güvenlik duruşu geliştirmesi için kritik disiplinlerdir. Bu bölümde modern tehdit istihbaratı, sürekli tehdit maruziyeti yönetimi (CTEM) ve tehdit avcılığı metodolojilerini ele alacağız.

NIST Siber Güvenlik Çerçevesi ve Sürekli Tehdit Maruziyeti Yönetimi (CTEM), organizasyonların tehdit ortamını sürekli olarak değerlendirmesi ve uygun önlemleri alması için entegre bir yaklaşım sağlar.

Modern siber güvenlik operasyonlarının proaktif boyutunu oluşturan bu disiplinler, tehdit istihbaratı toplama, analiz etme ve uygulama süreçlerini detaylı bir şekilde ele alır.

\section{Cyber Threat Intelligence (CTI) Fundamentals}

\subsection{Diamond Model ve Kill Chain Analysis}

Siber saldırıları anlamak ve analiz etmek için kullanılan iki önemli analitik çerçeve, Lockheed Martin'in Siber Kill Chain'i ve MITRE'nin Saldırı Analizının Elmas Modeli'dir (Diamond Model of Intrusion Analysis). Bu modeller, saldırı sürecini farklı açılardan ele alarak birbirini tamamlar.

\textbf{Lockheed Martin Siber Saldırı Zinciri (Cyber Kill Chain):} Bu model, siber saldırıları yedi aşamaya böler: Keşif, Silahlandırma, Teslimat, Sömürü, Kurulum, Komuta ve Kontrol, ve Hedefe Ulaşma. Her aşama, savunma ekiplerinin saldırıyı durdurabileceği bir fırsat sunar.
Siber Tehdit İstihbaratı (CTI), bir kuruluşun siber tehditlere karşı savunma yeteneklerini güçlendirmek için toplanan, işlenen ve analiz edilen bilgidir. CTI, bir kuruluşun potansiyel saldırganları, onların motivasyonlarını, yeteneklerini ve kullandıkları altyapıyı anlamasına yardımcı olur. Bu bilgiler, güvenlik ekiplerinin proaktif bir savunma stratejisi oluşturmasına, olay müdahale süreçlerini iyileştirmesine ve kaynaklarını daha etkili bir şekilde tahsis etmesine olanak tanır.

\subsection{Threat Intelligence Lifecycle ve Collection Methods}

Tehdit istihbaratı yaşam döngüsü, ham veriyi değerli istihbarata dönüştüren ve sürekli bir geri bildirim döngüsü ile kendini yenileyen, yapılandırılmış bir süreçtir. Bu döngü, CTI programının temelini oluşturur ve reaktif bir modelden proaktif bir savunma yaklaşımına geçişin temelini atar.

\textbf{Yaşam Döngüsü Aşamaları:}

\begin{enumerate}
    \item \textbf{Gereksinimler (Requirements):} Döngünün ilk ve en kritik aşamasıdır. Bu aşamada, güvenlik ekipleri, iş birimleri ve üst düzey yöneticiler bir araya gelerek istihbarat ihtiyaçlarını net bir şekilde tanımlar. Bu, korunması gereken en kritik varlıkları (“Crown Jewels”), organizasyonun karşılaştığı riskleri ve bu riskleri azaltmak için hangi bilgilerin gerekli olduğunu belirlemeyi içerir. Gereksinimlerin net bir şekilde belirlenmesi, istihbarat toplama çabalarının boşa gitmesini engeller ve kaynakların doğru hedeflere yönlendirilmesini sağlar. Aksi halde, ekipler alakasız verileri takip ederek zaman ve kaynak kaybedebilir veya kritik tehditleri gözden kaçırabilir.
    \item \textbf{Toplama (Collection):} Bu aşama, tanımlanan gereksinimleri karşılamak için geniş bir yelpazedeki kaynaklardan ham veri toplamayı içerir. Toplanan veriler hem teknik (Indicator of Compromise - IOC) hem de bağlamsal bilgiler (TTP’ler, motivasyonlar) içermelidir.
    \begin{itemize}
        \item \textbf{OSINT (Open-Source Intelligence):} Genel kullanıma açık ve serbestçe erişilebilen kaynaklardan bilgi toplama. Bu kaynaklar arasında haberler, bloglar, sosyal medya platformları, akademik çalışmalar ve endüstri raporları bulunur.
        \item \textbf{Ticari Tehdit Beslemeleri:} Güvenlik firmaları tarafından sağlanan ve genellikle otomatik sistemlere entegre edilen ücretli veri akışlarıdır. Bu beslemeler, binlerce IOC ve TTP bilgisini yüksek hacimde sunabilir.
        \item \textbf{Bilgi Paylaşım Toplulukları (ISACs):} Aynı sektördeki veya coğrafi bölgedeki organizasyonların tehdit istihbaratı paylaşımı için bir araya geldiği güvenilir platformlardır.
        \item \textbf{Dahili Kaynaklar:} Organizasyonun kendi güvenlik araçlarından (SIEM, EDR, IDS/IPS), ağ günlüklerinden, uç nokta telemetri verilerinden ve kimlik doğrulama kayıtlarından elde edilen verilerdir.
        \item \textbf{Derin ve Karanlık Web İzleme:} Gizli veya şifreli forumlar, siber suç pazarları ve sızdırılmış veri depolarından (örneğin, sızan kimlik bilgileri, saldırı planları) bilgi toplama.
    \end{itemize}
    \item \textbf{İşleme (Processing):} Toplanan ham verinin analiz edilebilir, yapılandırılmış ve temiz bir formata dönüştürüldüğü aşamadır. Bu, veri normalizasyonu (farklı formatlardaki verileri standartlaştırma), tekilleştirme (yinelenen kayıtları kaldırma), şifre çözme ve etiketleme işlemlerini içerir. Büyük veri kümeleri için otomasyon, bu aşamada hayati öneme sahiptir.
    \item \textbf{Analiz ve Yorumlama (Analysis and Interpretation):} İşlenmiş verinin anlamlı, eyleme dönüştürülebilir istihbarata dönüştüğü aşamadır. Analistler, kalıpları, eğilimleri, saldırı kampanyalarını ve potansiyel tehditleri belirlemek için verileri derinlemesine incelerler. Bu aşamada, insan uzmanlığı ve otomatik korelasyon mekanizmaları birleşir.
    \item \textbf{Yayma (Dissemination):} Analiz edilen istihbaratın ilgili paydaşlara zamanında ve uygun bir formatta sunulmasıdır. İstihbaratın formatı ve içeriği, hedef kitleye göre uyarlanmalıdır. Örneğin, CISO'lar için üst düzey yönetici brifingleri hazırlanırken, SOC analistleri için teknik IOC beslemleri sunulur.
    \item \textbf{Geri Bildirim (Feedback):} Sürecin sürekli iyileştirilmesini sağlayan bu aşamada, istihbarat tüketicileri (örneğin, SOC analistleri), sunulan bilgilerin yararlılığı, alaka düzeyi ve doğruluğu hakkında geri bildirimde bulunur. Bu geri bildirim, bir sonraki döngüde daha iyi gereksinimlerin belirlenmesini ve istihbarat toplama stratejilerinin hassaslaştırılmasını sağlar.
\end{enumerate}

Bu yaşam döngüsü, bir organizasyonun reaktif tehdit algılama modelinden proaktif bir savunma duruşuna geçiş yapmasına olanak tanır. Bu sürekli ve döngüsel süreç, savunma mekanizmalarının saldırganların evrilen taktik, teknik ve prosedürlerine (TTP'ler) göre dinamik olarak ayarlanmasını sağlar. Bu dinamik adaptasyon, bir saldırganın bir sistemde kalma süresini (dwell time) doğrudan azaltır. Saldırganın sistemde kalma süresinin azalması, fidye yazılımı veya veri sızdırma gibi nihai hedeflere ulaşma şansını düşürürken, ihlalden kaynaklanan maliyetleri de önemli ölçüde azaltır. Bu sürekli öğrenme ve adaptasyon mekanizması, tehdit istihbaratı programının değerini ve etkinliğini katlanarak artırır.

\textbf{CTI Yaşam Döngüsü Aşamaları ve Uygulamaları}

\begin{longtable}{|p{2.5cm}|p{3cm}|p{3cm}|p{3.5cm}|}
\hline
\textbf{Aşama} & \textbf{Kısa Tanım} & \textbf{Hedef} & \textbf{Pratik Uygulama} \\
\hline
\textbf{Gereksinimler} & İhtiyaç duyulan istihbaratın belirlenmesi & Kaynakları kritik risklere odaklamak & Tehdit aktörü TTP analizi \\
\hline
\textbf{Toplama} & Ham verinin kaynaklardan alınması & Gereksinim odaklı veri toplama & SIEM günlükleri, OSINT, karanlık web \\
\hline
\textbf{İşleme} & Ham veriyi analiz formatına getirme & Veri yapılandırma ve zenginleştirme & CSV normalizasyonu, VirusTotal entegrasyonu \\
\hline
\textbf{Analiz} & Veriden içgörüler çıkarma & Tehdit bağlamsallaştırma & Oltalama kampanya teknik analizi \\
\hline
\textbf{Yayma} & İstihbaratı paydaşlara iletme & Doğru kitleye doğru formatta sunum & Yönetici brifingi, SOC beslemesi \\
\hline
\textbf{Geri Bildirim} & Etkinlik geri bildirimi toplama & Sürekli iyileştirme & Yanlış pozitif raporları, değerlendirmeler \\
\hline
\end{longtable}

\subsection{Strategic, Tactical, Technical ve Operational Intelligence}

Siber tehdit istihbaratı, hedef kitlenin ihtiyaçlarına göre dört ana kategoriye ayrılır. Bu istihbarat türleri, bir organizasyonun savunma yeteneklerini tüm seviyelerde güçlendirmek için birlikte çalışır.

\begin{itemize}
    \item \textbf{Stratejik İstihbarat:} Genel tehdit ortamına ilişkin üst düzey bir bakış açısı sunar. Bu istihbarat, teknoloji dışı terimlerle hazırlanır ve öncelikli olarak üst düzey yöneticilere, CISO'lara ve risk yöneticilerine yöneliktir.
    \begin{itemize}
        \item \textbf{Amacı:} Güvenlik yatırımları, bütçe tahsisi ve kurumsal politikalar gibi uzun vadeli stratejik kararlara rehberlik etmektir.
        \item \textbf{Örnekler:} Bir APT grubunun bir sektörü veya belirli bir coğrafyayı hedeflemesi, fidye yazılımı eğilimleri veya jeopolitik olayların siber saldırı risklerine etkisi hakkında raporlar.
    \end{itemize}
    \item \textbf{Operasyonel İstihbarat:} Belirli bir tehdit aktörünün TTP'lerini, motivasyonlarını ve altyapısını anlamayı sağlar. Bu istihbarat, olay müdahale (IR) ekipleri ve tehdit avcıları için hayati öneme sahiptir.
    \begin{itemize}
        \item \textbf{Amacı:} Yaklaşan veya devam eden bir saldırının "kim, ne zaman, nerede, nasıl, neden" sorularına bağlamsal cevaplar sunmak. Bu, proaktif tehdit avcılığı ve olay müdahale planlaması için bir temel oluşturur.
        \item \textbf{Örnekler:} Bir saldırıda kullanılan belirli bir oltalama e-postası kampanyasının detayları veya bir Komuta ve Kontrol (C2) sunucusunun iletişim yöntemleri.
    \end{itemize}
    \item \textbf{Taktik İstihbarat:} Ağ ve uç noktalarda tehditleri tespit etmeye yardımcı olan, genellikle kısa ömürlü ve teknik göstergelerden oluşan bir istihbarat türüdür.
    \begin{itemize}
        \item \textbf{Amacı:} Otomatik tehdit tespiti için SIEM ve güvenlik duvarı gibi güvenlik kontrollerine entegre edilmek.
        \item \textbf{Örnekler:} Kötü amaçlı IP adresleri, dosya karmaları (hashes), kötü amaçlı alan adları.
    \end{itemize}
    \item \textbf{Teknik İstihbarat:} Taktik istihbaratın daha derin teknik detaylarını içerir ve genellikle kötü amaçlı yazılım analizi, tersine mühendislik ve IoC'lerin oluşturulmasını kapsar.
    \begin{itemize}
        \item \textbf{Amacı:} Saldırıların teknik işleyişini anlamak ve bu bilgilere dayanarak yeni imza tabanlı tespit kuralları oluşturmaktır.
    \end{itemize}
\end{itemize}

Bu dört istihbarat türü, birbirini besleyen ve destekleyen bir hiyerarşi içinde çalışır. Üst yönetim, genel güvenlik stratejisini belirlemek için stratejik istihbarata ihtiyaç duyar. Bu strateji, operasyonel ve taktiksel gereksinimlere dönüşür. Örneğin, stratejik istihbarat, jeopolitik gerilimlerin bir APT grubunun faaliyetlerini artıracağını gösterdiğinde, operasyonel ekip bu grubun TTP'lerini incelemeye odaklanır ve bu operasyonel bilgiler taktiksel IoC'lere dönüştürülerek güvenlik kontrollerine entegre edilir. Bu entegrasyon, istihbaratın tüm organizasyonel katmanlarda değer yaratmasını sağlar.

\textbf{İstihbarat Türleri Karşılaştırma Tablosu}

\begin{longtable}{|p{2cm}|p{3.5cm}|p{3.5cm}|p{3cm}|}
\hline
\textbf{Özellik} & \textbf{Stratejik} & \textbf{Operasyonel} & \textbf{Taktik} \\
\hline
\textbf{Hedef Kitle} & C-seviyesi yöneticiler & IR ekipleri, tehdit avcıları & SOC analistleri \\
\hline
\textbf{Odak Noktası} & Genel eğilimler, risk yönetimi & Saldırı kampanyaları, TTP'ler & IOC'ler, ağ etkinlikleri \\
\hline
\textbf{Zaman} & Uzun vadeli (aylar/yıllar) & Orta vadeli (hafta/ay) & Kısa vadeli (gerçek zamanlı) \\
\hline
\textbf{Amaç} & Politika ve yatırım belirleme & Müdahale ve proaktif avcılık & Otomatik tespit ve engelleme \\
\hline
\end{longtable}

\subsection{Threat Actor Profiling ve Attribution Challenges}

Tehdit aktörü profilleme, saldırıların arkasındaki "kim" sorusuna cevap aramayı amaçlar. Bir tehdit aktörü profili, sadece bir isimden ibaret değildir; bir saldırganın kimliğini, hedeflerini, TTP'lerini, motivasyonlarını, coğrafi konumunu ve kullandığı altyapıyı kapsamlı bir şekilde analiz eder. Bu profiller, güvenlik ekiplerine saldırganın olası davranışları hakkında bir resim sunar ve savunma stratejilerini saldırganın niyetleriyle uyumlu hale getirmeye yardımcı olur.

Ancak, bir siber saldırıyı kesin olarak belirli bir tehdit aktörüne atfetmek (attribution), son derece karmaşık ve zorlu bir süreçtir. Bu zorlukların birkaç temel nedeni vardır:

\begin{itemize}
    \item \textbf{Gizleme ve Obfüskasyon:} Saldırganlar, kimliklerini gizlemek için çok sayıda katman kullanır. Botnetler, vekil sunucular ve kiralanmış altyapı, saldırının gerçek kaynağını maskeler. Özellikle dağıtık servis dışı bırakma (DDoS) saldırılarında, trafik binlerce farklı cihazdan gelebilir, bu da kesin bir atıf yapmayı neredeyse imkansız hale getirir.
    \item \textbf{Yanlış Bayrak Operasyonları:} Bazı siber suç grupları, dikkat çekmek veya yanlış bilgi yaymak amacıyla gerçekleştirmediği saldırıların sorumluluğunu üstlenebilir. Bu tür eylemler, atıf sürecini manipüle etmeyi amaçlar.
    \item \textbf{Araç ve TTP Paylaşımı:} Tehdit grupları, sıklıkla araç setlerini ve TTP'lerini birbirleriyle paylaşır veya değiştirir. Bir saldırıda kullanılan belirli bir teknik, daha önce bilinen bir grubun imzası olsa bile, başka bir grup tarafından da kullanılmış olabilir. Bu durum, yalnızca teknik IOC'lere dayalı atıf yapmanın güvenilirliğini azaltır.
\end{itemize}

Bu nedenle, atıf birincil hedef olmamalıdır. Bir saldırı meydana geldiğinde, ilk ve en acil öncelik, hasarı durdurmak, sistemleri güvence altına almak ve devam eden tehditleri ortadan kaldırmaktır. Atıf, bu acil riskler ortadan kaldırıldıktan sonra, olay sonrası analiz aşamasında daha sonraki savunmaları güçlendirmek için bir araç olarak ele alınmalıdır. Atıf, bir "evet/hayır" cevabı yerine, farklı güven seviyelerine sahip (düşük, orta, yüksek) analitik bir değerlendirme süreci olarak görülmelidir. Güvenilir bir atıf için, sadece IOC'ler yerine, davranışsal kanıtlar ve TTP'lere odaklanmak daha geçerli bir yaklaşımdır.

\subsection{Diamond Model ve Kill Chain Analysis}

Siber saldırıları anlamak ve analiz etmek için kullanılan iki önemli analitik çerçeve, Lockheed Martin'in Siber Kill Chain’i ve MITRE’nin Saldırı Analizinin Elmas Modeli’dir (Diamond Model of Intrusion Analysis). Bu modeller, saldırı sürecini farklı açılardan ele alarak birbirini tamamlar.

\begin{itemize}
    \item \textbf{Siber Kill Chain (Siber Saldırı Zinciri):} Bu model, bir saldırının yedi aşamalı, doğrusal bir sürecini sunar. Her aşama, bir saldırganın başarılı bir siber saldırı gerçekleştirmek için tipik olarak izlediği adımları tanımlar.
    \begin{enumerate}
        \item \textbf{Keşif (Reconnaissance):} Saldırgan, hedefin sistemleri, ağ yapısı ve çalışanları hakkında bilgi toplar.
        \item \textbf{Silahlandırma (Weaponization):} Bir exploit ve bir arka kapı (payload) bir araya getirilerek tek bir saldırı paketi oluşturulur.
        \item \textbf{Teslimat (Delivery):} Saldırı paketi, hedef sisteme iletilir (örneğin, oltalama e-postası veya kötü amaçlı web sitesi aracılığıyla).
        \item \textbf{İstismar (Exploitation):} Bir sistemdeki zafiyet kullanılarak ilk erişim elde edilir.
        \item \textbf{Kurulum (Installation):} Saldırgan, sistemde kalıcılığı sağlamak için arka kapıyı kurar.
        \item \textbf{Komuta ve Kontrol (Command and Control):} Saldırganın ele geçirilmiş sistemle uzaktan iletişim kurarak onu kontrol etmesini sağlar.
        \item \textbf{Hedefler Üzerindeki Eylemler (Actions on Objectives):} Saldırgan, veri sızdırma, sistem bozulması veya fidye gibi nihai hedeflerine ulaşır.
    \end{enumerate}
    Kill Chain, saldırıların nasıl ilerlediğine dair adım adım bir yol haritası sunar ve özellikle olay müdahalesi ve taktiksel savunma kararları için çok değerlidir.
    \item \textbf{Diamond Model of Intrusion Analysis (Saldırı Analizinin Elmas Modeli):} Bu model, siber saldırıları dört temel bileşen arasındaki ilişkilere odaklanarak inceler. Doğrusal bir model değildir, daha çok saldırının bütüncül bir resmini sunar.
    \begin{enumerate}
        \item \textbf{Saldırgan (Adversary):} Saldırının arkasındaki tehdit aktörü.
        \item \textbf{Kabiliyet (Capability):} Saldırganın kullandığı araçlar, teknikler ve yöntemler.
        \item \textbf{Altyapı (Infrastructure):} Saldırıyı desteklemek için kullanılan ağ altyapısı (örneğin, C2 sunucuları, vekil sunucular).
        \item \textbf{Kurban (Victim):} Saldırıya uğrayan kişi, organizasyon veya sistem.
    \end{enumerate}
    Elmas Modeli, tehdit istihbaratı ve proaktif tehdit avcılığı için daha kullanışlıdır. Saldırıdan elde edilen herhangi bir bilgi parçası (örneğin, bir C2 sunucusu), diğer üç bileşenle ilişkilendirilerek saldırganın potansiyel diğer faaliyetlerini ve gelecekteki olası hedeflerini tahmin etmek için kullanılabilir.
\end{itemize}

Kill Chain, bir saldırının \textit{nasıl} ilerlediğini detaylandırırken, Diamond Model saldırının \textit{neden} ve \textit{kim tarafından} yapıldığını daha geniş bir bakış açısı sunar. Bu modeller birbirini tamamlar. Bir saldırı tespit edildiğinde, Kill Chain modeli olayın taktiksel olarak yönetilmesine yardımcı olurken, Diamond Model olayın bileşenlerini analiz ederek bu olayı bilinen diğer tehdit gruplarıyla ilişkilendirmeyi ve gelecekteki potansiyel saldırıları tahmin etmeyi mümkün kılar.

% Kill Chain ve Diamond Model Karşılaştırma Tablosu
\begin{longtable}{|p{3cm}|p{5.5cm}|p{5.5cm}|}
\hline
    	extbf{Özellik} & \textbf{Siber Kill Chain} & \textbf{Diamond Model} \\
\hline
    	extbf{Odak} & Saldırı aşamaları ve süreci & Aktör-kabiliyet-altyapı-kurban ilişkileri \\
\hline
    	extbf{Granülarite} & Spesifik ve doğrusal & Geniş, bağlamsal ve ilişkisel \\
\hline
    	extbf{Uygulama} & Olay müdahalesi, taktiksel savunma & Tehdit istihbaratı, proaktif avcılık \\
\hline
    	extbf{Avantajlar} & Adım adım müdahale yol haritası & Motivasyon ve kabiliyet analizi \\
\hline
\end{longtable}

\subsection{Threat Intelligence Lifecycle ve Collection Methods}

Tehdit istihbaratı yaşam döngüsü, ham veriyi değerli istihbarata dönüştüren ve sürekli bir geri bildirim döngüsü ile kendini yenileyen, yapılandırılmış bir süreçtir. Bu döngü, CTI programının temelini oluşturur ve reaktif bir modelden proaktif bir savunma yaklaşımına geçişin temelini atar.

\textbf{Yaşam Döngüsü Aşamaları:}

\begin{enumerate}
    \item \textbf{Gereksinimler (Requirements):} Döngünün ilk ve en kritik aşamasıdır. Bu aşamada, güvenlik ekipleri, iş birimleri ve üst düzey yöneticiler bir araya gelerek istihbarat ihtiyaçlarını net bir şekilde tanımlar. Bu, korunması gereken en kritik varlıkları (“Crown Jewels”), organizasyonun karşılaştığı riskleri ve bu riskleri azaltmak için hangi bilgilerin gerekli olduğunu belirlemeyi içerir. Gereksinimlerin net bir şekilde belirlenmesi, istihbarat toplama çabalarının boşa gitmesini engeller ve kaynakların doğru hedeflere yönlendirilmesini sağlar. Aksi halde, ekipler alakasız verileri takip ederek zaman ve kaynak kaybedebilir veya kritik tehditleri gözden kaçırabilir.
    \item \textbf{Toplama (Collection):} Bu aşama, tanımlanan gereksinimleri karşılamak için geniş bir yelpazedeki kaynaklardan ham veri toplamayı içerir. Toplanan veriler hem teknik (Indicator of Compromise - IOC) hem de bağlamsal bilgiler (TTP’ler, motivasyonlar) içermelidir.
    \begin{itemize}
        \item \textbf{OSINT (Open-Source Intelligence):} Genel kullanıma açık ve serbestçe erişilebilen kaynaklardan bilgi toplama. Bu kaynaklar arasında haberler, bloglar, sosyal medya platformları, akademik çalışmalar ve endüstri raporları bulunur.
        \item \textbf{Ticari Tehdit Beslemeleri:} Güvenlik firmaları tarafından sağlanan ve genellikle otomatik sistemlere entegre edilen ücretli veri akışlarıdır. Bu beslemeler, binlerce IOC ve TTP bilgisini yüksek hacimde sunabilir.
        \item \textbf{Bilgi Paylaşım Toplulukları (ISACs):} Aynı sektördeki veya coğrafi bölgedeki organizasyonların tehdit istihbaratı paylaşımı için bir araya geldiği güvenilir platformlardır.
        \item \textbf{Dahili Kaynaklar:} Organizasyonun kendi güvenlik araçlarından (SIEM, EDR, IDS/IPS), ağ günlüklerinden, uç nokta telemetri verilerinden ve kimlik doğrulama kayıtlarından elde edilen verilerdir.
        \item \textbf{Derin ve Karanlık Web İzleme:} Gizli veya şifreli forumlar, siber suç pazarları ve sızdırılmış veri depolarından (örneğin, sızan kimlik bilgileri, saldırı planları) bilgi toplama.
    \end{itemize}
    \item \textbf{İşleme (Processing):} Toplanan ham verinin analiz edilebilir, yapılandırılmış ve temiz bir formata dönüştürüldüğü aşamadır. Bu, veri normalizasyonu (farklı formatlardaki verileri standartlaştırma), tekilleştirme (yinelenen kayıtları kaldırma), şifre çözme ve etiketleme işlemlerini içerir. Büyük veri kümeleri için otomasyon, bu aşamada hayati öneme sahiptir.
    \item \textbf{Analiz ve Yorumlama (Analysis and Interpretation):} İşlenmiş verinin anlamlı, eyleme dönüştürülebilir istihbarata dönüştüğü aşamadır. Analistler, kalıpları, eğilimleri, saldırı kampanyalarını ve potansiyel tehditleri belirlemek için verileri derinlemesine incelerler. Bu aşamada, insan uzmanlığı ve otomatik korelasyon mekanizmaları birleşir.
    \item \textbf{Yayma (Dissemination):} Analiz edilen istihbaratın ilgili paydaşlara zamanında ve uygun bir formatta sunulmasıdır. İstihbaratın formatı ve içeriği, hedef kitleye göre uyarlanmalıdır. Örneğin, CISO'lar için üst düzey yönetici brifingleri hazırlanırken, SOC analistleri için teknik IOC beslemleri sunulur.
    \item \textbf{Geri Bildirim (Feedback):} Sürecin sürekli iyileştirilmesini sağlayan bu aşamada, istihbarat tüketicileri (örneğin, SOC analistleri), sunulan bilgilerin yararlılığı, alaka düzeyi ve doğruluğu hakkında geri bildirimde bulunur. Bu geri bildirim, bir sonraki döngüde daha iyi gereksinimlerin belirlenmesini ve istihbarat toplama stratejilerinin hassaslaştırılmasını sağlar.
\end{enumerate}

Bu yaşam döngüsü, bir organizasyonun reaktif tehdit algılama modelinden proaktif bir savunma duruşuna geçiş yapmasına olanak tanır. Bu sürekli ve döngüsel süreç, savunma mekanizmalarının saldırganların evrilen taktik, teknik ve prosedürlerine (TTP'ler) göre dinamik olarak ayarlanmasını sağlar. Bu dinamik adaptasyon, bir saldırganın bir sistemde kalma süresini (dwell time) doğrudan azaltır. Saldırganın sistemde kalma süresinin azalması, fidye yazılımı veya veri sızdırma gibi nihai hedeflere ulaşma şansını düşürürken, ihlalden kaynaklanan maliyetleri de önemli ölçüde azaltır. Bu sürekli öğrenme ve adaptasyon mekanizması, tehdit istihbaratı programının değerini ve etkinliğini katlanarak artırır.

\textbf{CTI Yaşam Döngüsü Aşamaları ve Uygulamaları}

\begin{longtable}{|p{2.5cm}|p{3cm}|p{3cm}|p{3.5cm}|}
\hline
\textbf{Aşama} & \textbf{Kısa Tanım} & \textbf{Hedef} & \textbf{Pratik Uygulama} \\
\hline
\textbf{Gereksinimler} & İhtiyaç duyulan istihbaratın belirlenmesi & Kaynakları kritik risklere odaklamak & Tehdit aktörü TTP analizi \\
\hline
\textbf{Toplama} & Ham verinin kaynaklardan alınması & Gereksinim odaklı veri toplama & SIEM günlükleri, OSINT, karanlık web \\
\hline
\textbf{İşleme} & Ham veriyi analiz formatına getirme & Veri yapılandırma ve zenginleştirme & CSV normalizasyonu, VirusTotal entegrasyonu \\
\hline
\textbf{Analiz} & Veriden içgörüler çıkarma & Tehdit bağlamsallaştırma & Oltalama kampanya teknik analizi \\
\hline
\textbf{Yayma} & İstihbaratı paydaşlara iletme & Doğru kitleye doğru formatta sunum & Yönetici brifingi, SOC beslemesi \\
\hline
\textbf{Geri Bildirim} & Etkinlik geri bildirimi toplama & Sürekli iyileştirme & Yanlış pozitif raporları, değerlendirmeler \\
\hline
\end{longtable}

\subsection{Strategic, Tactical, Technical ve Operational Intelligence}

Siber tehdit istihbaratı, hedef kitlenin ihtiyaçlarına göre dört ana kategoriye ayrılır. Bu istihbarat türleri, bir organizasyonun savunma yeteneklerini tüm seviyelerde güçlendirmek için birlikte çalışır.

\begin{itemize}
    \item \textbf{Stratejik İstihbarat:} Genel tehdit ortamına ilişkin üst düzey bir bakış açısı sunar. Bu istihbarat, teknoloji dışı terimlerle hazırlanır ve öncelikli olarak üst düzey yöneticilere, CISO'lara ve risk yöneticilerine yöneliktir.
    \begin{itemize}
        \item \textbf{Amacı:} Güvenlik yatırımları, bütçe tahsisi ve kurumsal politikalar gibi uzun vadeli stratejik kararlara rehberlik etmektir.
        \item \textbf{Örnekler:} Bir APT grubunun bir sektörü veya belirli bir coğrafyayı hedeflemesi, fidye yazılımı eğilimleri veya jeopolitik olayların siber saldırı risklerine etkisi hakkında raporlar.
    \end{itemize}
    \item \textbf{Operasyonel İstihbarat:} Belirli bir tehdit aktörünün TTP'lerini, motivasyonlarını ve altyapısını anlamayı sağlar. Bu istihbarat, olay müdahale (IR) ekipleri ve tehdit avcıları için hayati öneme sahiptir.
    \begin{itemize}
        \item \textbf{Amacı:} Yaklaşan veya devam eden bir saldırının "kim, ne zaman, nerede, nasıl, neden" sorularına bağlamsal cevaplar sunmak. Bu, proaktif tehdit avcılığı ve olay müdahale planlaması için bir temel oluşturur.
        \item \textbf{Örnekler:} Bir saldırıda kullanılan belirli bir oltalama e-postası kampanyasının detayları veya bir Komuta ve Kontrol (C2) sunucusunun iletişim yöntemleri.
    \end{itemize}
    \item \textbf{Taktik İstihbarat:} Ağ ve uç noktalarda tehditleri tespit etmeye yardımcı olan, genellikle kısa ömürlü ve teknik göstergelerden oluşan bir istihbarat türüdür.
    \begin{itemize}
        \item \textbf{Amacı:} Otomatik tehdit tespiti için SIEM ve güvenlik duvarı gibi güvenlik kontrollerine entegre edilmek.
        \item \textbf{Örnekler:} Kötü amaçlı IP adresleri, dosya karmaları (hashes), kötü amaçlı alan adları.
    \end{itemize}
    \item \textbf{Teknik İstihbarat:} Taktik istihbaratın daha derin teknik detaylarını içerir ve genellikle kötü amaçlı yazılım analizi, tersine mühendislik ve IoC'lerin oluşturulmasını kapsar.
    \begin{itemize}
        \item \textbf{Amacı:} Saldırıların teknik işleyişini anlamak ve bu bilgilere dayanarak yeni imza tabanlı tespit kuralları oluşturmaktır.
    \end{itemize}
\end{itemize}

Bu dört istihbarat türü, birbirini besleyen ve destekleyen bir hiyerarşi içinde çalışır. Üst yönetim, genel güvenlik stratejisini belirlemek için stratejik istihbarata ihtiyaç duyar. Bu strateji, operasyonel ve taktiksel gereksinimlere dönüşür. Örneğin, stratejik istihbarat, jeopolitik gerilimlerin bir APT grubunun faaliyetlerini artıracağını gösterdiğinde, operasyonel ekip bu grubun TTP'lerini incelemeye odaklanır ve bu operasyonel bilgiler taktiksel IoC'lere dönüştürülerek güvenlik kontrollerine entegre edilir. Bu entegrasyon, istihbaratın tüm organizasyonel katmanlarda değer yaratmasını sağlar.

\textbf{İstihbarat Türleri Karşılaştırma Tablosu}

\begin{longtable}{|p{2cm}|p{3.5cm}|p{3.5cm}|p{3cm}|}
\hline
\textbf{Özellik} & \textbf{Stratejik} & \textbf{Operasyonel} & \textbf{Taktik} \\
\hline
\textbf{Hedef Kitle} & C-seviyesi yöneticiler & IR ekipleri, tehdit avcıları & SOC analistleri \\
\hline
\textbf{Odak Noktası} & Genel eğilimler, risk yönetimi & Saldırı kampanyaları, TTP'ler & IOC'ler, ağ etkinlikleri \\
\hline
\textbf{Zaman} & Uzun vadeli (aylar/yıllar) & Orta vadeli (hafta/ay) & Kısa vadeli (gerçek zamanlı) \\
\hline
\textbf{Amaç} & Politika ve yatırım belirleme & Müdahale ve proaktif avcılık & Otomatik tespit ve engelleme \\
\hline
\end{longtable}

\subsection{Threat Actor Profiling ve Attribution Challenges}

Tehdit aktörü profilleme, saldırıların arkasındaki "kim" sorusuna cevap aramayı amaçlar. Bir tehdit aktörü profili, sadece bir isimden ibaret değildir; bir saldırganın kimliğini, hedeflerini, TTP'lerini, motivasyonlarını, coğrafi konumunu ve kullandığı altyapıyı kapsamlı bir şekilde analiz eder. Bu profiller, güvenlik ekiplerine saldırganın olası davranışları hakkında bir resim sunar ve savunma stratejilerini saldırganın niyetleriyle uyumlu hale getirmeye yardımcı olur.

Ancak, bir siber saldırıyı kesin olarak belirli bir tehdit aktörüne atfetmek (attribution), son derece karmaşık ve zorlu bir süreçtir. Bu zorlukların birkaç temel nedeni vardır:

\begin{itemize}
    \item \textbf{Gizleme ve Obfüskasyon:} Saldırganlar, kimliklerini gizlemek için çok sayıda katman kullanır. Botnetler, vekil sunucular ve kiralanmış altyapı, saldırının gerçek kaynağını maskeler. Özellikle dağıtık servis dışı bırakma (DDoS) saldırılarında, trafik binlerce farklı cihazdan gelebilir, bu da kesin bir atıf yapmayı neredeyse imkansız hale getirir.
    \item \textbf{Yanlış Bayrak Operasyonları:} Bazı siber suç grupları, dikkat çekmek veya yanlış bilgi yaymak amacıyla gerçekleştirmediği saldırıların sorumluluğunu üstlenebilir. Bu tür eylemler, atıf sürecini manipüle etmeyi amaçlar.
    \item \textbf{Araç ve TTP Paylaşımı:} Tehdit grupları, sıklıkla araç setlerini ve TTP'lerini birbirleriyle paylaşır veya değiştirir. Bir saldırıda kullanılan belirli bir teknik, daha önce bilinen bir grubun imzası olsa bile, başka bir grup tarafından da kullanılmış olabilir. Bu durum, yalnızca teknik IOC'lere dayalı atıf yapmanın güvenilirliğini azaltır.
\end{itemize}

Bu nedenle, atıf birincil hedef olmamalıdır. Bir saldırı meydana geldiğinde, ilk ve en acil öncelik, hasarı durdurmak, sistemleri güvence altına almak ve devam eden tehditleri ortadan kaldırmaktır. Atıf, bu acil riskler ortadan kaldırıldıktan sonra, olay sonrası analiz aşamasında daha sonraki savunmaları güçlendirmek için bir araç olarak ele alınmalıdır. Atıf, bir "evet/hayır" cevabı yerine, farklı güven seviyelerine sahip (düşük, orta, yüksek) analitik bir değerlendirme süreci olarak görülmelidir. Güvenilir bir atıf için, sadece IOC'ler yerine, davranışsal kanıtlar ve TTP'lere odaklanmak daha geçerli bir yaklaşımdır.

\subsection{Diamond Model ve Kill Chain Analysis}

Siber saldırıları anlamak ve analiz etmek için kullanılan iki önemli analitik çerçeve, Lockheed Martin'in Siber Kill Chain’i ve MITRE’nin Saldırı Analizinin Elmas Modeli’dir (Diamond Model of Intrusion Analysis). Bu modeller, saldırı sürecini farklı açılardan ele alarak birbirini tamamlar.

\begin{itemize}
    \item \textbf{Siber Kill Chain (Siber Saldırı Zinciri):} Bu model, bir saldırının yedi aşamalı, doğrusal bir sürecini sunar. Her aşama, bir saldırganın başarılı bir siber saldırı gerçekleştirmek için tipik olarak izlediği adımları tanımlar.
    \begin{enumerate}
        \item \textbf{Keşif (Reconnaissance):} Saldırgan, hedefin sistemleri, ağ yapısı ve çalışanları hakkında bilgi toplar.
        \item \textbf{Silahlandırma (Weaponization):} Bir exploit ve bir arka kapı (payload) bir araya getirilerek tek bir saldırı paketi oluşturulur.
        \item \textbf{Teslimat (Delivery):} Saldırı paketi, hedef sisteme iletilir (örneğin, oltalama e-postası veya kötü amaçlı web sitesi aracılığıyla).
        \item \textbf{İstismar (Exploitation):} Bir sistemdeki zafiyet kullanılarak ilk erişim elde edilir.
        \item \textbf{Kurulum (Installation):} Saldırgan, sistemde kalıcılığı sağlamak için arka kapıyı kurar.
        \item \textbf{Komuta ve Kontrol (Command and Control):} Saldırganın ele geçirilmiş sistemle uzaktan iletişim kurarak onu kontrol etmesini sağlar.
        \item \textbf{Hedefler Üzerindeki Eylemler (Actions on Objectives):} Saldırgan, veri sızdırma, sistem bozulması veya fidye gibi nihai hedeflerine ulaşır.
    \end{enumerate}
    Kill Chain, saldırıların nasıl ilerlediğine dair adım adım bir yol haritası sunar ve özellikle olay müdahalesi ve taktiksel savunma kararları için çok değerlidir.
    \item \textbf{Diamond Model of Intrusion Analysis (Saldırı Analizinin Elmas Modeli):} Bu model, siber saldırıları dört temel bileşen arasındaki ilişkilere odaklanarak inceler. Doğrusal bir model değildir, daha çok saldırının bütüncül bir resmini sunar.
    \begin{enumerate}
        \item \textbf{Saldırgan (Adversary):} Saldırının arkasındaki tehdit aktörü.
        \item \textbf{Kabiliyet (Capability):} Saldırganın kullandığı araçlar, teknikler ve yöntemler.
        \item \textbf{Altyapı (Infrastructure):} Saldırıyı desteklemek için kullanılan ağ altyapısı (örneğin, C2 sunucuları, vekil sunucular).
        \item \textbf{Kurban (Victim):} Saldırıya uğrayan kişi, organizasyon veya sistem.
    \end{enumerate}
    Elmas Modeli, tehdit istihbaratı ve proaktif tehdit avcılığı için daha kullanışlıdır. Saldırıdan elde edilen herhangi bir bilgi parçası (örneğin, bir C2 sunucusu), diğer üç bileşenle ilişkilendirilerek saldırganın potansiyel diğer faaliyetlerini ve gelecekteki olası hedeflerini tahmin etmek için kullanılabilir.
\end{itemize}

Kill Chain, bir saldırının \textit{nasıl} ilerlediğini detaylandırırken, Diamond Model saldırının \textit{neden} ve \textit{kim tarafından} yapıldığını daha geniş bir bakış açısı sunar. Bu modeller birbirini tamamlar. Bir saldırı tespit edildiğinde, Kill Chain modeli olayın taktiksel olarak yönetilmesine yardımcı olurken, Diamond Model olayın bileşenlerini analiz ederek bu olayı bilinen diğer tehdit gruplarıyla ilişkilendirmeyi ve gelecekteki potansiyel saldırıları tahmin etmeyi mümkün kılar.

% Kill Chain ve Diamond Model Karşılaştırma Tablosu
\begin{longtable}{|p{3cm}|p{5.5cm}|p{5.5cm}|}
\hline
\textbf{Özellik} & \textbf{Siber Kill Chain} & \textbf{Diamond Model} \\
\hline
\textbf{Odak} & Saldırı aşamaları ve süreci & Aktör-kabiliyet-altyapı-kurban ilişkileri \\
\hline
\textbf{Granülarite} & Spesifik ve doğrusal & Geniş, bağlamsal ve ilişkisel \\
\hline
\textbf{Uygulama} & Olay müdahalesi, taktiksel savunma & Tehdit istihbaratı, proaktif avcılık \\
\hline
\textbf{Avantajlar} & Adım adım müdahale yol haritası & Motivasyon ve kabiliyet analizi \\
\hline
\end{longtable}

\subsection{Threat Intelligence Lifecycle ve Collection Methods}

Tehdit istihbaratı yaşam döngüsü, ham veriyi değerli istihbarata dönüştüren ve sürekli bir geri bildirim döngüsü ile kendini yenileyen, yapılandırılmış bir süreçtir. Bu döngü, CTI programının temelini oluşturur ve reaktif bir modelden proaktif bir savunma yaklaşımına geçişin temelini atar.

\textbf{Yaşam Döngüsü Aşamaları:}

\begin{enumerate}
    \item \textbf{Gereksinimler (Requirements):} Döngünün ilk ve en kritik aşamasıdır. Bu aşamada, güvenlik ekipleri, iş birimleri ve üst düzey yöneticiler bir araya gelerek istihbarat ihtiyaçlarını net bir şekilde tanımlar. Bu, korunması gereken en kritik varlıkları (“Crown Jewels”), organizasyonun karşılaştığı riskleri ve bu riskleri azaltmak için hangi bilgilerin gerekli olduğunu belirlemeyi içerir. Gereksinimlerin net bir şekilde belirlenmesi, istihbarat toplama çabalarının boşa gitmesini engeller ve kaynakların doğru hedeflere yönlendirilmesini sağlar. Aksi halde, ekipler alakasız verileri takip ederek zaman ve kaynak kaybedebilir veya kritik tehditleri gözden kaçırabilir.
    \item \textbf{Toplama (Collection):} Bu aşama, tanımlanan gereksinimleri karşılamak için geniş bir yelpazedeki kaynaklardan ham veri toplamayı içerir. Toplanan veriler hem teknik (Indicator of Compromise - IOC) hem de bağlamsal bilgiler (TTP’ler, motivasyonlar) içermelidir.
    \begin{itemize}
        \item \textbf{OSINT (Open-Source Intelligence):} Genel kullanıma açık ve serbestçe erişilebilen kaynaklardan bilgi toplama. Bu kaynaklar arasında haberler, bloglar, sosyal medya platformları, akademik çalışmalar ve endüstri raporları bulunur.
        \item \textbf{Ticari Tehdit Beslemeleri:} Güvenlik firmaları tarafından sağlanan ve genellikle otomatik sistemlere entegre edilen ücretli veri akışlarıdır. Bu beslemeler, binlerce IOC ve TTP bilgisini yüksek hacimde sunabilir.
        \item \textbf{Bilgi Paylaşım Toplulukları (ISACs):} Aynı sektördeki veya coğrafi bölgedeki organizasyonların tehdit istihbaratı paylaşımı için bir araya geldiği güvenilir platformlardır.
        \item \textbf{Dahili Kaynaklar:} Organizasyonun kendi güvenlik araçlarından (SIEM, EDR, IDS/IPS), ağ günlüklerinden, uç nokta telemetri verilerinden ve kimlik doğrulama kayıtlarından elde edilen verilerdir.
        \item \textbf{Derin ve Karanlık Web İzleme:} Gizli veya şifreli forumlar, siber suç pazarları ve sızdırılmış veri depolarından (örneğin, sızan kimlik bilgileri, saldırı planları) bilgi toplama.
    \end{itemize}
    \item \textbf{İşleme (Processing):} Toplanan ham verinin analiz edilebilir, yapılandırılmış ve temiz bir formata dönüştürüldüğü aşamadır. Bu, veri normalizasyonu (farklı formatlardaki verileri standartlaştırma), tekilleştirme (yinelenen kayıtları kaldırma), şifre çözme ve etiketleme işlemlerini içerir. Büyük veri kümeleri için otomasyon, bu aşamada hayati öneme sahiptir.
    \item \textbf{Analiz ve Yorumlama (Analysis and Interpretation):} İşlenmiş verinin anlamlı, eyleme dönüştürülebilir istihbarata dönüştüğü aşamadır. Analistler, kalıpları, eğilimleri, saldırı kampanyalarını ve potansiyel tehditleri belirlemek için verileri derinlemesine incelerler. Bu aşamada, insan uzmanlığı ve otomatik korelasyon mekanizmaları birleşir.
    \item \textbf{Yayma (Dissemination):} Analiz edilen istihbaratın ilgili paydaşlara zamanında ve uygun bir formatta sunulmasıdır. İstihbaratın formatı ve içeriği, hedef kitleye göre uyarlanmalıdır. Örneğin, CISO'lar için üst düzey yönetici brifingleri hazırlanırken, SOC analistleri için teknik IOC beslemleri sunulur.
    \item \textbf{Geri Bildirim (Feedback):} Sürecin sürekli iyileştirilmesini sağlayan bu aşamada, istihbarat tüketicileri (örneğin, SOC analistleri), sunulan bilgilerin yararlılığı, alaka düzeyi ve doğruluğu hakkında geri bildirimde bulunur. Bu geri bildirim, bir sonraki döngüde daha iyi gereksinimlerin belirlenmesini ve istihbarat toplama stratejilerinin hassaslaştırılmasını sağlar.
\end{enumerate}

Bu yaşam döngüsü, bir organizasyonun reaktif tehdit algılama modelinden proaktif bir savunma duruşuna geçiş yapmasına olanak tanır. Bu sürekli ve döngüsel süreç, savunma mekanizmalarının saldırganların evrilen taktik, teknik ve prosedürlerine (TTP'ler) göre dinamik olarak ayarlanmasını sağlar. Bu dinamik adaptasyon, bir saldırganın bir sistemde kalma süresini (dwell time) doğrudan azaltır. Saldırganın sistemde kalma süresinin azalması, fidye yazılımı veya veri sızdırma gibi nihai hedeflere ulaşma şansını düşürürken, ihlalden kaynaklanan maliyetleri de önemli ölçüde azaltır. Bu sürekli öğrenme ve adaptasyon mekanizması, tehdit istihbaratı programının değerini ve etkinliğini katlanarak artırır.

\textbf{CTI Yaşam Döngüsü Aşamaları ve Uygulamaları}

\begin{longtable}{|p{2.5cm}|p{3cm}|p{3cm}|p{3.5cm}|}
\hline
\textbf{Aşama} & \textbf{Kısa Tanım} & \textbf{Hedef} & \textbf{Pratik Uygulama} \\
\hline
\textbf{Gereksinimler} & İhtiyaç duyulan istihbaratın belirlenmesi & Kaynakları kritik risklere odaklamak & Tehdit aktörü TTP analizi \\
\hline
\textbf{Toplama} & Ham verinin kaynaklardan alınması & Gereksinim odaklı veri toplama & SIEM günlükleri, OSINT, karanlık web \\
\hline
\textbf{İşleme} & Ham veriyi analiz formatına getirme & Veri yapılandırma ve zenginleştirme & CSV normalizasyonu, VirusTotal entegrasyonu \\
\hline
\textbf{Analiz} & Veriden içgörüler çıkarma & Tehdit bağlamsallaştırma & Oltalama kampanya teknik analizi \\
\hline
\textbf{Yayma} & İstihbaratı paydaşlara iletme & Doğru kitleye doğru formatta sunum & Yönetici brifingi, SOC beslemesi \\
\hline
\textbf{Geri Bildirim} & Etkinlik geri bildirimi toplama & Sürekli iyileştirme & Yanlış pozitif raporları, değerlendirmeler \\
\hline
\end{longtable}

\subsection{Strategic, Tactical, Technical ve Operational Intelligence}

Siber tehdit istihbaratı, hedef kitlenin ihtiyaçlarına göre dört ana kategoriye ayrılır. Bu istihbarat türleri, bir organizasyonun savunma yeteneklerini tüm seviyelerde güçlendirmek için birlikte çalışır.

\begin{itemize}
    \item \textbf{Stratejik İstihbarat:} Genel tehdit ortamına ilişkin üst düzey bir bakış açısı sunar. Bu istihbarat, teknoloji dışı terimlerle hazırlanır ve öncelikli olarak üst düzey yöneticilere, CISO'lara ve risk yöneticilerine yöneliktir.
    \begin{itemize}
        \item \textbf{Amacı:} Güvenlik yatırımları, bütçe tahsisi ve kurumsal politikalar gibi uzun vadeli stratejik kararlara rehberlik etmektir.
        \item \textbf{Örnekler:} Bir APT grubunun bir sektörü veya belirli bir coğrafyayı hedeflemesi, fidye yazılımı eğilimleri veya jeopolitik olayların siber saldırı risklerine etkisi hakkında raporlar.
    \end{itemize}
    \item \textbf{Operasyonel İstihbarat:} Belirli bir tehdit aktörünün TTP'lerini, motivasyonlarını ve altyapısını anlamayı sağlar. Bu istihbarat, olay müdahale (IR) ekipleri ve tehdit avcıları için hayati öneme sahiptir.
    \begin{itemize}
        \item \textbf{Amacı:} Yaklaşan veya devam eden bir saldırının "kim, ne zaman, nerede, nasıl, neden" sorularına bağlamsal cevaplar sunmak. Bu, proaktif tehdit avcılığı ve olay müdahale planlaması için bir temel oluşturur.
        \item \textbf{Örnekler:} Bir saldırıda kullanılan belirli bir oltalama e-postası kampanyasının detayları veya bir Komuta ve Kontrol (C2) sunucusunun iletişim yöntemleri.
    \end{itemize}
    \item \textbf{Taktik İstihbarat:} Ağ ve uç noktalarda tehditleri tespit etmeye yardımcı olan, genellikle kısa ömürlü ve teknik göstergelerden oluşan bir istihbarat türüdür.
    \begin{itemize}
        \item \textbf{Amacı:} Otomatik tehdit tespiti için SIEM ve güvenlik duvarı gibi güvenlik kontrollerine entegre edilmek.
        \item \textbf{Örnekler:} Kötü amaçlı IP adresleri, dosya karmaları (hashes), kötü amaçlı alan adları.
    \end{itemize}
    \item \textbf{Teknik İstihbarat:} Taktik istihbaratın daha derin teknik detaylarını içerir ve genellikle kötü amaçlı yazılım analizi, tersine mühendislik ve IoC'lerin oluşturulmasını kapsar.
    \begin{itemize}
        \item \textbf{Amacı:} Saldırıların teknik işleyişini anlamak ve bu bilgilere dayanarak yeni imza tabanlı tespit kuralları oluşturmaktır.
    \end{itemize}
\end{itemize}

Bu dört istihbarat türü, birbirini besleyen ve destekleyen bir hiyerarşi içinde çalışır. Üst yönetim, genel güvenlik stratejisini belirlemek için stratejik istihbarata ihtiyaç duyar. Bu strateji, operasyonel ve taktiksel gereksinimlere dönüşür. Örneğin, stratejik istihbarat, jeopolitik gerilimlerin bir APT grubunun faaliyetlerini artıracağını gösterdiğinde, operasyonel ekip bu grubun TTP'lerini incelemeye odaklanır ve bu operasyonel bilgiler taktiksel IoC'lere dönüştürülerek güvenlik kontrollerine entegre edilir. Bu entegrasyon, istihbaratın tüm organizasyonel katmanlarda değer yaratmasını sağlar.

\textbf{İstihbarat Türleri Karşılaştırma Tablosu}

\begin{longtable}{|p{2cm}|p{3.5cm}|p{3.5cm}|p{3cm}|}
\hline
\textbf{Özellik} & \textbf{Stratejik} & \textbf{Operasyonel} & \textbf{Taktik} \\
\hline
\textbf{Hedef Kitle} & C-seviyesi yöneticiler & IR ekipleri, tehdit avcıları & SOC analistleri \\
\hline
\textbf{Odak Noktası} & Genel eğilimler, risk yönetimi & Saldırı kampanyaları, TTP'ler & IOC'ler, ağ etkinlikleri \\
\hline
\textbf{Zaman} & Uzun vadeli (aylar/yıllar) & Orta vadeli (hafta/ay) & Kısa vadeli (gerçek zamanlı) \\
\hline
\textbf{Amaç} & Politika ve yatırım belirleme & Müdahale ve proaktif avcılık & Otomatik tespit ve engelleme \\
\hline
\end{longtable}

\subsection{Threat Actor Profiling ve Attribution Challenges}

Tehdit aktörü profilleme, saldırıların arkasındaki "kim" sorusuna cevap aramayı amaçlar. Bir tehdit aktörü profili, sadece bir isimden ibaret değildir; bir saldırganın kimliğini, hedeflerini, TTP'lerini, motivasyonlarını, coğrafi konumunu ve kullandığı altyapıyı kapsamlı bir şekilde analiz eder. Bu profiller, güvenlik ekiplerine saldırganın olası davranışları hakkında bir resim sunar ve savunma stratejilerini saldırganın niyetleriyle uyumlu hale getirmeye yardımcı olur.

Ancak, bir siber saldırıyı kesin olarak belirli bir tehdit aktörüne atfetmek (attribution), son derece karmaşık ve zorlu bir süreçtir. Bu zorlukların birkaç temel nedeni vardır:

\begin{itemize}
    \item \textbf{Gizleme ve Obfüskasyon:} Saldırganlar, kimliklerini gizlemek için çok sayıda katman kullanır. Botnetler, vekil sunucular ve kiralanmış altyapı, saldırının gerçek kaynağını maskeler. Özellikle dağıtık servis dışı bırakma (DDoS) saldırılarında, trafik binlerce farklı cihazdan gelebilir, bu da kesin bir atıf yapmayı neredeyse imkansız hale getirir.
    \item \textbf{Yanlış Bayrak Operasyonları:} Bazı siber suç grupları, dikkat çekmek veya yanlış bilgi yaymak amacıyla gerçekleştirmediği saldırıların sorumluluğunu üstlenebilir. Bu tür eylemler, atıf sürecini manipüle etmeyi amaçlar.
    \item \textbf{Araç ve TTP Paylaşımı:} Tehdit grupları, sıklıkla araç setlerini ve TTP'lerini birbirleriyle paylaşır veya değiştirir. Bir saldırıda kullanılan belirli bir teknik, daha önce bilinen bir grubun imzası olsa bile, başka bir grup tarafından da kullanılmış olabilir. Bu durum, yalnızca teknik IOC'lere dayalı atıf yapmanın güvenilirliğini azaltır.
\end{itemize}

Bu nedenle, atıf birincil hedef olmamalıdır. Bir saldırı meydana geldiğinde, ilk ve en acil öncelik, hasarı durdurmak, sistemleri güvence altına almak ve devam eden tehditleri ortadan kaldırmaktır. Atıf, bu acil riskler ortadan kaldırıldıktan sonra, olay sonrası analiz aşamasında daha sonraki savunmaları güçlendirmek için bir araç olarak ele alınmalıdır. Atıf, bir "evet/hayır" cevabı yerine, farklı güven seviyelerine sahip (düşük, orta, yüksek) analitik bir değerlendirme süreci olarak görülmelidir. Güvenilir bir atıf için, sadece IOC'ler yerine, davranışsal kanıtlar ve TTP'lere odaklanmak daha geçerli bir yaklaşımdır.

\subsection{Diamond Model ve Kill Chain Analysis}

Siber saldırıları anlamak ve analiz etmek için kullanılan iki önemli analitik çerçeve, Lockheed Martin'in Siber Kill Chain’i ve MITRE’nin Saldırı Analizinin Elmas Modeli’dir (Diamond Model of Intrusion Analysis). Bu modeller, saldırı sürecini farklı açılardan ele alarak birbirini tamamlar.

\begin{itemize}
    \item \textbf{Siber Kill Chain (Siber Saldırı Zinciri):} Bu model, bir saldırının yedi aşamalı, doğrusal bir sürecini sunar. Her aşama, bir saldırganın başarılı bir siber saldırı gerçekleştirmek için tipik olarak izlediği adımları tanımlar.
    \begin{enumerate}
        \item \textbf{Keşif (Reconnaissance):} Saldırgan, hedefin sistemleri, ağ yapısı ve çalışanları hakkında bilgi toplar.
        \item \textbf{Silahlandırma (Weaponization):} Bir exploit ve bir arka kapı (payload) bir araya getirilerek tek bir saldırı paketi oluşturulur.
        \item \textbf{Teslimat (Delivery):} Saldırı paketi, hedef sisteme iletilir (örneğin, oltalama e-postası veya kötü amaçlı web sitesi aracılığıyla).
        \item \textbf{İstismar (Exploitation):} Bir sistemdeki zafiyet kullanılarak ilk erişim elde edilir.
        \item \textbf{Kurulum (Installation):} Saldırgan, sistemde kalıcılığı sağlamak için arka kapıyı kurar.
        \item \textbf{Komuta ve Kontrol (Command and Control):} Saldırganın ele geçirilmiş sistemle uzaktan iletişim kurarak onu kontrol etmesini sağlar.
        \item \textbf{Hedefler Üzerindeki Eylemler (Actions on Objectives):} Saldırgan, veri sızdırma, sistem bozulması veya fidye gibi nihai hedeflerine ulaşır.
    \end{enumerate}
    Kill Chain, saldırıların nasıl ilerlediğine dair adım adım bir yol haritası sunar ve özellikle olay müdahalesi ve taktiksel savunma kararları için çok değerlidir.
    \item \textbf{Diamond Model of Intrusion Analysis (Saldırı Analizinin Elmas Modeli):} Bu model, siber saldırıları dört temel bileşen arasındaki ilişkilere odaklanarak inceler. Doğrusal bir model değildir, daha çok saldırının bütüncül bir resmini sunar.
    \begin{enumerate}
        \item \textbf{Saldırgan (Adversary):} Saldırının arkasındaki tehdit aktörü.
        \item \textbf{Kabiliyet (Capability):} Saldırganın kullandığı araçlar, teknikler ve yöntemler.
        \item \textbf{Altyapı (Infrastructure):} Saldırıyı desteklemek için kullanılan ağ altyapısı (örneğin, C2 sunucuları, vekil sunucular).
        \item \textbf{Kurban (Victim):} Saldırıya uğrayan kişi, organizasyon veya sistem.
    \end{enumerate}
    Elmas Modeli, tehdit istihbaratı ve proaktif tehdit avcılığı için daha kullanışlıdır. Saldırıdan elde edilen herhangi bir bilgi parçası (örneğin, bir C2 sunucusu), diğer üç bileşenle ilişkilendirilerek saldırganın potansiyel diğer faaliyetlerini ve gelecekteki olası hedeflerini tahmin etmek için kullanılabilir.
\end{itemize}

Kill Chain, bir saldırının \textit{nasıl} ilerlediğini detaylandırırken, Diamond Model saldırının \textit{neden} ve \textit{kim tarafından} yapıldığını daha geniş bir bakış açısı sunar. Bu modeller birbirini tamamlar. Bir saldırı tespit edildiğinde, Kill Chain modeli olayın taktiksel olarak yönetilmesine yardımcı olurken, Diamond Model olayın bileşenlerini analiz ederek bu olayı bilinen diğer tehdit gruplarıyla ilişkilendirmeyi ve gelecekteki potansiyel saldırıları tahmin etmeyi mümkün kılar.

% Kill Chain ve Diamond Model Karşılaştırma Tablosu
\begin{longtable}{|p{3cm}|p{5.5cm}|p{5.5cm}|}
\hline
\textbf{Özellik} & \textbf{Siber Kill Chain} & \textbf{Diamond Model} \\
\hline
\textbf{Odak} & Saldırı aşamaları ve süreci & Aktör-kabiliyet-altyapı-kurban ilişkileri \\
\hline
\textbf{Granülarite} & Spesifik ve doğrusal & Geniş, bağlamsal ve ilişkisel \\
\hline
\textbf{Uygulama} & Olay müdahalesi, taktiksel savunma & Tehdit istihbaratı, proaktif avcılık \\
\hline
\textbf{Avantajlar} & Adım adım müdahale yol haritası & Motivasyon ve kabiliyet analizi \\
\hline
\end{longtable}

\subsection{Threat Intelligence Lifecycle ve Collection Methods}

Tehdit istihbaratı yaşam döngüsü, ham veriyi değerli istihbarata dönüştüren ve sürekli bir geri bildirim döngüsü ile kendini yenileyen, yapılandırılmış bir süreçtir. Bu döngü, CTI programının temelini oluşturur ve reaktif bir modelden proaktif bir savunma yaklaşımına geçişin temelini atar.

\textbf{Yaşam Döngüsü Aşamaları:}

\begin{enumerate}
    \item \textbf{Gereksinimler (Requirements):} Döngünün ilk ve en kritik aşamasıdır. Bu aşamada, güvenlik ekipleri, iş birimleri ve üst düzey yöneticiler bir araya gelerek istihbarat ihtiyaçlarını net bir şekilde tanımlar. Bu, korunması gereken en kritik varlıkları (“Crown Jewels”), organizasyonun karşılaştığı riskleri ve bu riskleri azaltmak için hangi bilgilerin gerekli olduğunu belirlemeyi içerir. Gereksinimlerin net bir şekilde belirlenmesi, istihbarat toplama çabalarının boşa gitmesini engeller ve kaynakların doğru hedeflere yönlendirilmesini sağlar. Aksi halde, ekipler alakasız verileri takip ederek zaman ve kaynak kaybedebilir veya kritik tehditleri gözden kaçırabilir.
    \item \textbf{Toplama (Collection):} Bu aşama, tanımlanan gereksinimleri karşılamak için geniş bir yelpazedeki kaynaklardan ham veri toplamayı içerir. Toplanan veriler hem teknik (Indicator of Compromise - IOC) hem de bağlamsal bilgiler (TTP’ler, motivasyonlar) içermelidir.
    \begin{itemize}
        \item \textbf{OSINT (Open-Source Intelligence):} Genel kullanıma açık ve serbestçe erişilebilen kaynaklardan bilgi toplama. Bu kaynaklar arasında haberler, bloglar, sosyal medya platformları, akademik çalışmalar ve endüstri raporları bulunur.
        \item \textbf{Ticari Tehdit Beslemeleri:} Güvenlik firmaları tarafından sağlanan ve genellikle otomatik sistemlere entegre edilen ücretli veri akışlarıdır. Bu beslemeler, binlerce IOC ve TTP bilgisini yüksek hacimde sunabilir.
        \item \textbf{Bilgi Paylaşım Toplulukları (ISACs):} Aynı sektördeki veya coğrafi bölgedeki organizasyonların tehdit istihbaratı paylaşımı için bir araya geldiği güvenilir platformlardır.
        \item \textbf{Dahili Kaynaklar:} Organizasyonun kendi güvenlik araçlarından (SIEM, EDR, IDS/IPS), ağ günlüklerinden, uç nokta telemetri verilerinden ve kimlik doğrulama kayıtlarından elde edilen verilerdir.
        \item \textbf{Derin ve Karanlık Web İzleme:} Gizli veya şifreli forumlar, siber suç pazarları ve sızdırılmış veri depolarından (örneğin, sızan kimlik bilgileri, saldırı planları) bilgi toplama.
    \end{itemize}
    \item \textbf{İşleme (Processing):} Toplanan ham verinin analiz edilebilir, yapılandırılmış ve temiz bir formata dönüştürüldüğü aşamadır. Bu, veri normalizasyonu (farklı formatlardaki verileri standartlaştırma), tekilleştirme (yinelenen kayıtları kaldırma), şifre çözme ve etiketleme işlemlerini içerir. Büyük veri kümeleri için otomasyon, bu aşamada hayati öneme sahiptir.
    \item \textbf{Analiz ve Yorumlama (Analysis and Interpretation):} İşlenmiş verinin anlamlı, eyleme dönüştürülebilir istihbarata dönüştüğü aşamadır. Analistler, kalıpları, eğilimleri, saldırı kampanyalarını ve potansiyel tehditleri belirlemek için verileri derinlemesine incelerler. Bu aşamada, insan uzmanlığı ve otomatik korelasyon mekanizmaları birleşir.
    \item \textbf{Yayma (Dissemination):} Analiz edilen istihbaratın ilgili paydaşlara zamanında ve uygun bir formatta sunulmasıdır. İstihbaratın formatı ve içeriği, hedef kitleye göre uyarlanmalıdır. Örneğin, CISO'lar için üst düzey yönetici brifingleri hazırlanırken, SOC analistleri için teknik IOC beslemleri sunulur.
    \item \textbf{Geri Bildirim (Feedback):} Sürecin sürekli iyileştirilmesini sağlayan bu aşamada, istihbarat tüketicileri (örneğin, SOC analistleri), sunulan bilgilerin yararlılığı, alaka düzeyi ve doğruluğu hakkında geri bildirimde bulunur. Bu geri bildirim, bir sonraki döngüde daha iyi gereksinimlerin belirlenmesini ve istihbarat toplama stratejilerinin hassaslaştırılmasını sağlar.
\end{enumerate}

Bu yaşam döngüsü, bir organizasyonun reaktif tehdit algılama modelinden proaktif bir savunma duruşuna geçiş yapmasına olanak tanır. Bu sürekli ve döngüsel süreç, savunma mekanizmalarının saldırganların evrilen taktik, teknik ve prosedürlerine (TTP'ler) göre dinamik olarak ayarlanmasını sağlar. Bu dinamik adaptasyon, bir saldırganın bir sistemde kalma süresini (dwell time) doğrudan azaltır. Saldırganın sistemde kalma süresinin azalması, fidye yazılımı veya veri sızdırma gibi nihai hedeflere ulaşma şansını düşürürken, ihlalden kaynaklanan maliyetleri de önemli ölçüde azaltır. Bu sürekli öğrenme ve adaptasyon mekanizması, tehdit istihbaratı programının değerini ve etkinliğini katlanarak artırır.

\textbf{CTI Yaşam Döngüsü Aşamaları ve Uygulamaları}

\begin{longtable}{|p{2.5cm}|p{3cm}|p{3cm}|p{3.5cm}|}
\hline
\textbf{Aşama} & \textbf{Kısa Tanım} & \textbf{Hedef} & \textbf{Pratik Uygulama} \\
\hline
\textbf{Gereksinimler} & İhtiyaç duyulan istihbaratın belirlenmesi & Kaynakları kritik risklere odaklamak & Tehdit aktörü TTP analizi \\
\hline
\textbf{Toplama} & Ham verinin kaynaklardan alınması & Gereksinim odaklı veri toplama & SIEM günlükleri, OSINT, karanlık web \\
\hline
\textbf{İşleme} & Ham veriyi analiz formatına getirme & Veri yapılandırma ve zenginleştirme & CSV normalizasyonu, VirusTotal entegrasyonu \\
\hline
\textbf{Analiz} & Veriden içgörüler çıkarma & Tehdit bağlamsallaştırma & Oltalama kampanya teknik analizi \\
\hline
\textbf{Yayma} & İstihbaratı paydaşlara iletme & Doğru kitleye doğru formatta sunum & Yönetici brifingi, SOC beslemesi \\
\hline
\textbf{Geri Bildirim} & Etkinlik geri bildirimi toplama & Sürekli iyileştirme & Yanlış pozitif raporları, değerlendirmeler \\
\hline
\end{longtable}

\subsection{Strategic, Tactical, Technical ve Operational Intelligence}

Siber tehdit istihbaratı, hedef kitlenin ihtiyaçlarına göre dört ana kategoriye ayrılır. Bu istihbarat türleri, bir organizasyonun savunma yeteneklerini tüm seviyelerde güçlendirmek için birlikte çalışır.

\begin{itemize}
    \item \textbf{Stratejik İstihbarat:} Genel tehdit ortamına ilişkin üst düzey bir bakış açısı sunar. Bu istihbarat, teknoloji dışı terimlerle hazırlanır ve öncelikli olarak üst düzey yöneticilere, CISO'lara ve risk yöneticilerine yöneliktir.
    \begin{itemize}
        \item \textbf{Amacı:} Güvenlik yatırımları, bütçe tahsisi ve kurumsal politikalar gibi uzun vadeli stratejik kararlara rehberlik etmektir.
        \item \textbf{Örnekler:} Bir APT grubunun bir sektörü veya belirli bir coğrafyayı hedeflemesi, fidye yazılımı eğilimleri veya jeopolitik olayların siber saldırı risklerine etkisi hakkında raporlar.
    \end{itemize}
    \item \textbf{Operasyonel İstihbarat:} Belirli bir tehdit aktörünün TTP'lerini, motivasyonlarını ve altyapısını anlamayı sağlar. Bu istihbarat, olay müdahale (IR) ekipleri ve tehdit avcıları için hayati öneme sahiptir.
    \begin{itemize}
        \item \textbf{Amacı:} Yaklaşan veya devam eden bir saldırının "kim, ne zaman, nerede, nasıl, neden" sorularına bağlamsal cevaplar sunmak. Bu, proaktif tehdit avcılığı ve olay müdahale planlaması için bir temel oluşturur.
        \item \textbf{Örnekler:} Bir saldırıda kullanılan belirli bir oltalama e-postası kampanyasının detayları veya bir Komuta ve Kontrol (C2) sunucusunun iletişim yöntemleri.
    \end{itemize}
    \item \textbf{Taktik İstihbarat:} Ağ ve uç noktalarda tehditleri tespit etmeye yardımcı olan, genellikle kısa ömürlü ve teknik göstergelerden oluşan bir istihbarat türüdür.
    \begin{itemize}
        \item \textbf{Amacı:} Otomatik tehdit tespiti için SIEM ve güvenlik duvarı gibi güvenlik kontrollerine entegre edilmek.
        \item \textbf{Örnekler:} Kötü amaçlı IP adresleri, dosya karmaları (hashes), kötü amaçlı alan adları.
    \end{itemize}
    \item \textbf{Teknik İstihbarat:} Taktik istihbaratın daha derin teknik detaylarını içerir ve genellikle kötü amaçlı yazılım analizi, tersine mühendislik ve IoC'lerin oluşturulmasını kapsar.
    \begin{itemize}
        \item \textbf{Amacı:} Saldırıların teknik işleyişini anlamak ve bu bilgilere dayanarak yeni imza tabanlı tespit kuralları oluşturmaktır.
    \end{itemize}
\end{itemize}

Bu dört istihbarat türü, birbirini besleyen ve destekleyen bir hiyerarşi içinde çalışır. Üst yönetim, genel güvenlik stratejisini belirlemek için stratejik istihbarata ihtiyaç duyar. Bu strateji, operasyonel ve taktiksel gereksinimlere dönüşür. Örneğin, stratejik istihbarat, jeopolitik gerilimlerin bir APT grubunun faaliyetlerini artıracağını gösterdiğinde, operasyonel ekip bu grubun TTP'lerini incelemeye odaklanır ve bu operasyonel bilgiler taktiksel IoC'lere dönüştürülerek güvenlik kontrollerine entegre edilir. Bu entegrasyon, istihbaratın tüm organizasyonel katmanlarda değer yaratmasını sağlar.

\textbf{İstihbarat Türleri Karşılaştırma Tablosu}

\begin{longtable}{|p{2cm}|p{3.5cm}|p{3.5cm}|p{3cm}|}
\hline
\textbf{Özellik} & \textbf{Stratejik} & \textbf{Operasyonel} & \textbf{Taktik} \\
\hline
\textbf{Hedef Kitle} & C-seviyesi yöneticiler & IR ekipleri, tehdit avcıları & SOC analistleri \\
\hline
\textbf{Odak Noktası} & Genel eğilimler, risk yönetimi & Saldırı kampanyaları, TTP'ler & IOC'ler, ağ etkinlikleri \\
\hline
\textbf{Zaman} & Uzun vadeli (aylar/yıllar) & Orta vadeli (hafta/ay) & Kısa vadeli (gerçek zamanlı) \\
\hline
\textbf{Amaç} & Politika ve yatırım belirleme & Müdahale ve proaktif avcılık & Otomatik tespit ve engelleme \\
\hline
\end{longtable}

\subsection{Threat Actor Profiling ve Attribution Challenges}

Tehdit aktörü profilleme, saldırıların arkasındaki "kim" sorusuna cevap aramayı amaçlar. Bir tehdit aktörü profili, sadece bir isimden ibaret değildir; bir saldırganın kimliğini, hedeflerini, TTP'lerini, motivasyonlarını, coğrafi konumunu ve kullandığı altyapıyı kapsamlı bir şekilde analiz eder. Bu profiller, güvenlik ekiplerine saldırganın olası davranışları hakkında bir resim sunar ve savunma stratejilerini saldırganın niyetleriyle uyumlu hale getirmeye yardımcı olur.

Ancak, bir siber saldırıyı kesin olarak belirli bir tehdit aktörüne atfetmek (attribution), son derece karmaşık ve zorlu bir süreçtir. Bu zorlukların birkaç temel nedeni vardır:

\begin{itemize}
    \item \textbf{Gizleme ve Obfüskasyon:} Saldırganlar, kimliklerini gizlemek için çok sayıda katman kullanır. Botnetler, vekil sunucular ve kiralanmış altyapı, saldırının gerçek kaynağını maskeler. Özellikle dağıtık servis dışı bırakma (DDoS) saldırılarında, trafik binlerce farklı cihazdan gelebilir, bu da kesin bir atıf yapmayı neredeyse imkansız hale getirir.
    \item \textbf{Yanlış Bayrak Operasyonları:} Bazı siber suç grupları, dikkat çekmek veya yanlış bilgi yaymak amacıyla gerçekleştirmediği saldırıların sorumluluğunu üstlenebilir. Bu tür eylemler, atıf sürecini manipüle etmeyi amaçlar.
    \item \textbf{Araç ve TTP Paylaşımı:} Tehdit grupları, sıklıkla araç setlerini ve TTP'lerini birbirleriyle paylaşır veya değiştirir. Bir saldırıda kullanılan belirli bir teknik, daha önce bilinen bir grubun imzası olsa bile, başka bir grup tarafından da kullanılmış olabilir. Bu durum, yalnızca teknik IOC'lere dayalı atıf yapmanın güvenilirliğini azaltır.
\end{itemize}

Bu nedenle, atıf birincil hedef olmamalıdır. Bir saldırı meydana geldiğinde, ilk ve en acil öncelik, hasarı durdurmak, sistemleri güvence altına almak ve devam eden tehditleri ortadan kaldırmaktır. Atıf, bu acil riskler ortadan kaldırıldıktan sonra, olay sonrası analiz aşamasında daha sonraki savunmaları güçlendirmek için bir araç olarak ele alınmalıdır. Atıf, bir "evet/hayır" cevabı yerine, farklı güven seviyelerine sahip (düşük, orta, yüksek) analitik bir değerlendirme süreci olarak görülmelidir. Güvenilir bir atıf için, sadece IOC'ler yerine, davranışsal kanıtlar ve TTP'lere odaklanmak daha geçerli bir yaklaşımdır.

\subsection{Diamond Model ve Kill Chain Analysis}

Siber saldırıları anlamak ve analiz etmek için kullanılan iki önemli analitik çerçeve, Lockheed Martin'in Siber Kill Chain’i ve MITRE’nin Saldırı Analizinin Elmas Modeli’dir (Diamond Model of Intrusion Analysis). Bu modeller, saldırı sürecini farklı açılardan ele alarak birbirini tamamlar.

\begin{itemize}
    \item \textbf{Siber Kill Chain (Siber Saldırı Zinciri):} Bu model, bir saldırının yedi aşamalı, doğrusal bir sürecini sunar. Her aşama, bir saldırganın başarılı bir siber saldırı gerçekleştirmek için tipik olarak izlediği adımları tanımlar.
    \begin{enumerate}
        \item \textbf{Keşif (Reconnaissance):} Saldırgan, hedefin sistemleri, ağ yapısı ve çalışanları hakkında bilgi toplar.
        \item \textbf{Silahlandırma (Weaponization):} Bir exploit ve bir arka kapı (payload) bir araya getirilerek tek bir saldırı paketi oluşturulur.
        \item \textbf{Teslimat (Delivery):} Saldırı paketi, hedef sisteme iletilir (örneğin, oltalama e-postası veya kötü amaçlı web sitesi aracılığıyla).
        \item \textbf{İstismar (Exploitation):} Bir sistemdeki zafiyet kullanılarak ilk erişim elde edilir.
        \item \textbf{Kurulum (Installation):} Saldırgan, sistemde kalıcılığı sağlamak için arka kapıyı kurar.
        \item \textbf{Komuta ve Kontrol (Command and Control):} Saldırganın ele geçirilmiş sistemle uzaktan iletişim kurarak onu kontrol etmesini sağlar.
        \item \textbf{Hedefler Üzerindeki Eylemler (Actions on Objectives):} Saldırgan, veri sızdırma, sistem bozulması veya fidye gibi nihai hedeflerine ulaşır.
    \end{enumerate}
    Kill Chain, saldırıların nasıl ilerlediğine dair adım adım bir yol haritası sunar ve özellikle olay müdahalesi ve taktiksel savunma kararları için çok değerlidir.
    \item \textbf{Diamond Model of Intrusion Analysis (Saldırı Analizinin Elmas Modeli):} Bu model, siber saldırıları dört temel bileşen arasındaki ilişkilere odaklanarak inceler. Doğrusal bir model değildir, daha çok saldırının bütüncül bir resmini sunar.
    \begin{enumerate}
        \item \textbf{Saldırgan (Adversary):} Saldırının arkasındaki tehdit aktörü.
        \item \textbf{Kabiliyet (Capability):} Saldırganın kullandığı araçlar, teknikler ve yöntemler.
        \item \textbf{Altyapı (Infrastructure):} Saldırıyı desteklemek için kullanılan ağ altyapısı (örneğin, C2 sunucuları, vekil sunucular).
        \item \textbf{Kurban (Victim):} Saldırıya uğrayan kişi, organizasyon veya sistem.
    \end{enumerate}
    Elmas Modeli, tehdit istihbaratı ve proaktif tehdit avcılığı için daha kullanışlıdır. Saldırıdan elde edilen herhangi bir bilgi parçası (örneğin, bir C2 sunucusu), diğer üç bileşenle ilişkilendirilerek saldırganın potansiyel diğer faaliyetlerini ve gelecekteki olası hedeflerini tahmin etmek için kullanılabilir.
\end{itemize}

Kill Chain, bir saldırının \textit{nasıl} ilerlediğini detaylandırırken, Diamond Model saldırının \textit{neden} ve \textit{kim tarafından} yapıldığını daha geniş bir bakış açısı sunar. Bu modeller birbirini tamamlar. Bir saldırı tespit edildiğinde, Kill Chain modeli olayın taktiksel olarak yönetilmesine yardımcı olurken, Diamond Model olayın bileşenlerini analiz ederek bu olayı bilinen diğer tehdit gruplarıyla ilişkilendirmeyi ve gelecekteki potansiyel saldırıları tahmin etmeyi mümkün kılar.

% Kill Chain ve Diamond Model Karşılaştırma Tablosu
\begin{longtable}{|p{3cm}|p{5.5cm}|p{5.5cm}|}
\hline
\textbf{Özellik} & \textbf{Siber Kill Chain} & \textbf{Diamond Model} \\
\hline
\textbf{Odak} & Saldırı aşamaları ve süreci & Aktör-kabiliyet-altyapı-kurban ilişkileri \\
\hline
\textbf{Granülarite} & Spesifik ve doğrusal & Geniş, bağlamsal ve ilişkisel \\
\hline
\textbf{Uygulama} & Olay müdahalesi, taktiksel savunma & Tehdit istihbaratı, proaktif avcılık \\
\hline
\textbf{Avantajlar} & Adım adım müdahale yol haritası & Motivasyon ve kabiliyet analizi \\
\hline
\end{longtable}

\subsection{Threat Intelligence Lifecycle ve Collection Methods}

Tehdit istihbaratı yaşam döngüsü, ham veriyi değerli istihbarata dönüştüren ve sürekli bir geri bildirim döngüsü ile kendini yenileyen, yapılandırılmış bir süreçtir. Bu döngü, CTI programının temelini oluşturur ve reaktif bir modelden proaktif bir savunma yaklaşımına geçişin temelini atar.

\textbf{Yaşam Döngüsü Aşamaları:}

\begin{enumerate}
    \item \textbf{Gereksinimler (Requirements):} Döngünün ilk ve en kritik aşamasıdır. Bu aşamada, güvenlik ekipleri, iş birimleri ve üst düzey yöneticiler bir araya gelerek istihbarat ihtiyaçlarını net bir şekilde tanımlar. Bu, korunması gereken en kritik varlıkları (“Crown Jewels”), organizasyonun karşılaştığı riskleri ve bu riskleri azaltmak için hangi bilgilerin gerekli olduğunu belirlemeyi içerir. Gereksinimlerin net bir şekilde belirlenmesi, istihbarat toplama çabalarının boşa gitmesini engeller ve kaynakların doğru hedeflere yönlendirilmesini sağlar. Aksi halde, ekipler alakasız verileri takip ederek zaman ve kaynak kaybedebilir veya kritik tehditleri gözden kaçırabilir.
    \item \textbf{Toplama (Collection):} Bu aşama, tanımlanan gereksinimleri karşılamak için geniş bir yelpazedeki kaynaklardan ham veri toplamayı içerir. Toplanan veriler hem teknik (Indicator of Compromise - IOC) hem de bağlamsal bilgiler (TTP’ler, motivasyonlar) içermelidir.
    \begin{itemize}
        \item \textbf{OSINT (Open-Source Intelligence):} Genel kullanıma açık ve serbestçe erişilebilen kaynaklardan bilgi toplama. Bu kaynaklar arasında haberler, bloglar, sosyal medya platformları, akademik çalışmalar ve endüstri raporları bulunur.
        \item \textbf{Ticari Tehdit Beslemeleri:} Güvenlik firmaları tarafından sağlanan ve genellikle otomatik sistemlere entegre edilen ücretli veri akışlarıdır. Bu beslemeler, binlerce IOC ve TTP bilgisini yüksek hacimde sunabilir.
        \item \textbf{Bilgi Paylaşım Toplulukları (ISACs):} Aynı sektördeki veya coğrafi bölgedeki organizasyonların tehdit istihbaratı paylaşımı için bir araya geldiği güvenilir platformlardır.
        \item \textbf{Dahili Kaynaklar:} Organizasyonun kendi güvenlik araçlarından (SIEM, EDR, IDS/IPS), ağ günlüklerinden, uç nokta telemetri verilerinden ve kimlik doğrulama kayıtlarından elde edilen verilerdir.
        \item \textbf{Derin ve Karanlık Web İzleme:} Gizli veya şifreli forumlar, siber suç pazarları ve sızdırılmış veri depolarından (örneğin, sızan kimlik bilgileri, saldırı planları) bilgi toplama.
    \end{itemize}
    \item \textbf{İşleme (Processing):} Toplanan ham verinin analiz edilebilir, yapılandırılmış ve temiz bir formata dönüştürüldüğü aşamadır. Bu, veri normalizasyonu (farklı formatlardaki verileri standartlaştırma), tekilleştirme (yinelenen kayıtları kaldırma), şifre çözme ve etiketleme işlemlerini içerir. Büyük veri kümeleri için otomasyon, bu aşamada hayati öneme sahiptir.
    \item \textbf{Analiz ve Yorumlama (Analysis and Interpretation):} İşlenmiş verinin anlamlı, eyleme dönüştürülebilir istihbarata dönüştüğü aşamadır. Analistler, kalıpları, eğilimleri, saldırı kampanyalarını ve potansiyel tehditleri belirlemek için verileri derinlemesine incelerler. Bu aşamada, insan uzmanlığı ve otomatik korelasyon mekanizmaları birleşir.
    \item \textbf{Yayma (Dissemination):} Analiz edilen istihbaratın ilgili paydaşlara zamanında ve uygun bir formatta sunulmasıdır. İstihbaratın formatı ve içeriği, hedef kitleye göre uyarlanmalıdır. Örneğin, CISO'lar için üst düzey yönetici brifingleri hazırlanırken, SOC analistleri için teknik IOC beslemleri sunulur.
    \item \textbf{Geri Bildirim (Feedback):} Sürecin sürekli iyileştirilmesini sağlayan bu aşamada, istihbarat tüketicileri (örneğin, SOC analistleri), sunulan bilgilerin yararlılığı, alaka düzeyi ve doğruluğu hakkında geri bildirimde bulunur. Bu geri bildirim, bir sonraki döngüde daha iyi gereksinimlerin belirlenmesini ve istihbarat toplama stratejilerinin hassaslaştırılmasını sağlar.
\end{enumerate}

Bu yaşam döngüsü, bir organizasyonun reaktif tehdit algılama modelinden proaktif bir savunma duruşuna geçiş yapmasına olanak tanır. Bu sürekli ve döngüsel süreç, savunma mekanizmalarının saldırganların evrilen taktik, teknik ve prosedürlerine (TTP'ler) göre dinamik olarak ayarlanmasını sağlar. Bu dinamik adaptasyon, bir saldırganın bir sistemde kalma süresini (dwell time) doğrudan azaltır. Saldırganın sistemde kalma süresinin azalması, fidye yazılımı veya veri sızdırma gibi nihai hedeflere ulaşma şansını düşürürken, ihlalden kaynaklanan maliyetleri de önemli ölçüde azaltır. Bu sürekli öğrenme ve adaptasyon mekanizması, tehdit istihbaratı programının değerini ve etkinliğini katlanarak artırır.

\textbf{CTI Yaşam Döngüsü Aşamaları ve Uygulamaları}

\begin{longtable}{|p{2.5cm}|p{3cm}|p{3cm}|p{3.5cm}|}
\hline
\textbf{Aşama} & \textbf{Kısa Tanım} & \textbf{Hedef} & \textbf{Pratik Uygulama} \\
\hline
\textbf{Gereksinimler} & İhtiyaç duyulan istihbaratın belirlenmesi & Kaynakları kritik risklere odaklamak & Tehdit aktörü TTP analizi \\
\hline
\textbf{Toplama} & Ham verinin kaynaklardan alınması & Gereksinim odaklı veri toplama & SIEM günlükleri, OSINT, karanlık web \\
\hline
\textbf{İşleme} & Ham veriyi analiz formatına getirme & Veri yapılandırma ve zenginleştirme & CSV normalizasyonu, VirusTotal entegrasyonu \\
\hline
\textbf{Analiz} & Veriden içgörüler çıkarma & Tehdit bağlamsallaştırma & Oltalama kampanya teknik analizi \\
\hline
\textbf{Yayma} & İstihbaratı paydaşlara iletme & Doğru kitleye doğru formatta sunum & Yönetici brifingi, SOC beslemesi \\
\hline
\textbf{Geri Bildirim} & Etkinlik geri bildirimi toplama & Sürekli iyileştirme & Yanlış pozitif raporları, değerlendirmeler \\
\hline
\end{longtable}

\subsection{Strategic, Tactical, Technical ve Operational Intelligence}

Siber tehdit istihbaratı, hedef kitlenin ihtiyaçlarına göre dört ana kategoriye ayrılır. Bu istihbarat türleri, bir organizasyonun savunma yeteneklerini tüm seviyelerde güçlendirmek için birlikte çalışır.

\begin{itemize}
    \item \textbf{Stratejik İstihbarat:} Genel tehdit ortamına ilişkin üst düzey bir bakış açısı sunar. Bu istihbarat, teknoloji dışı terimlerle hazırlanır ve öncelikli olarak üst düzey yöneticilere, CISO'lara ve risk yöneticilerine yöneliktir.
    \begin{itemize}
        \item \textbf{Amacı:} Güvenlik yatırımları, bütçe tahsisi ve kurumsal politikalar gibi uzun vadeli stratejik kararlara rehberlik etmektir.
        \item \textbf{Örnekler:} Bir APT grubunun bir sektörü veya belirli bir coğrafyayı hedeflemesi, fidye yazılımı eğilimleri veya jeopolitik olayların siber saldırı risklerine etkisi hakkında raporlar.
    \end{itemize}
    \item \textbf{Operasyonel İstihbarat:} Belirli bir tehdit aktörünün TTP'lerini, motivasyonlarını ve altyapısını anlamayı sağlar. Bu istihbarat, olay müdahale (IR) ekipleri ve tehdit avcıları için hayati öneme sahiptir.
    \begin{itemize}
        \item \textbf{Amacı:} Yaklaşan veya devam eden bir saldırının "kim, ne zaman, nerede, nasıl, neden" sorularına bağlamsal cevaplar sunmak. Bu, proaktif tehdit avcılığı ve olay müdahale planlaması için bir temel oluşturur.
        \item \textbf{Örnekler:} Bir saldırıda kullanılan belirli bir oltalama e-postası kampanyasının detayları veya bir Komuta ve Kontrol (C2) sunucusunun iletişim yöntemleri.
    \end{itemize}
    \item \textbf{Taktik İstihbarat:} Ağ ve uç noktalarda tehditleri tespit etmeye yardımcı olan, genellikle kısa ömürlü ve teknik göstergelerden oluşan bir istihbarat türüdür.
    \begin{itemize}
        \item \textbf{Amacı:} Otomatik tehdit tespiti için SIEM ve güvenlik duvarı gibi güvenlik kontrollerine entegre edilmek.
        \item \textbf{Örnekler:} Kötü amaçlı IP adresleri, dosya karmaları (hashes), kötü amaçlı alan adları.
    \end{itemize}
    \item \textbf{Teknik İstihbarat:} Taktik istihbaratın daha derin teknik detaylarını içerir ve genellikle kötü amaçlı yazılım analizi, tersine mühendislik ve IoC'lerin oluşturulmasını kapsar.
    \begin{itemize}
        \item \textbf{Amacı:} Saldırıların teknik işleyişini anlamak ve bu bilgilere dayanarak yeni imza tabanlı tespit kuralları oluşturmaktır.
    \end{itemize}
\end{itemize}

Bu dört istihbarat türü, birbirini besleyen ve destekleyen bir hiyerarşi içinde çalışır. Üst yönetim, genel güvenlik stratejisini belirlemek için stratejik istihbarata ihtiyaç duyar. Bu strateji, operasyonel ve taktiksel gereksinimlere dönüşür. Örneğin, stratejik istihbarat, jeopolitik gerilimlerin bir APT grubunun faaliyetlerini artıracağını gösterdiğinde, operasyonel ekip bu grubun TTP'lerini incelemeye odaklanır ve bu operasyonel bilgiler taktiksel IoC'lere dönüştürülerek güvenlik kontrollerine entegre edilir. Bu entegrasyon, istihbaratın tüm organizasyonel katmanlarda değer yaratmasını sağlar.

\textbf{İstihbarat Türleri Karşılaştırma Tablosu}

\begin{longtable}{|p{2cm}|p{3.5cm}|p{3.5cm}|p{3cm}|}
\hline
\textbf{Özellik} & \textbf{Stratejik} & \textbf{Operasyonel} & \textbf{Taktik} \\
\hline
\textbf{Hedef Kitle} & C-seviyesi yöneticiler & IR ekipleri, tehdit avcıları & SOC analistleri \\
\hline
\textbf{Odak Noktası} & Genel eğilimler, risk yönetimi & Saldırı kampanyaları, TTP'ler & IOC'ler, ağ etkinlikleri \\
\hline
\textbf{Zaman} & Uzun vadeli (aylar/yıllar) & Orta vadeli (hafta/ay) & Kısa vadeli (gerçek zamanlı) \\
\hline
\textbf{Amaç} & Politika ve yatırım belirleme & Müdahale ve proaktif avcılık & Otomatik tespit ve engelleme \\
\hline
\end{longtable}

\subsection{Threat Actor Profiling ve Attribution Challenges}

Tehdit aktörü profilleme, saldırıların arkasındaki "kim" sorusuna cevap aramayı amaçlar. Bir tehdit aktörü profili, sadece bir isimden ibaret değildir; bir saldırganın kimliğini, hedeflerini, TTP'lerini, motivasyonlarını, coğrafi konumunu ve kullandığı altyapıyı kapsamlı bir şekilde analiz eder. Bu profiller, güvenlik ekiplerine saldırganın olası davranışları hakkında bir resim sunar ve savunma stratejilerini saldırganın niyetleriyle uyumlu hale getirmeye yardımcı olur.

Ancak, bir siber saldırıyı kesin olarak belirli bir tehdit aktörüne atfetmek (attribution), son derece karmaşık ve zorlu bir süreçtir. Bu zorlukların birkaç temel nedeni vardır:

\begin{itemize}
    \item \textbf{Gizleme ve Obfüskasyon:} Saldırganlar, kimliklerini gizlemek için çok sayıda katman kullanır. Botnetler, vekil sunucular ve kiralanmış altyapı, saldırının gerçek kaynağını maskeler. Özellikle dağıtık servis dışı bırakma (DDoS) saldırılarında, trafik binlerce farklı cihazdan gelebilir, bu da kesin bir atıf yapmayı neredeyse imkansız hale getirir.
    \item \textbf{Yanlış Bayrak Operasyonları:} Bazı siber suç grupları, dikkat çekmek veya yanlış bilgi yaymak amacıyla gerçekleştirmediği saldırıların sorumluluğunu üstlenebilir. Bu tür eylemler, atıf sürecini manipüle etmeyi amaçlar.
    \item \textbf{Araç ve TTP Paylaşımı:} Tehdit grupları, sıklıkla araç setlerini ve TTP'lerini birbirleriyle paylaşır veya değiştirir. Bir saldırıda kullanılan belirli bir teknik, daha önce bilinen bir grubun imzası olsa bile, başka bir grup tarafından da kullanılmış olabilir. Bu durum, yalnızca teknik IOC'lere dayalı atıf yapmanın güvenilirliğini azaltır.
\end{itemize}

Bu nedenle, atıf birincil hedef olmamalıdır. Bir saldırı meydana geldiğinde, ilk ve en acil öncelik, hasarı durdurmak, sistemleri güvence altına almak ve devam eden tehditleri ortadan kaldırmaktır. Atıf, bu acil riskler ortadan kaldırıldıktan sonra, olay sonrası analiz aşamasında daha sonraki savunmaları güçlendirmek için bir araç olarak ele alınmalıdır. Atıf, bir "evet/hayır" cevabı yerine, farklı güven seviyelerine sahip (düşük, orta, yüksek) analitik bir değerlendirme süreci olarak görülmelidir. Güvenilir bir atıf için, sadece IOC'ler yerine, davranışsal kanıtlar ve TTP'lere odaklanmak daha geçerli bir yaklaşımdır.

\subsection{Diamond Model ve Kill Chain Analysis}

Siber saldırıları anlamak ve analiz etmek için kullanılan iki önemli analitik çerçeve, Lockheed Martin'in Siber Kill Chain’i ve MITRE’nin Saldırı Analizinin Elmas Modeli’dir (Diamond Model of Intrusion Analysis). Bu modeller, saldırı sürecini farklı açılardan ele alarak birbirini tamamlar.

\begin{itemize}
    \item \textbf{Siber Kill Chain (Siber Saldırı Zinciri):} Bu model, bir saldırının yedi aşamalı, doğrusal bir sürecini sunar. Her aşama, bir saldırganın başarılı bir siber saldırı gerçekleştirmek için tipik olarak izlediği adımları tanımlar.
    \begin{enumerate}
        \item \textbf{Keşif (Reconnaissance):} Saldırgan, hedefin sistemleri, ağ yapısı ve çalışanları hakkında bilgi toplar.
        \item \textbf{Silahlandırma (Weaponization):} Bir exploit ve bir arka kapı (payload) bir araya getirilerek tek bir saldırı paketi oluşturulur.
        \item \textbf{Teslimat (Delivery):} Saldırı paketi, hedef sisteme iletilir (örneğin, oltalama e-postası veya kötü amaçlı web sitesi aracılığıyla).
        \item \textbf{İstismar (Exploitation):} Bir sistemdeki zafiyet kullanılarak ilk erişim elde edilir.
        \item \textbf{Kurulum (Installation):} Saldırgan, sistemde kalıcılığı sağlamak için arka kapıyı kurar.
        \item \textbf{Komuta ve Kontrol (Command and Control):} Saldırganın ele geçirilmiş sistemle uzaktan iletişim kurarak onu kontrol etmesini sağlar.
        \item \textbf{Hedefler Üzerindeki Eylemler (Actions on Objectives):} Saldırgan, veri sızdırma, sistem bozulması veya fidye gibi nihai hedeflerine ulaşır.
    \end{enumerate}
    Kill Chain, saldırıların nasıl ilerlediğine dair adım adım bir yol haritası sunar ve özellikle olay müdahalesi ve taktiksel savunma kararları için çok değerlidir.
    \item \textbf{Diamond Model of Intrusion Analysis (Saldırı Analizinin Elmas Modeli):} Bu model, siber saldırıları dört temel bileşen arasındaki ilişkilere odaklanarak inceler. Doğrusal bir model değildir, daha çok saldırının bütüncül bir resmini sunar.
    \begin{enumerate}
        \item \textbf{Saldırgan (Adversary):} Saldırının arkasındaki tehdit aktörü.
        \item \textbf{Kabiliyet (Capability):} Saldırganın kullandığı araçlar, teknikler ve yöntemler.
        \item \textbf{Altyapı (Infrastructure):} Saldırıyı desteklemek için kullanılan ağ altyapısı (örneğin, C2 sunucuları, vekil sunucular).
        \item \textbf{Kurban (Victim):} Saldırıya uğrayan kişi, organizasyon veya sistem.
    \end{enumerate}
    Elmas Modeli, tehdit istihbaratı ve proaktif tehdit avcılığı için daha kullanışlıdır. Saldırıdan elde edilen herhangi bir bilgi parçası (örneğin, bir C2 sunucusu), diğer üç bileşenle ilişkilendirilerek saldırganın potansiyel diğer faaliyetlerini ve gelecekteki olası hedeflerini tahmin etmek için kullanılabilir.
\end{itemize}

Kill Chain, bir saldırının \textit{nasıl} ilerlediğini detaylandırırken, Diamond Model saldırının \textit{neden} ve \textit{kim tarafından} yapıldığını daha geniş bir bakış açısı sunar. Bu modeller birbirini tamamlar. Bir saldırı tespit edildiğinde, Kill Chain modeli olayın taktiksel olarak yönetilmesine yardımcı olurken, Diamond Model olayın bileşenlerini analiz ederek bu olayı bilinen diğer tehdit gruplarıyla ilişkilendirmeyi ve gelecekteki potansiyel saldırıları tahmin etmeyi mümkün kılar.

% Kill Chain ve Diamond Model Karşılaştırma Tablosu
\begin{longtable}{|p{3cm}|p{5.5cm}|p{5.5cm}|}
\hline
\textbf{Özellik} & \textbf{Siber Kill Chain} & \textbf{Diamond Model} \\
\hline
\textbf{Odak} & Saldırı aşamaları ve süreci & Aktör-kabiliyet-altyapı-kurban ilişkileri \\
\hline
\textbf{Granülarite} & Spesifik ve doğrusal & Geniş, bağlamsal ve ilişkisel \\
\hline
\textbf{Uygulama} & Olay müdahalesi, taktiksel savunma & Tehdit istihbaratı, proaktif avcılık \\
\hline
\textbf{Avantajlar} & Adım adım müdahale yol haritası & Motivasyon ve kabiliyet analizi \\
\hline
\end{longtable}

\subsection{Threat Intelligence Lifecycle ve Collection Methods}

Tehdit istihbaratı yaşam döngüsü, ham veriyi değerli istihbarata dönüştüren ve sürekli bir geri bildirim döngüsü ile kendini yenileyen, yapılandırılmış bir süreçtir. Bu döngü, CTI programının temelini oluşturur ve reaktif bir modelden proaktif bir savunma yaklaşımına geçişin temelini atar.

\textbf{Yaşam Döngüsü Aşamaları:}

\begin{enumerate}
    \item \textbf{Gereksinimler (Requirements):} Döngünün ilk ve en kritik aşamasıdır. Bu aşamada, güvenlik ekipleri, iş birimleri ve üst düzey yöneticiler bir araya gelerek istihbarat ihtiyaçlarını net bir şekilde tanımlar. Bu, korunması gereken en kritik varlıkları (“Crown Jewels”), organizasyonun karşılaştığı riskleri ve bu riskleri azaltmak için hangi bilgilerin gerekli olduğunu belirlemeyi içerir. Gereksinimlerin net bir şekilde belirlenmesi, istihbarat toplama çabalarının boşa gitmesini engeller ve kaynakların doğru hedeflere yönlendirilmesini sağlar. Aksi halde, ekipler alakasız verileri takip ederek zaman ve kaynak kaybedebilir veya kritik tehditleri gözden kaçırabilir.
    \item \textbf{Toplama (Collection):} Bu aşama, tanımlanan gereksinimleri karşılamak için geniş bir yelpazedeki kaynaklardan ham veri toplamayı içerir. Toplanan veriler hem teknik (Indicator of Compromise - IOC) hem de bağlamsal bilgiler (TTP’ler, motivasyonlar) içermelidir.
    \begin{itemize}
        \item \textbf{OSINT (Open-Source Intelligence):} Genel kullanıma açık ve serbestçe erişilebilen kaynaklardan bilgi toplama. Bu kaynaklar arasında haberler, bloglar, sosyal medya platformları, akademik çalışmalar ve endüstri raporları bulunur.
        \item \textbf{Ticari Tehdit Beslemeleri:} Güvenlik firmaları tarafından sağlanan ve genellikle otomatik sistemlere entegre edilen ücretli veri akışlarıdır. Bu beslemeler, binlerce IOC ve TTP bilgisini yüksek hacimde sunabilir.
        \item \textbf{Bilgi Paylaşım Toplulukları (ISACs):} Aynı sektördeki veya coğrafi bölgedeki organizasyonların tehdit istihbaratı paylaşımı için bir araya geldiği güvenilir platformlardır.
        \item \textbf{Dahili Kaynaklar:} Organizasyonun kendi güvenlik araçlarından (SIEM, EDR, IDS/IPS), ağ günlüklerinden, uç nokta telemetri verilerinden ve kimlik doğrulama kayıtlarından elde edilen verilerdir.
        \item \textbf{Derin ve Karanlık Web İzleme:} Gizli veya şifreli forumlar, siber suç pazarları ve sızdırılmış veri depolarından (örneğin, sızan kimlik bilgileri, saldırı planları) bilgi toplama.
    \end{itemize}
    \item \textbf{İşleme (Processing):} Toplanan ham verinin analiz edilebilir, yapılandırılmış ve temiz bir formata dönüştürüldüğü aşamadır. Bu, veri normalizasyonu (farklı formatlardaki verileri standartlaştırma), tekilleştirme (yinelenen kayıtları kaldırma), şifre çözme ve etiketleme işlemlerini içerir. Büyük veri kümeleri için otomasyon, bu aşamada hayati öneme sahiptir.
    \item \textbf{Analiz ve Yorumlama (Analysis and Interpretation):} İşlenmiş verinin anlamlı, eyleme dönüştürülebilir istihbarata dönüştüğü aşamadır. Analistler, kalıpları, eğilimleri, saldırı kampanyalarını ve potansiyel tehditleri belirlemek için verileri derinlemesine incelerler. Bu aşamada, insan uzmanlığı ve otomatik korelasyon mekanizmaları birleşir.
    \item \textbf{Yayma (Dissemination):} Analiz edilen istihbaratın ilgili paydaşlara zamanında ve uygun bir formatta sunulmasıdır. İstihbaratın formatı ve içeriği, hedef kitleye göre uyarlanmalıdır. Örneğin, CISO'lar için üst düzey yönetici brifingleri hazırlanırken, SOC analistleri için teknik IOC beslemleri sunulur.
    \item \textbf{Geri Bildirim (Feedback):} Sürecin sürekli iyileştirilmesini sağlayan bu aşamada, istihbarat tüketicileri (örneğin, SOC analistleri), sunulan bilgilerin yararlılığı, alaka düzeyi ve doğruluğu hakkında geri bildirimde bulunur. Bu geri bildirim, bir sonraki döngüde daha iyi gereksinimlerin belirlenmesini ve istihbarat toplama stratejilerinin hassaslaştırılmasını sağlar.
\end{enumerate}

Bu yaşam döngüsü, bir organizasyonun reaktif tehdit algılama modelinden proaktif bir savunma duruşuna geçiş yapmasına olanak tanır. Bu sürekli ve döngüsel süreç, savunma mekanizmalarının saldırganların evrilen taktik, teknik ve prosedürlerine (TTP'ler) göre dinamik olarak ayarlanmasını sağlar. Bu dinamik adaptasyon, bir saldırganın bir sistemde kalma süresini (dwell time) doğrudan azaltır. Saldırganın sistemde kalma süresinin azalması, fidye yazılımı veya veri sızdırma gibi nihai hedeflere ulaşma şansını düşürürken, ihlalden kaynaklanan maliyetleri de önemli ölçüde azaltır. Bu sürekli öğrenme ve adaptasyon mekanizması, tehdit istihbaratı programının değerini ve etkinliğini katlanarak artırır.

\textbf{CTI Yaşam Döngüsü Aşamaları ve Uygulamaları}

\begin{longtable}{|p{2.5cm}|p{3cm}|p{3cm}|p{3.5cm}|}
\hline
\textbf{Aşama} & \textbf{Kısa Tanım} & \textbf{Hedef} & \textbf{Pratik Uygulama} \\
\hline
\textbf{Gereksinimler} & İhtiyaç duyulan istihbaratın belirlenmesi & Kaynakları kritik risklere odaklamak & Tehdit aktörü TTP analizi \\
\hline
\textbf{Toplama} & Ham verinin kaynaklardan alınması & Gereksinim odaklı veri toplama & SIEM günlükleri, OSINT, karanlık web \\
\hline
\textbf{İşleme} & Ham veriyi analiz formatına getirme & Veri yapılandırma ve zenginleştirme & CSV normalizasyonu, VirusTotal entegrasyonu \\
\hline
\textbf{Analiz} & Veriden içgörüler çıkarma & Tehdit bağlamsallaştırma & Oltalama kampanya teknik analizi \\
\hline
\textbf{Yayma} & İstihbaratı paydaşlara iletme & Doğru kitleye doğru formatta sunum & Yönetici brifingi, SOC beslemesi \\
\hline
\textbf{Geri Bildirim} & Etkinlik geri bildirimi toplama & Sürekli iyileştirme & Yanlış pozitif raporları, değerlendirmeler \\
\hline
\end{longtable}

\subsection{Strategic, Tactical, Technical ve Operational Intelligence}

Siber tehdit istihbaratı, hedef kitlenin ihtiyaçlarına göre dört ana kategoriye ayrılır. Bu istihbarat türleri, bir organizasyonun savunma yeteneklerini tüm seviyelerde güçlendirmek için birlikte çalışır.

\begin{itemize}
    \item \textbf{Stratejik İstihbarat:} Genel tehdit ortamına ilişkin üst düzey bir bakış açısı sunar. Bu istihbarat, teknoloji dışı terimlerle hazırlanır ve öncelikli olarak üst düzey yöneticilere, CISO'lara ve risk yöneticilerine yöneliktir.
    \begin{itemize}
        \item \textbf{Amacı:} Güvenlik yatırımları, bütçe tahsisi ve kurumsal politikalar gibi uzun vadeli stratejik kararlara rehberlik etmektir.
        \item \textbf{Örnekler:} Bir APT grubunun bir sektörü veya belirli bir coğrafyayı hedeflemesi, fidye yazılımı eğilimleri veya jeopolitik olayların siber saldırı risklerine etkisi hakkında raporlar.
    \end{itemize}
    \item \textbf{Operasyonel İstihbarat:} Belirli bir tehdit aktörünün TTP'lerini, motivasyonlarını ve altyapısını anlamayı sağlar. Bu istihbarat, olay müdahale (IR) ekipleri ve tehdit avcıları için hayati öneme sahiptir.
    \begin{itemize}
        \item \textbf{Amacı:} Yaklaşan veya devam eden bir saldırının "kim, ne zaman, nerede, nasıl, neden" sorularına bağlamsal cevaplar sunmak. Bu, proaktif tehdit avcılığı ve olay müdahale planlaması için bir temel oluşturur.
        \item \textbf{Örnekler:} Bir saldırıda kullanılan belirli bir oltalama e-postası kampanyasının detayları veya bir Komuta ve Kontrol (C2) sunucusunun iletişim yöntemleri.
    \end{itemize}
    \item \textbf{Taktik İstihbarat:} Ağ ve uç noktalarda tehditleri tespit etmeye yardımcı olan, genellikle kısa ömürlü ve teknik göstergelerden oluşan bir istihbarat türüdür.
    \begin{itemize}
        \item \textbf{Amacı:} Otomatik tehdit tespiti için SIEM ve güvenlik duvarı gibi güvenlik kontrollerine entegre edilmek.
        \item \textbf{Örnekler:} Kötü amaçlı IP adresleri, dosya karmaları (hashes), kötü amaçlı alan adları.
    \end{itemize}
    \item \textbf{Teknik İstihbarat:} Taktik istihbaratın daha derin teknik detaylarını içerir ve genellikle kötü amaçlı yazılım analizi, tersine mühendislik ve IoC'lerin oluşturulmasını kapsar.
    \begin{itemize}
        \item \textbf{Amacı:} Saldırıların teknik işleyişini anlamak ve bu bilgilere dayanarak yeni imza tabanlı tespit kuralları oluşturmaktır.
    \end{itemize}
\end{itemize}

Bu dört istihbarat türü, birbirini besleyen ve destekleyen bir hiyerarşi içinde çalışır. Üst yönetim, genel güvenlik stratejisini belirlemek için stratejik istihbarata ihtiyaç duyar. Bu strateji, operasyonel ve taktiksel gereksinimlere dönüşür. Örneğin, stratejik istihbarat, jeopolitik gerilimlerin bir APT grubunun faaliyetlerini artıracağını gösterdiğinde, operasyonel ekip bu grubun TTP'lerini incelemeye odaklanır ve bu operasyonel bilgiler taktiksel IoC'lere dönüştürülerek güvenlik kontrollerine entegre edilir. Bu entegrasyon, istihbaratın tüm organizasyonel katmanlarda değer yaratmasını sağlar.

\textbf{İstihbarat Türleri Karşılaştırma Tablosu}

\begin{longtable}{|p{2cm}|p{3.5cm}|p{3.5cm}|p{3cm}|}
\hline
\textbf{Özellik} & \textbf{Stratejik} & \textbf{Operasyonel} & \textbf{Taktik} \\
\hline
\textbf{Hedef Kitle} & C-seviyesi yöneticiler & IR ekipleri, tehdit avcıları & SOC analistleri \\
\hline
\textbf{Odak Noktası} & Genel eğilimler, risk yönetimi & Saldırı kampanyaları, TTP'ler & IOC'ler, ağ etkinlikleri \\
\hline
\textbf{Zaman} & Uzun vadeli (aylar/yıllar) & Orta vadeli (hafta/ay) & Kısa vadeli (gerçek zamanlı) \\
\hline
\textbf{Amaç} & Politika ve yatırım belirleme & Müdahale ve proaktif avcılık & Otomatik tespit ve engelleme \\
\hline
\end{longtable}

\subsection{Threat Actor Profiling ve Attribution Challenges}

Tehdit aktörü profilleme, saldırıların arkasındaki "kim" sorusuna cevap aramayı amaçlar. Bir tehdit aktörü profili, sadece bir isimden ibaret değildir; bir saldırganın kimliğini, hedeflerini, TTP'lerini, motivasyonlarını, coğrafi konumunu ve kullandığı altyapıyı kapsamlı bir şekilde analiz eder. Bu profiller, güvenlik ekiplerine saldırganın olası davranışları hakkında bir resim sunar ve savunma stratejilerini saldırganın niyetleriyle uyumlu hale getirmeye yardımcı olur.

Ancak, bir siber saldırıyı kesin olarak belirli bir tehdit aktörüne atfetmek (attribution), son derece karmaşık ve zorlu bir süreçtir. Bu zorlukların birkaç temel nedeni vardır:

\begin{itemize}
    \item \textbf{Gizleme ve Obfüskasyon:} Saldırganlar, kimliklerini gizlemek için çok sayıda katman kullanır. Botnetler, vekil sunucular ve kiralanmış altyapı, saldırının gerçek kaynağını maskeler. Özellikle dağıtık servis dışı bırakma (DDoS) saldırılarında, trafik binlerce farklı cihazdan gelebilir, bu da kesin bir atıf yapmayı neredeyse imkansız hale getirir.
    \item \textbf{Yanlış Bayrak Operasyonları:} Bazı siber suç grupları, dikkat çekmek veya yanlış bilgi yaymak amacıyla gerçekleştirmediği saldırıların sorumluluğunu üstlenebilir. Bu tür eylemler, atıf sürecini manipüle etmeyi amaçlar.
    \item \textbf{Araç ve TTP Paylaşımı:} Tehdit grupları, sıklıkla araç setlerini ve TTP'lerini birbirleriyle paylaşır veya değiştirir. Bir saldırıda kullanılan belirli bir teknik, daha önce bilinen bir grubun imzası olsa bile, başka bir grup tarafından da kullanılmış olabilir. Bu durum, yalnızca teknik IOC'lere dayalı atıf yapmanın güvenilirliğini azaltır.
\end{itemize}

Bu nedenle, atıf birincil hedef olmamalıdır. Bir saldırı meydana geldiğinde, ilk ve en acil öncelik, hasarı durdurmak, sistemleri güvence altına almak ve devam eden tehditleri ortadan kaldırmaktır. Atıf, bu acil riskler ortadan kaldırıldıktan sonra, olay sonrası analiz aşamasında daha sonraki savunmaları güçlendirmek için bir araç olarak ele alınmalıdır. Atıf, bir "evet/hayır" cevabı yerine, farklı güven seviyelerine sahip (düşük, orta, yüksek) analitik bir değerlendirme süreci olarak görülmelidir. Güvenilir bir atıf için, sadece IOC'ler yerine, davranışsal kanıtlar ve TTP'lere odaklanmak daha geçerli bir yaklaşımdır.

\subsection{Diamond Model ve Kill Chain Analysis}

Siber saldırıları anlamak ve analiz etmek için kullanılan iki önemli analitik çerçeve, Lockheed Martin'in Siber Kill Chain’i ve MITRE’nin Saldırı Analizinin Elmas Modeli’dir (Diamond Model of Intrusion Analysis). Bu modeller, saldırı sürecini farklı açılardan ele alarak birbirini tamamlar.

\begin{itemize}
    \item \textbf{Siber Kill Chain (Siber Saldırı Zinciri):} Bu model, bir saldırının yedi aşamalı, doğrusal bir sürecini sunar. Her aşama, bir saldırganın başarılı bir siber saldırı gerçekleştirmek için tipik olarak izlediği adımları tanımlar.
    \begin{enumerate}
        \item \textbf{Keşif (Reconnaissance):} Saldırgan, hedefin sistemleri, ağ yapısı ve çalışanları hakkında bilgi toplar.
        \item \textbf{Silahlandırma (Weaponization):} Bir exploit ve bir arka kapı (payload) bir araya getirilerek tek bir saldırı paketi oluşturulur.
        \item \textbf{Teslimat (Delivery):} Saldırı paketi, hedef sisteme iletilir (örneğin, oltalama e-postası veya kötü amaçlı web sitesi aracılığıyla).
        \item \textbf{İstismar (Exploitation):} Bir sistemdeki zafiyet kullanılarak ilk erişim elde edilir.
        \item \textbf{Kurulum (Installation):} Saldırgan, sistemde kalıcılığı sağlamak için arka kapıyı kurar.
        \item \textbf{Komuta ve Kontrol (Command and Control):} Saldırganın ele geçirilmiş sistemle uzaktan iletişim kurarak onu kontrol etmesini sağlar.
        \item \textbf{Hedefler Üzerindeki Eylemler (Actions on Objectives):} Saldırgan, veri sızdırma, sistem bozulması veya fidye gibi nihai hedeflerine ulaşır.
    \end{enumerate}
    Kill Chain, saldırıların nasıl ilerlediğine dair adım adım bir yol haritası sunar ve özellikle olay müdahalesi ve taktiksel savunma kararları için çok değerlidir.
    \item \textbf{Diamond Model of Intrusion Analysis (Saldırı Analizinin Elmas Modeli):} Bu model, siber saldırıları dört temel bileşen arasındaki ilişkilere odaklanarak inceler. Doğrusal bir model değildir, daha çok saldırının bütüncül bir resmini sunar.
    \begin{enumerate}
        \item \textbf{Saldırgan (Adversary):} Saldırının arkasındaki tehdit aktörü.
        \item \textbf{Kabiliyet (Capability):} Saldırganın kullandığı araçlar, teknikler ve yöntemler.
        \item \textbf{Altyapı (Infrastructure):} Saldırıyı desteklemek için kullanılan ağ altyapısı (örneğin, C2 sunucuları, vekil sunucular).
        \item \textbf{Kurban (Victim):} Saldırıya uğrayan kişi, organizasyon veya sistem.
    \end{enumerate}
    Elmas Modeli, tehdit istihbaratı ve proaktif tehdit avcılığı için daha kullanışlıdır. Saldırıdan elde edilen herhangi bir bilgi parçası (örneğin, bir C2 sunucusu), diğer üç bileşenle ilişkilendirilerek saldırganın potansiyel diğer faaliyetlerini ve gelecekteki olası hedeflerini tahmin etmek için kullanılabilir.
\end{itemize}

Kill Chain, bir saldırının \textit{nasıl} ilerlediğini detaylandırırken, Diamond Model saldırının \textit{neden} ve \textit{kim tarafından} yapıldığını daha geniş bir bakış açısı sunar. Bu modeller birbirini tamamlar. Bir saldırı tespit edildiğinde, Kill Chain modeli olayın taktiksel olarak yönetilmesine yardımcı olurken, Diamond Model olayın bileşenlerini analiz ederek bu olayı bilinen diğer tehdit gruplarıyla ilişkilendirmeyi ve gelecekteki potansiyel saldırıları tahmin etmeyi mümkün kılar.

% Kill Chain ve Diamond Model Karşılaştırma Tablosu
\begin{longtable}{|p{3cm}|p{5.5cm}|p{5.5cm}|}
\hline
\textbf{Özellik} & \textbf{Siber Kill Chain} & \textbf{Diamond Model} \\
\hline
\textbf{Odak} & Saldırı aşamaları ve süreci & Aktör-kabiliyet-altyapı-kurban ilişkileri \\
\hline
\textbf{Granülarite} & Spesifik ve doğrusal & Geniş, bağlamsal ve ilişkisel \\
\hline
\textbf{Uygulama} & Olay müdahalesi, taktiksel savunma & Tehdit istihbaratı, proaktif avcılık \\
\hline
\textbf{Avantajlar} & Adım adım müdahale yol haritası & Motivasyon ve kabiliyet analizi \\
\hline
\end{longtable}

\subsection{Threat Intelligence Lifecycle ve Collection Methods}

Tehdit istihbaratı yaşam döngüsü, ham veriyi değerli istihbarata dönüştüren ve sürekli bir geri bildirim döngüsü ile kendini yenileyen, yapılandırılmış bir süreçtir. Bu döngü, CTI programının temelini oluşturur ve reaktif bir modelden proaktif bir savunma yaklaşımına geçişin temelini atar.

\textbf{Yaşam Döngüsü Aşamaları:}

\begin{enumerate}
    \item \textbf{Gereksinimler (Requirements):} Döngünün ilk ve en kritik aşamasıdır. Bu aşamada, güvenlik ekipleri, iş birimleri ve üst düzey yöneticiler bir araya gelerek istihbarat ihtiyaçlarını net bir şekilde tanımlar. Bu, korunması gereken en kritik varlıkları (“Crown Jewels”), organizasyonun karşılaştığı riskleri ve bu riskleri azaltmak için hangi bilgilerin gerekli olduğunu belirlemeyi içerir. Gereksinimlerin net bir şekilde belirlenmesi, istihbarat toplama çabalarının boşa gitmesini engeller ve kaynakların doğru hedeflere yönlendirilmesini sağlar. Aksi halde, ekipler alakasız verileri takip ederek zaman ve kaynak kaybedebilir veya kritik tehditleri gözden kaçırabilir.
    \item \textbf{Toplama (Collection):} Bu aşama, tanımlanan gereksinimleri karşılamak için geniş bir yelpazedeki kaynaklardan ham veri toplamayı içerir. Toplanan veriler hem teknik (Indicator of Compromise - IOC) hem de bağlamsal bilgiler (TTP’ler, motivasyonlar) içermelidir.
    \begin{itemize}
        \item \textbf{OSINT (Open-Source Intelligence):} Genel kullanıma açık ve serbestçe erişilebilen kaynaklardan bilgi toplama. Bu kaynaklar arasında haberler, bloglar, sosyal medya platformları, akademik çalışmalar ve endüstri raporları bulunur.
        \item \textbf{Ticari Tehdit Beslemeleri:} Güvenlik firmaları tarafından sağlanan ve genellikle otomatik sistemlere entegre edilen ücretli veri akışlarıdır. Bu beslemeler, binlerce IOC ve TTP bilgisini yüksek hacimde sunabilir.
        \item \textbf{Bilgi Paylaşım Toplulukları (ISACs):} Aynı sektördeki veya coğrafi bölgedeki organizasyonların tehdit istihbaratı paylaşımı için bir araya geldiği güvenilir platformlardır.
        \item \textbf{Dahili Kaynaklar:} Organizasyonun kendi güvenlik araçlarından (SIEM, EDR, IDS/IPS), ağ günlüklerinden, uç nokta telemetri verilerinden ve kimlik doğrulama kayıtlarından elde edilen verilerdir.
        \item \textbf{Derin ve Karanlık Web İzleme:} Gizli veya şifreli forumlar, siber suç pazarları ve sızdırılmış veri depolarından (örneğin, sızan kimlik bilgileri, saldırı planları) bilgi toplama.
    \end{itemize}
    \item \textbf{İşleme (Processing):} Toplanan ham verinin analiz edilebilir, yapılandırılmış ve temiz bir formata dönüştürüldüğü aşamadır. Bu, veri normalizasyonu (farklı formatlardaki verileri standartlaştırma), tekilleştirme (yinelenen kayıtları kaldırma), şifre çözme ve etiketleme işlemlerini içerir. Büyük veri kümeleri için otomasyon, bu aşamada hayati öneme sahiptir.
    \item \textbf{Analiz ve Yorumlama (Analysis and Interpretation):} İşlenmiş verinin anlamlı, eyleme dönüştürülebilir istihbarata dönüştüğü aşamadır. Analistler, kalıpları, eğilimleri, saldırı kampanyalarını ve potansiyel tehditleri belirlemek için verileri derinlemesine incelerler. Bu aşamada, insan uzmanlığı ve otomatik korelasyon mekanizmaları birleşir.
    \item \textbf{Yayma (Dissemination):} Analiz edilen istihbaratın ilgili paydaşlara zamanında ve uygun bir formatta sunulmasıdır. İstihbaratın formatı ve içeriği, hedef kitleye göre uyarlanmalıdır. Örneğin, CISO'lar için üst düzey yönetici brifingleri hazırlanırken, SOC analistleri için teknik IOC beslemleri sunulur.
    \item \textbf{Geri Bildirim (Feedback):} Sürecin sürekli iyileştirilmesini sağlayan bu aşamada, istihbarat tüketicileri (örneğin, SOC analistleri), sunulan bilgilerin yararlılığı, alaka düzeyi ve doğruluğu hakkında geri bildirimde bulunur. Bu geri bildirim, bir sonraki döngüde daha iyi gereksinimlerin belirlenmesini ve istihbarat toplama stratejilerinin hassaslaştırılmasını sağlar.
\end{enumerate}

Bu yaşam döngüsü, bir organizasyonun reaktif tehdit algılama modelinden proaktif bir savunma duruşuna geçiş yapmasına olanak tanır. Bu sürekli ve döngüsel süreç, savunma mekanizmalarının saldırganların evrilen taktik, teknik ve prosedürlerine (TTP'ler) göre dinamik olarak ayarlanmasını sağlar. Bu dinamik adaptasyon, bir saldırganın bir sistemde kalma süresini (dwell time) doğrudan azaltır. Saldırganın sistemde kalma süresinin azalması, fidye yazılımı veya veri sızdırma gibi nihai hedeflere ulaşma şansını düşürürken, ihlalden kaynaklanan maliyetleri de önemli ölçüde azaltır. Bu sürekli öğrenme ve adaptasyon mekanizması, tehdit istihbaratı programının değerini ve etkinliğini katlanarak artırır.

\textbf{CTI Yaşam Döngüsü Aşamaları ve Uygulamaları}

\begin{longtable}{|p{2.5cm}|p{3cm}|p{3cm}|p{3.5cm}|}
\hline
\textbf{Aşama} & \textbf{Kısa Tanım} & \textbf{Hedef} & \textbf{Pratik Uygulama} \\
\hline
\textbf{Gereksinimler} & İhtiyaç duyulan istihbaratın belirlenmesi & Kaynakları kritik risklere odaklamak & Tehdit aktörü TTP analizi \\
\hline
\textbf{Toplama} & Ham verinin kaynaklardan alınması & Gereksinim odaklı veri toplama & SIEM günlükleri, OSINT, karanlık web \\
\hline
\textbf{İşleme} & Ham veriyi analiz formatına getirme & Veri yapılandırma ve zenginleştirme & CSV normalizasyonu, VirusTotal entegrasyonu \\
\hline
\textbf{Analiz} & Veriden içgörüler çıkarma & Tehdit bağlamsallaştırma & Oltalama kampanya teknik analizi \\
\hline
\textbf{Yayma} & İstihbaratı paydaşlara iletme & Doğru kitleye doğru formatta sunum & Yönetici brifingi, SOC beslemesi \\
\hline
\textbf{Geri Bildirim} & Etkinlik geri bildirimi toplama & Sürekli iyileştirme & Yanlış pozitif raporları, değerlendirmeler \\
\hline
\end{longtable}

\subsection{Strategic, Tactical, Technical ve Operational Intelligence}

Siber tehdit istihbaratı, hedef kitlenin ihtiyaçlarına göre dört ana kategoriye ayrılır. Bu istihbarat türleri, bir organizasyonun savunma yeteneklerini tüm seviyelerde güçlendirmek için birlikte çalışır.

\begin{itemize}
    \item \textbf{Stratejik İstihbarat:} Genel tehdit ortamına ilişkin üst düzey bir bakış açısı sunar. Bu istihbarat, teknoloji dışı terimlerle hazırlanır ve öncelikli olarak üst düzey yöneticilere, CISO'lara ve risk yöneticilerine yöneliktir.
    \begin{itemize}
        \item \textbf{Amacı:} Güvenlik yatırımları, bütçe tahsisi ve kurumsal politikalar gibi uzun vadeli stratejik kararlara rehberlik etmektir.
        \item \textbf{Örnekler:} Bir APT grubunun bir sektörü veya belirli bir coğrafyayı hedeflemesi, fidye yazılımı eğilimleri veya jeopolitik olayların siber saldırı risklerine etkisi hakkında raporlar.
    \end{itemize}
    \item \textbf{Operasyonel İstihbarat:} Belirli bir tehdit aktörünün TTP'lerini, motivasyonlarını ve altyapısını anlamayı sağlar. Bu istihbarat, olay müdahale (IR) ekipleri ve tehdit avcıları için hayati öneme sahiptir.
    \begin{itemize}
        \item \textbf{Amacı:} Yaklaşan veya devam eden bir saldırının "kim, ne zaman, nerede, nasıl, neden" sorularına bağlamsal cevaplar sunmak. Bu, proaktif tehdit avcılığı ve olay müdahale planlaması için bir temel oluşturur.
        \item \textbf{Örnekler:} Bir saldırıda kullanılan belirli bir oltalama e-postası kampanyasının detayları veya bir Komuta ve Kontrol (C2) sunucusunun iletişim yöntemleri.
    \end{itemize}
    \item \textbf{Taktik İstihbarat:} Ağ ve uç noktalarda tehditleri tespit etmeye yardımcı olan, genellikle kısa ömürlü ve teknik göstergelerden oluşan bir istihbarat türüdür.
    \begin{itemize}
        \item \textbf{Amacı:} Otomatik tehdit tespiti için SIEM ve güvenlik duvarı gibi güvenlik kontrollerine entegre edilmek.
        \item \textbf{Örnekler:} Kötü amaçlı IP adresleri, dosya karmaları (hashes), kötü amaçlı alan adları.
    \end{itemize}
    \item \textbf{Teknik İstihbarat:} Taktik istihbaratın daha derin teknik detaylarını içerir ve genellikle kötü amaçlı yazılım analizi, tersine mühendislik ve IoC'lerin oluşturulmasını kapsar.
    \begin{itemize}
        \item \textbf{Amacı:} Saldırıların teknik işleyişini anlamak ve bu bilgilere dayanarak yeni imza tabanlı tespit kuralları oluşturmaktır.
    \end{itemize}
\end{itemize}

Bu dört istihbarat türü, birbirini besleyen ve destekleyen bir hiyerarşi içinde çalışır. Üst yönetim, genel güvenlik stratejisini belirlemek için stratejik istihbarata ihtiyaç duyar. Bu strateji, operasyonel ve taktiksel gereksinimlere dönüşür. Örneğin, stratejik istihbarat, jeopolitik gerilimlerin bir APT grubunun faaliyetlerini artıracağını gösterdiğinde, operasyonel ekip bu grubun TTP'lerini incelemeye odaklanır ve bu operasyonel bilgiler taktiksel IoC'lere dönüştürülerek güvenlik kontrollerine entegre edilir. Bu entegrasyon, istihbaratın tüm organizasyonel katmanlarda değer yaratmasını sağlar.

\textbf{İstihbarat Türleri Karşılaştırma Tablosu}

\begin{longtable}{|p{2cm}|p{3.5cm}|p{3.5cm}|p{3cm}|}
\hline
\textbf{Özellik} & \textbf{Stratejik} & \textbf{Operasyonel} & \textbf{Taktik} \\
\hline
\textbf{Hedef Kitle} & C-seviyesi yöneticiler & IR ekipleri, tehdit avcıları & SOC analistleri \\
\hline
\textbf{Odak Noktası} & Genel eğilimler, risk yönetimi & Saldırı kampanyaları, TTP'ler & IOC'ler, ağ etkinlikleri \\
\hline
\textbf{Zaman} & Uzun vadeli (aylar/yıllar) & Orta vadeli (hafta/ay) & Kısa vadeli (gerçek zamanlı) \\
\hline
\textbf{Amaç} & Politika ve yatırım belirleme & Müdahale ve proaktif avcılık & Otomatik tespit ve engelleme \\
\hline
\end{longtable}

\subsection{Threat Actor Profiling ve Attribution Challenges}

Tehdit aktörü profilleme, saldırıların arkasındaki "kim" sorusuna cevap aramayı amaçlar. Bir tehdit aktörü profili, sadece bir isimden ibaret değildir; bir saldırganın kimliğini, hedeflerini, TTP'lerini, motivasyonlarını, coğrafi konumunu ve kullandığı altyapıyı kapsamlı bir şekilde analiz eder. Bu profiller, güvenlik ekiplerine saldırganın olası davranışları hakkında bir resim sunar ve savunma stratejilerini saldırganın niyetleriyle uyumlu hale getirmeye yardımcı olur.

Ancak, bir siber saldırıyı kesin olarak belirli bir tehdit aktörüne atfetmek (attribution), son derece karmaşık ve zorlu bir süreçtir. Bu zorlukların birkaç temel nedeni vardır:

\begin{itemize}
    \item \textbf{Gizleme ve Obfüskasyon:} Saldırganlar, kimliklerini gizlemek için çok sayıda katman kullanır. Botnetler, vekil sunucular ve kiralanmış altyapı, saldırının gerçek kaynağını maskeler. Özellikle dağıtık servis dışı bırakma (DDoS) saldırılarında, trafik binlerce farklı cihazdan gelebilir, bu da kesin bir atıf yapmayı neredeyse imkansız hale getirir.
    \item \textbf{Yanlış Bayrak Operasyonları:} Bazı siber suç grupları, dikkat çekmek veya yanlış bilgi yaymak amacıyla gerçekleştirmediği saldırıların sorumluluğunu üstlenebilir. Bu tür eylemler, atıf sürecini manipüle etmeyi amaçlar.
    \item \textbf{Araç ve TTP Paylaşımı:} Tehdit grupları, sıklıkla araç setlerini ve TTP'lerini birbirleriyle paylaşır veya değiştirir. Bir saldırıda kullanılan belirli bir teknik, daha önce bilinen bir grubun imzası olsa bile, başka bir grup tarafından da kullanılmış olabilir. Bu durum, yalnızca teknik IOC'lere dayalı atıf yapmanın güvenilirliğini azaltır.
\end{itemize}

Bu nedenle, atıf birincil hedef olmamalıdır. Bir saldırı meydana geldiğinde, ilk ve en acil öncelik, hasarı durdurmak, sistemleri güvence altına almak ve devam eden tehditleri ortadan kaldırmaktır. Atıf, bu acil riskler ortadan kaldırıldıktan sonra, olay sonrası analiz aşamasında daha sonraki savunmaları güçlendirmek için bir araç olarak ele alınmalıdır. Atıf, bir "evet/hayır" cevabı yerine, farklı güven seviyelerine sahip (düşük, orta, yüksek) analitik bir değerlendirme süreci olarak görülmelidir. Güvenilir bir atıf için, sadece IOC'ler yerine, davranışsal kanıtlar ve TTP'lere odaklanmak daha geçerli bir yaklaşımdır.

\subsection{Diamond Model ve Kill Chain Analysis}

Siber saldırıları anlamak ve analiz etmek için kullanılan iki önemli analitik çerçeve, Lockheed Martin'in Siber Kill Chain’i ve MITRE’nin Saldırı Analizinin Elmas Modeli’dir (Diamond Model of Intrusion Analysis). Bu modeller, saldırı sürecini farklı açılardan ele alarak birbirini tamamlar.

\begin{itemize}
    \item \textbf{Siber Kill Chain (Siber Saldırı Zinciri):} Bu model, bir saldırının yedi aşamalı, doğrusal bir sürecini sunar. Her aşama, bir saldırganın başarılı bir siber saldırı gerçekleştirmek için tipik olarak izlediği adımları tanımlar.
    \begin{enumerate}
        \item \textbf{Keşif (Reconnaissance):} Saldırgan, hedefin sistemleri, ağ yapısı ve çalışanları hakkında bilgi toplar.
        \item \textbf{Silahlandırma (Weaponization):} Bir exploit ve bir arka kapı (payload) bir araya getirilerek tek bir saldırı paketi oluşturulur.
        \item \textbf{Teslimat (Delivery):} Saldırı paketi, hedef sisteme iletilir (örneğin, oltalama e-postası veya kötü amaçlı web sitesi aracılığıyla).
        \item \textbf{İstismar (Exploitation):} Bir sistemdeki zafiyet kullanılarak ilk erişim elde edilir.
        \item \textbf{Kurulum (Installation):} Saldırgan, sistemde kalıcılığı sağlamak için arka kapıyı kurar.
        \item \textbf{Komuta ve Kontrol (Command and Control):} Saldırganın ele geçirilmiş sistemle uzaktan iletişim kurarak onu kontrol etmesini sağlar.
        \item \textbf{Hedefler Üzerindeki Eylemler (Actions on Objectives):} Saldırgan, veri sızdırma, sistem bozulması veya fidye gibi nihai hedeflerine ulaşır.
    \end{enumerate}
    Kill Chain, saldırıların nasıl ilerlediğine dair adım adım bir yol haritası sunar ve özellikle olay müdahalesi ve taktiksel savunma kararları için çok değerlidir.
    \item \textbf{Diamond Model of Intrusion Analysis (Saldırı Analizinin Elmas Modeli):} Bu model, siber saldırıları dört temel bileşen arasındaki ilişkilere odaklanarak inceler. Doğrusal bir model değildir, daha çok saldırının bütüncül bir resmini sunar.
    \begin{enumerate}
        \item \textbf{Saldırgan (Adversary):} Saldırının arkasındaki tehdit aktörü.
        \item \textbf{Kabiliyet (Capability):} Saldırganın kullandığı araçlar, teknikler ve yöntemler.
        \item \textbf{Altyapı (Infrastructure):} Saldırıyı desteklemek için kullanılan ağ altyapısı (örneğin, C2 sunucuları, vekil sunucular).
        \item \textbf{Kurban (Victim):} Saldırıya uğrayan kişi, organizasyon veya sistem.
    \end{enumerate}
    Elmas Modeli, tehdit istihbaratı ve proaktif tehdit avcılığı için daha kullanışlıdır. Saldırıdan elde edilen herhangi bir bilgi parçası (örneğin, bir C2 sunucusu), diğer üç bileşenle ilişkilendirilerek saldırganın potansiyel diğer faaliyetlerini ve gelecekteki olası hedeflerini tahmin etmek için kullanılabilir.
\end{itemize}

Kill Chain, bir saldırının \textit{nasıl} ilerlediğini detaylandırırken, Diamond Model saldırının \textit{neden} ve \textit{kim tarafından} yapıldığını daha geniş bir bakış açısı sunar. Bu modeller birbirini tamamlar. Bir saldırı tespit edildiğinde, Kill Chain modeli olayın taktiksel olarak yönetilmesine yardımcı olurken, Diamond Model olayın bileşenlerini analiz ederek bu olayı bilinen diğer tehdit gruplarıyla ilişkilendirmeyi ve gelecekteki potansiyel saldırıları tahmin etmeyi mümkün kılar.

% Kill Chain ve Diamond Model Karşılaştırma Tablosu
\begin{longtable}{|p{3cm}|p{5.5cm}|p{5.5cm}|}
\hline
\textbf{Özellik} & \textbf{Siber Kill Chain} & \textbf{Diamond Model} \\
\hline
\textbf{Odak} & Saldırı aşamaları ve süreci & Aktör-kabiliyet-altyapı-kurban ilişkileri \\
\hline
\textbf{Granülarite} & Spesifik ve doğrusal & Geniş, bağlamsal ve ilişkisel \\
\hline
\textbf{Uygulama} & Olay müdahalesi, taktiksel savunma & Tehdit istihbaratı, proaktif avcılık \\
\hline
\textbf{Avantajlar} & Adım adım müdahale yol haritası & Motivasyon ve kabiliyet analizi \\
\hline
\end{longtable}

\subsection{Threat Intelligence Lifecycle ve Collection Methods}

Tehdit istihbaratı yaşam döngüsü, ham veriyi değerli istihbarata dönüştüren ve sürekli bir geri bildirim döngüsü ile kendini yenileyen, yapılandırılmış bir süreçtir. Bu döngü, CTI programının temelini oluşturur ve reaktif bir modelden proaktif bir savunma yaklaşımına geçişin temelini atar.

\textbf{Yaşam Döngüsü Aşamaları:}

\begin{enumerate}
    \item \textbf{Gereksinimler (Requirements):} Döngünün ilk ve en kritik aşamasıdır. Bu aşamada, güvenlik ekipleri, iş birimleri ve üst düzey yöneticiler bir araya gelerek istihbarat ihtiyaçlarını net bir şekilde tanımlar. Bu, korunması gereken en kritik varlıkları (“Crown Jewels”), organizasyonun karşılaştığı riskleri ve bu riskleri azaltmak için hangi bilgilerin gerekli olduğunu belirlemeyi içerir. Gereksinimlerin net bir şekilde belirlenmesi, istihbarat toplama çabalarının boşa gitmesini engeller ve kaynakların doğru hedeflere yönlendirilmesini sağlar. Aksi halde, ekipler alakasız verileri takip ederek zaman ve kaynak kaybedebilir veya kritik tehditleri gözden kaçırabilir.
    \item \textbf{Toplama (Collection):} Bu aşama, tanımlanan gereksinimleri karşılamak için geniş bir yelpazedeki kaynaklardan ham veri toplamayı içerir. Toplanan veriler hem teknik (Indicator of Compromise - IOC) hem de bağlamsal bilgiler (TTP’ler, motivasyonlar) içermelidir.
    \begin{itemize}
        \item \textbf{OSINT (Open-Source Intelligence):} Genel kullanıma açık ve serbestçe erişilebilen kaynaklardan bilgi toplama. Bu kaynaklar arasında haberler, bloglar, sosyal medya platformları, akademik çalışmalar ve endüstri raporları bulunur.
        \item \textbf{Ticari Tehdit Beslemeleri:} Güvenlik firmaları tarafından sağlanan ve genellikle otomatik sistemlere entegre edilen ücretli veri akışlarıdır. Bu beslemeler, binlerce IOC ve TTP bilgisini yüksek hacimde sunabilir.
        \item \textbf{Bilgi Paylaşım Toplulukları (ISACs):} Aynı sektördeki veya coğrafi bölgedeki organizasyonların tehdit istihbaratı paylaşımı için bir araya geldiği güvenilir platformlardır.
        \item \textbf{Dahili Kaynaklar:} Organizasyonun kendi güvenlik araçlarından (SIEM, EDR, IDS/IPS), ağ günlüklerinden, uç nokta telemetri verilerinden ve kimlik doğrulama kayıtlarından elde edilen verilerdir.
        \item \textbf{Derin ve Karanlık Web İzleme:} Gizli veya şifreli forumlar, siber suç pazarları ve sızdırılmış veri depolarından (örneğin, sızan kimlik bilgileri, saldırı planları) bilgi toplama.
    \end{itemize}
    \item \textbf{İşleme (Processing):} Toplanan ham verinin analiz edilebilir, yapılandırılmış ve temiz bir formata dönüştürüldüğü aşamadır. Bu, veri normalizasyonu (farklı formatlardaki verileri standartlaştırma), tekilleştirme (yinelenen kayıtları kaldırma), şifre çözme ve etiketleme işlemlerini içerir. Büyük veri kümeleri için otomasyon, bu aşamada hayati öneme sahiptir.
    \item \textbf{Analiz ve Yorumlama (Analysis and Interpretation):} İşlenmiş verinin anlamlı, eyleme dönüştürülebilir istihbarata dönüştüğü aşamadır. Analistler, kalıpları, eğilimleri, saldırı kampanyalarını ve potansiyel tehditleri belirlemek için verileri derinlemesine incelerler. Bu aşamada, insan uzmanlığı ve otomatik korelasyon mekanizmaları birleşir.
    \item \textbf{Yayma (Dissemination):} Analiz edilen istihbaratın ilgili paydaşlara zamanında ve uygun bir formatta sunulmasıdır. İstihbaratın formatı ve içeriği, hedef kitleye göre uyarlanmalıdır. Örneğin, CISO'lar için üst düzey yönetici brifingleri hazırlanırken, SOC analistleri için teknik IOC beslemleri sunulur.
    \item \textbf{Geri Bildirim (Feedback):} Sürecin sürekli iyileştirilmesini sağlayan bu aşamada, istihbarat tüketicileri (örneğin, SOC analistleri), sunulan bilgilerin yararlılığı, alaka düzeyi ve doğruluğu hakkında geri bildirimde bulunur. Bu geri bildirim, bir sonraki döngüde daha iyi gereksinimlerin belirlenmesini ve istihbarat toplama stratejilerinin hassaslaştırılmasını sağlar.
\end{enumerate}

Bu yaşam döngüsü, bir organizasyonun reaktif tehdit algılama modelinden proaktif bir savunma duruşuna geçiş yapmasına olanak tanır. Bu sürekli ve döngüsel süreç, savunma mekanizmalarının saldırganların evrilen taktik, teknik ve prosedürlerine (TTP'ler) göre dinamik olarak ayarlanmasını sağlar. Bu dinamik adaptasyon, bir saldırganın bir sistemde kalma süresini (dwell time) doğrudan azaltır. Saldırganın sistemde kalma süresinin azalması, fidye yazılımı veya veri sızdırma gibi nihai hedeflere ulaşma şansını düşürürken, ihlalden kaynaklanan maliyetleri de önemli ölçüde azaltır. Bu sürekli öğrenme ve adaptasyon mekanizması, tehdit istihbaratı programının değerini ve etkinliğini katlanarak artırır.

\textbf{CTI Yaşam Döngüsü Aşamaları ve Uygulamaları}

\begin{longtable}{|p{2.5cm}|p{3cm}|p{3cm}|p{3.5cm}|}
\hline
\textbf{Aşama} & \textbf{Kısa Tanım} & \textbf{Hedef} & \textbf{Pratik Uygulama} \\
\hline
\textbf{Gereksinimler} & İhtiyaç duyulan istihbaratın belirlenmesi & Kaynakları kritik risklere odaklamak & Tehdit aktörü TTP analizi \\
\hline
\textbf{Toplama} & Ham verinin kaynaklardan alınması & Gereksinim odaklı veri toplama & SIEM günlükleri, OSINT, karanlık web \\
\hline
\textbf{İşleme} & Ham veriyi analiz formatına getirme & Veri yapılandırma ve zenginleştirme & CSV normalizasyonu, VirusTotal entegrasyonu \\
\hline
\textbf{Analiz} & Veriden içgörüler çıkarma & Tehdit bağlamsallaştırma & Oltalama kampanya teknik analizi \\
\hline
\textbf{Yayma} & İstihbaratı paydaşlara iletme & Doğru kitleye doğru formatta sunum & Yönetici brifingi, SOC beslemesi \\
\hline
\textbf{Geri Bildirim} & Etkinlik geri bildirimi toplama & Sürekli iyileştirme & Yanlış pozitif raporları, değerlendirmeler \\
\hline
\end{longtable}

\subsection{Strategic, Tactical, Technical ve Operational Intelligence}

Siber tehdit istihbaratı, hedef kitlenin ihtiyaçlarına göre dört ana kategoriye ayrılır. Bu istihbarat türleri, bir organizasyonun savunma yeteneklerini tüm seviyelerde güçlendirmek için birlikte çalışır.

\begin{itemize}
    \item \textbf{Stratejik İstihbarat:} Genel tehdit ortamına ilişkin üst düzey bir bakış açısı sunar. Bu istihbarat, teknoloji dışı terimlerle hazırlanır ve öncelikli olarak üst düzey yöneticilere, CISO'lara ve risk yöneticilerine yöneliktir.
    \begin{itemize}
        \item \textbf{Amacı:} Güvenlik yatırımları, bütçe tahsisi ve kurumsal politikalar gibi uzun vadeli stratejik kararlara rehberlik etmektir.
        \item \textbf{Örnekler:} Bir APT grubunun bir sektörü veya belirli bir coğrafyayı hedeflemesi, fidye yazılımı eğilimleri veya jeopolitik olayların siber saldırı risklerine etkisi hakkında raporlar.
    \end{itemize}
    \item \textbf{Operasyonel İstihbarat:} Belirli bir tehdit aktörünün TTP'lerini, motivasyonlarını ve altyapısını anlamayı sağlar. Bu istihbarat, olay müdahale (IR) ekipleri ve tehdit avcıları için hayati öneme sahiptir.
    \begin{itemize}
        \item \textbf{Amacı:} Yaklaşan veya devam eden bir saldırının "kim, ne zaman, nerede, nasıl, neden" sorularına bağlamsal cevaplar sunmak. Bu, proaktif tehdit avcılığı ve olay müdahale planlaması için bir temel oluşturur.
        \item \textbf{Örnekler:} Bir saldırıda kullanılan belirli bir oltalama e-postası kampanyasının detayları veya bir Komuta ve Kontrol (C2) sunucusunun iletişim yöntemleri.
    \end{itemize}
    \item \textbf{Taktik İstihbarat:} Ağ ve uç noktalarda tehditleri tespit etmeye yardımcı olan, genellikle kısa ömürlü ve teknik göstergelerden oluşan bir istihbarat türüdür.
    \begin{itemize}
        \item \textbf{Amacı:} Otomatik tehdit tespiti için SIEM ve güvenlik duvarı gibi güvenlik kontrollerine entegre edilmek.
        \item \textbf{Örnekler:} Kötü amaçlı IP adresleri, dosya karmaları (hashes), kötü amaçlı alan adları.
    \end{itemize}
    \item \textbf{Teknik İstihbarat:} Taktik istihbaratın daha derin teknik detaylarını içerir ve genellikle kötü amaçlı yazılım analizi, tersine mühendislik ve IoC'lerin oluşturulmasını kapsar.
    \begin{itemize}
        \item \textbf{Amacı:} Saldırıların teknik işleyişini anlamak ve bu bilgilere dayanarak yeni imza tabanlı tespit kuralları oluşturmaktır.
    \end{itemize}
\end{itemize}

Bu dört istihbarat türü, birbirini besleyen ve destekleyen bir hiyerarşi içinde çalışır. Üst yönetim, genel güvenlik stratejisini belirlemek için stratejik istihbarata ihtiyaç duyar. Bu strateji, operasyonel ve taktiksel gereksinimlere dönüşür. Örneğin, stratejik istihbarat, jeopolitik gerilimlerin bir APT grubunun faaliyetlerini artıracağını gösterdiğinde, operasyonel ekip bu grubun TTP'lerini incelemeye odaklanır ve bu operasyonel bilgiler taktiksel IoC'lere dönüştürülerek güvenlik kontrollerine entegre edilir. Bu entegrasyon, istihbaratın tüm organizasyonel katmanlarda değer yaratmasını sağlar.

\textbf{İstihbarat Türleri Karşılaştırma Tablosu}

\begin{longtable}{|p{2cm}|p{3.5cm}|p{3.5cm}|p{3cm}|}
\hline
\textbf{Özellik} & \textbf{Stratejik} & \textbf{Operasyonel} & \textbf{Taktik} \\
\hline
\textbf{Hedef Kitle} & C-seviyesi yöneticiler & IR ekipleri, tehdit avcıları & SOC analistleri \\
\hline
\textbf{Odak Noktası} & Genel eğilimler, risk yönetimi & Saldırı kampanyaları, TTP'ler & IOC'ler, ağ etkinlikleri \\
\hline
\textbf{Zaman} & Uzun vadeli (aylar/yıllar) & Orta vadeli (hafta/ay) & Kısa vadeli (gerçek zamanlı) \\
\hline
\textbf{Amaç} & Politika ve yatırım belirleme & Müdahale ve proaktif avcılık & Otomatik tespit ve engelleme \\
\hline
\end{longtable}

\subsection{Threat Actor Profiling ve Attribution Challenges}

Tehdit aktörü profilleme, saldırıların arkasındaki "kim" sorusuna cevap aramayı amaçlar. Bir tehdit aktörü profili, sadece bir isimden ibaret değildir; bir saldırganın kimliğini, hedeflerini, TTP'lerini, motivasyonlarını, coğrafi konumunu ve kullandığı altyapıyı kapsamlı bir şekilde analiz eder. Bu profiller, güvenlik ekiplerine saldırganın olası davranışları hakkında bir resim sunar ve savunma stratejilerini saldırganın niyetleriyle uyumlu hale getirmeye yardımcı olur.

Ancak, bir siber saldırıyı kesin olarak belirli bir tehdit aktörüne atfetmek (attribution), son derece karmaşık ve zorlu bir süreçtir. Bu zorlukların birkaç temel nedeni vardır:

\begin{itemize}
    \item \textbf{Gizleme ve Obfüskasyon:} Saldırganlar, kimliklerini gizlemek için çok sayıda katman kullanır. Botnetler, vekil sunucular ve kiralanmış altyapı, saldırının gerçek kaynağını maskeler. Özellikle dağıtık servis dışı bırakma (DDoS) saldırılarında, trafik binlerce farklı cihazdan gelebilir, bu da kesin bir atıf yapmayı neredeyse imkansız hale getirir.
    \item \textbf{Yanlış Bayrak Operasyonları:} Bazı siber suç grupları, dikkat çekmek veya yanlış bilgi yaymak amacıyla gerçekleştirmediği saldırıların sorumluluğunu üstlenebilir. Bu tür eylemler, atıf sürecini manipüle etmeyi amaçlar.
    \item \textbf{Araç ve TTP Paylaşımı:} Tehdit grupları, sıklıkla araç setlerini ve TTP'lerini birbirleriyle paylaşır veya değiştirir. Bir saldırıda kullanılan belirli bir teknik, daha önce bilinen bir grubun imzası olsa bile, başka bir grup tarafından da kullanılmış olabilir. Bu durum, yalnızca teknik IOC'lere dayalı atıf yapmanın güvenilirliğini azaltır.
\end{itemize}

Bu nedenle, atıf birincil hedef olmamalıdır. Bir saldırı meydana geldiğinde, ilk ve en acil öncelik, hasarı durdurmak, sistemleri güvence altına almak ve devam eden tehditleri ortadan kaldırmaktır. Atıf, bu acil riskler ortadan kaldırıldıktan sonra, olay sonrası analiz aşamasında daha sonraki savunmaları güçlendirmek için bir araç olarak ele alınmalıdır. Atıf, bir "evet/hayır" cevabı yerine, farklı güven seviyelerine sahip (düşük, orta, yüksek) analitik bir değerlendirme süreci olarak görülmelidir. Güvenilir bir atıf için, sadece IOC'ler yerine, davranışsal kanıtlar ve TTP'lere odaklanmak daha geçerli bir yaklaşımdır.

\subsection{Diamond Model ve Kill Chain Analysis}

Siber saldırıları anlamak ve analiz etmek için kullanılan iki önemli analitik çerçeve, Lockheed Martin'in Siber Kill Chain’i ve MITRE’nin Saldırı Analizinin Elmas Modeli’dir (Diamond Model of Intrusion Analysis). Bu modeller, saldırı sürecini farklı açılardan ele alarak birbirini tamamlar.

\begin{itemize}
    \item \textbf{Siber Kill Chain (Siber Saldırı Zinciri):} Bu model, bir saldırının yedi aşamalı, doğrusal bir sürecini sunar. Her aşama, bir saldırganın başarılı bir siber saldırı gerçekleştirmek için tipik olarak izlediği adımları tanımlar.
    \begin{enumerate}
        \item \textbf{Keşif (Reconnaissance):} Saldırgan, hedefin sistemleri, ağ yapısı ve çalışanları hakkında bilgi toplar.
        \item \textbf{Silahlandırma (Weaponization):} Bir exploit ve bir arka kapı (payload) bir araya getirilerek tek bir saldırı paketi oluşturulur.
        \item \textbf{Teslimat (Delivery):} Saldırı paketi, hedef sisteme iletilir (örneğin, oltalama e-postası veya kötü amaçlı web sitesi aracılığıyla).
        \item \textbf{İstismar (Exploitation):} Bir sistemdeki zafiyet kullanılarak ilk erişim elde edilir.
        \item \textbf{Kurulum (Installation):} Saldırgan, sistemde kalıcılığı sağlamak için arka kapıyı kurar.
        \item \textbf{Komuta ve Kontrol (Command and Control):} Saldırganın ele geçirilmiş sistemle uzaktan iletişim kurarak onu kontrol etmesini sağlar.
        \item \textbf{Hedefler Üzerindeki Eylemler (Actions on Objectives):} Saldırgan, veri sızdırma, sistem bozulması veya fidye gibi nihai hedeflerine ulaşır.
    \end{enumerate}
    Kill Chain, saldırıların nasıl ilerlediğine dair adım adım bir yol haritası sunar ve özellikle olay müdahalesi ve taktiksel savunma kararları için çok değerlidir.
    \item \textbf{Diamond Model of Intrusion Analysis (Saldırı Analizinin Elmas Modeli):} Bu model, siber saldırıları dört temel bileşen arasındaki ilişkilere odaklanarak inceler. Doğrusal bir model değildir, daha çok saldırının bütüncül bir resmini sunar.
    \begin{enumerate}
        \item \textbf{Saldırgan (Adversary):} Saldırının arkasındaki tehdit aktörü.
        \item \textbf{Kabiliyet (Capability):} Saldırganın kullandığı araçlar, teknikler ve yöntemler.
        \item \textbf{Altyapı (Infrastructure):} Saldırıyı desteklemek için kullanılan ağ altyapısı (örneğin, C2 sunucuları, vekil sunucular).
        \item \textbf{Kurban (Victim):} Saldırıya uğrayan kişi, organizasyon veya sistem.
    \end{enumerate}
    Elmas Modeli, tehdit istihbaratı ve proaktif tehdit avcılığı için daha kullanışlıdır. Saldırıdan elde edilen herhangi bir bilgi parçası (örneğin, bir C2 sunucusu), diğer üç bileşenle ilişkilendirilerek saldırganın potansiyel diğer faaliyetlerini ve gelecekteki olası hedeflerini tahmin etmek için kullanılabilir.
\end{itemize}

Kill Chain, bir saldırının \textit{nasıl} ilerlediğini detaylandırırken, Diamond Model saldırının \textit{neden} ve \textit{kim tarafından} yapıldığını daha geniş bir bakış açısı sunar. Bu modeller birbirini tamamlar. Bir saldırı tespit edildiğinde, Kill Chain modeli olayın taktiksel olarak yönetilmesine yardımcı olurken, Diamond Model olayın bileşenlerini analiz ederek bu olayı bilinen diğer tehdit gruplarıyla ilişkilendirmeyi ve gelecekteki potansiyel saldırıları tahmin etmeyi mümkün kılar.

% Kill Chain ve Diamond Model Karşılaştırma Tablosu
\begin{longtable}{|p{3cm}|p{5.5cm}|p{5.5cm}|}
\hline
\textbf{Özellik} & \textbf{Siber Kill Chain} & \textbf{Diamond Model} \\
\hline
\textbf{Odak} & Saldırı aşamaları ve süreci & Aktör-kabiliyet-altyapı-kurban ilişkileri \\
\hline
\textbf{Granülarite} & Spesifik ve doğrusal & Geniş, bağlamsal ve ilişkisel \\
\hline
\textbf{Uygulama} & Olay müdahalesi, taktiksel savunma & Tehdit istihbaratı, proaktif avcılık \\
\hline
\textbf{Avantajlar} & Adım adım müdahale yol haritası & Motivasyon ve kabiliyet analizi \\
\hline
\end{longtable}

\subsection{Threat Intelligence Lifecycle ve Collection Methods}

Tehdit istihbaratı yaşam döngüsü, ham veriyi değerli istihbarata dönüştüren ve sürekli bir geri bildirim döngüsü ile kendini yenileyen, yapılandırılmış bir süreçtir. Bu döngü, CTI programının temelini oluşturur ve reaktif bir modelden proaktif bir savunma yaklaşımına geçişin temelini atar.

\textbf{Yaşam Döngüsü Aşamaları:}

\begin{enumerate}
    \item \textbf{Gereksinimler (Requirements):} Döngünün ilk ve en kritik aşamasıdır. Bu aşamada, güvenlik ekipleri, iş birimleri ve üst düzey yöneticiler bir araya gelerek istihbarat ihtiyaçlarını net bir şekilde tanımlar. Bu, korunması gereken en kritik varlıkları (“Crown Jewels”), organizasyonun karşılaştığı riskleri ve bu riskleri azaltmak için hangi bilgilerin gerekli olduğunu belirlemeyi içerir. Gereksinimlerin net bir şekilde belirlenmesi, istihbarat toplama çabalarının boşa gitmesini engeller ve kaynakların doğru hedeflere yönlendirilmesini sağlar. Aksi halde, ekipler alakasız verileri takip ederek zaman ve kaynak kaybedebilir veya kritik tehditleri gözden kaçırabilir.
    \item \textbf{Toplama (Collection):} Bu aşama, tanımlanan gereksinimleri karşılamak için geniş bir yelpazedeki kaynaklardan ham veri toplamayı içerir. Toplanan veriler hem teknik (Indicator of Compromise - IOC) hem de bağlamsal bilgiler (TTP’ler, motivasyonlar) içermelidir.
    \begin{itemize}
        \item \textbf{OSINT (Open-Source Intelligence):} Genel kullanıma açık ve serbestçe erişilebilen kaynaklardan bilgi toplama. Bu kaynaklar arasında haberler, bloglar, sosyal medya platformları, akademik çalışmalar ve endüstri raporları bulunur.
        \item \textbf{Ticari Tehdit Beslemeleri:} Güvenlik firmaları tarafından sağlanan ve genellikle otomatik sistemlere entegre edilen ücretli veri akışlarıdır. Bu beslemeler, binlerce IOC ve TTP bilgisini yüksek hacimde sunabilir.
        \item \textbf{Bilgi Paylaşım Toplulukları (ISACs):} Aynı sektördeki veya coğrafi bölgedeki organizasyonların tehdit istihbaratı paylaşımı için bir araya geldiği güvenilir platformlardır.
        \item \textbf{Dahili Kaynaklar:} Organizasyonun kendi güvenlik araçlarından (SIEM, EDR, IDS/IPS), ağ günlüklerinden, uç nokta telemetri verilerinden ve kimlik doğrulama kayıtlarından elde edilen verilerdir.
        \item \textbf{Derin ve Karanlık Web İzleme:} Gizli veya şifreli forumlar, siber suç pazarları ve sızdırılmış veri depolarından (örneğin, sızan kimlik bilgileri, saldırı planları) bilgi toplama.
    \end{itemize}
    \item \textbf{İşleme (Processing):} Toplanan ham verinin analiz edilebilir, yapılandırılmış ve temiz bir formata dönüştürüldüğü aşamadır. Bu, veri normalizasyonu (farklı formatlardaki verileri standartlaştırma), tekilleştirme (yinelenen kayıtları kaldırma), şifre çözme ve etiketleme işlemlerini içerir. Büyük veri kümeleri için otomasyon, bu aşamada hayati öneme sahiptir.
    \item \textbf{Analiz ve Yorumlama (Analysis and Interpretation):} İşlenmiş verinin anlamlı, eyleme dönüştürülebilir istihbarata dönüştüğü aşamadır. Analistler, kalıpları, eğilimleri, saldırı kampanyalarını ve potansiyel tehditleri belirlemek için verileri derinlemesine incelerler. Bu aşamada, insan uzmanlığı ve otomatik korelasyon mekanizmaları birleşir.
    \item \textbf{Yayma (Dissemination):} Analiz edilen istihbaratın ilgili paydaşlara zamanında ve uygun bir formatta sunulmasıdır. İstihbaratın formatı ve içeriği, hedef kitleye göre uyarlanmalıdır. Örneğin, CISO'lar için üst düzey yönetici brifingleri hazırlanırken, SOC analistleri için teknik IOC beslemleri sunulur.
    \item \textbf{Geri Bildirim (Feedback):} Sürecin sürekli iyileştirilmesini sağlayan bu aşamada, istihbarat tüketicileri (örneğin, SOC analistleri), sunulan bilgilerin yararlılığı, alaka düzeyi ve doğruluğu hakkında geri bildirimde bulunur. Bu geri bildirim, bir sonraki döngüde daha iyi gereksinimlerin belirlenmesini ve istihbarat toplama stratejilerinin hassaslaştırılmasını sağlar.
\end{enumerate}

Bu yaşam döngüsü, bir organizasyonun reaktif tehdit algılama modelinden proaktif bir savunma duruşuna geçiş yapmasına olanak tanır. Bu sürekli ve döngüsel süreç, savunma mekanizmalarının saldırganların evrilen taktik, teknik ve prosedürlerine (TTP'ler) göre dinamik olarak ayarlanmasını sağlar. Bu dinamik adaptasyon, bir saldırganın bir sistemde kalma süresini (dwell time) doğrudan azaltır. Saldırganın sistemde kalma süresinin azalması, fidye yazılımı veya veri sızdırma gibi nihai hedeflere ulaşma şansını düşürürken, ihlalden kaynaklanan maliyetleri de önemli ölçüde azaltır. Bu sürekli öğrenme ve adaptasyon mekanizması, tehdit istihbaratı programının değerini ve etkinliğini katlanarak artırır.

\textbf{CTI Yaşam Döngüsü Aşamaları ve Uygulamaları}

\begin{longtable}{|p{2.5cm}|p{3cm}|p{3cm}|p{3.5cm}|}
\hline
\textbf{Aşama} & \textbf{Kısa Tanım} & \textbf{Hedef} & \textbf{Pratik Uygulama} \\
\hline
\textbf{Gereksinimler} & İhtiyaç duyulan istihbaratın belirlenmesi & Kaynakları kritik risklere odaklamak & Tehdit aktörü TTP analizi \\
\hline
\textbf{Toplama} & Ham verinin kaynaklardan alınması & Gereksinim odaklı veri toplama & SIEM günlükleri, OSINT, karanlık web \\
\hline
\textbf{İşleme} & Ham veriyi analiz formatına getirme & Veri yapılandırma ve zenginleştirme & CSV normalizasyonu, VirusTotal entegrasyonu \\
\hline
\textbf{Analiz} & Veriden içgörüler çıkarma & Tehdit bağlamsallaştırma & Oltalama kampanya teknik analizi \\
\hline
\textbf{Yayma} & İstihbaratı paydaşlara iletme & Doğru kitleye doğru formatta sunum & Yönetici brifingi, SOC beslemesi \\
\hline
\textbf{Geri Bildirim} & Etkinlik geri bildirimi toplama & Sürekli iyileştirme & Yanlış pozitif raporları, değerlendirmeler \\
\hline
\end{longtable}

\subsection{Strategic, Tactical, Technical ve Operational Intelligence}

Siber tehdit istihbaratı, hedef kitlenin ihtiyaçlarına göre dört ana kategoriye ayrılır. Bu istihbarat türleri, bir organizasyonun savunma yeteneklerini tüm seviyelerde güçlendirmek için birlikte çalışır.

\begin{itemize}
    \item \textbf{Stratejik İstihbarat:} Genel tehdit ortamına ilişkin üst düzey bir bakış açısı sunar. Bu istihbarat, teknoloji dışı terimlerle hazırlanır ve öncelikli olarak üst düzey yöneticilere, CISO'lara ve risk yöneticilerine yöneliktir.
    \begin{itemize}
        \item \textbf{Amacı:} Güvenlik yatırımları, bütçe tahsisi ve kurumsal politikalar gibi uzun vadeli stratejik kararlara rehberlik etmektir.
        \item \textbf{Örnekler:} Bir APT grubunun bir sektörü veya belirli bir coğrafyayı hedeflemesi, fidye yazılımı eğilimleri veya jeopolitik olayların siber saldırı risklerine etkisi hakkında raporlar.
    \end{itemize}
    \item \textbf{Operasyonel İstihbarat:} Belirli bir tehdit aktörünün TTP'lerini, motivasyonlarını ve altyapısını anlamayı sağlar. Bu istihbarat, olay müdahale (IR) ekipleri ve tehdit avcıları için hayati öneme sahiptir.
    \begin{itemize}
        \item \textbf{Amacı:} Yaklaşan veya devam eden bir saldırının "kim, ne zaman, nerede, nasıl, neden" sorularına bağlamsal cevaplar sunmak. Bu, proaktif tehdit avcılığı ve olay müdahale planlaması için bir temel oluşturur.
        \item \textbf{Örnekler:} Bir saldırıda kullanılan belirli bir oltalama e-postası kampanyasının detayları veya bir Komuta ve Kontrol (C2) sunucusunun iletişim yöntemleri.
    \end{itemize}
    \item \textbf{Taktik İstihbarat:} Ağ ve uç noktalarda tehditleri tespit etmeye yardımcı olan, genellikle kısa ömürlü ve teknik göstergelerden oluşan bir istihbarat türüdür.
    \begin{itemize}
        \item \textbf{Amacı:} Otomatik tehdit tespiti için SIEM ve güvenlik duvarı gibi güvenlik kontrollerine entegre edilmek.
        \item \textbf{Örnekler:} Kötü amaçlı IP adresleri, dosya karmaları (hashes), kötü amaçlı alan adları.
    \end{itemize}
    \item \textbf{Teknik İstihbarat:} Taktik istihbaratın daha derin teknik detaylarını içerir ve genellikle kötü amaçlı yazılım analizi, tersine mühendislik ve IoC'lerin oluşturulmasını kapsar.
    \begin{itemize}
        \item \textbf{Amacı:} Saldırıların teknik işleyişini anlamak ve bu bilgilere dayanarak yeni imza tabanlı tespit kuralları oluşturmaktır.
    \end{itemize}
\end{itemize}

Bu dört istihbarat türü, birbirini besleyen ve destekleyen bir hiyerarşi içinde çalışır. Üst yönetim, genel güvenlik stratejisini belirlemek için stratejik istihbarata ihtiyaç duyar. Bu strateji, operasyonel ve taktiksel gereksinimlere dönüşür. Örneğin, stratejik istihbarat, jeopolitik gerilimlerin bir APT grubunun faaliyetlerini artıracağını gösterdiğinde, operasyonel ekip bu grubun TTP'lerini incelemeye odaklanır ve bu operasyonel bilgiler taktiksel IoC'lere dönüştürülerek güvenlik kontrollerine entegre edilir. Bu entegrasyon, istihbaratın tüm organizasyonel katmanlarda değer yaratmasını sağlar.

\textbf{İstihbarat Türleri Karşılaştırma Tablosu}

\begin{longtable}{|p{2cm}|p{3.5cm}|p{3.5cm}|p{3cm}|}
\hline
\textbf{Özellik} & \textbf{Stratejik} & \textbf{Operasyonel} & \textbf{Taktik} \\
\hline
\textbf{Hedef Kitle} & C-seviyesi yöneticiler & IR ekipleri, tehdit avcıları & SOC analistleri \\
\hline
\textbf{Odak Noktası} & Genel eğilimler, risk yönetimi & Saldırı kampanyaları, TTP'ler & IOC'ler, ağ etkinlikleri \\
\hline
\textbf{Zaman} & Uzun vadeli (aylar/yıllar) & Orta vadeli (hafta/ay) & Kısa vadeli (gerçek zamanlı) \\
\hline
\textbf{Amaç} & Politika ve yatırım belirleme & Müdahale ve proaktif avcılık & Otomatik tespit ve engelleme \\
\hline
\end{longtable}

\subsection{Threat Actor Profiling ve Attribution Challenges}

Tehdit aktörü profilleme, saldırıların arkasındaki "kim" sorusuna cevap aramayı amaçlar. Bir tehdit aktörü profili, sadece bir isimden ibaret değildir; bir saldırganın kimliğini, hedeflerini, TTP'lerini, motivasyonlarını, coğrafi konumunu ve kullandığı altyapıyı kapsamlı bir şekilde analiz eder. Bu profiller, güvenlik ekiplerine saldırganın olası davranışları hakkında bir resim sunar ve savunma stratejilerini saldırganın niyetleriyle uyumlu hale getirmeye yardımcı olur.

Ancak, bir siber saldırıyı kesin olarak belirli bir tehdit aktörüne atfetmek (attribution), son derece karmaşık ve zorlu bir süreçtir. Bu zorlukların birkaç temel nedeni vardır:

\begin{itemize}
    \item \textbf{Gizleme ve Obfüskasyon:} Saldırganlar, kimliklerini gizlemek için çok sayıda katman kullanır. Botnetler, vekil sunucular ve kiralanmış altyapı, saldırının gerçek kaynağını maskeler. Özellikle dağıtık servis dışı bırakma (DDoS) saldırılarında, trafik binlerce farklı cihazdan gelebilir, bu da kesin bir atıf yapmayı neredeyse imkansız hale getirir.
    \item \textbf{Yanlış Bayrak Operasyonları:} Bazı siber suç grupları, dikkat çekmek veya yanlış bilgi yaymak amacıyla gerçekleştirmediği saldırıların sorumluluğunu üstlenebilir. Bu tür eylemler, atıf sürecini manipüle etmeyi amaçlar.
    \item \textbf{Araç ve TTP Paylaşımı:} Tehdit grupları, sıklıkla araç setlerini ve TTP'lerini birbirleriyle paylaşır veya değiştirir. Bir saldırıda kullanılan belirli bir teknik, daha önce bilinen bir grubun imzası olsa bile, başka bir grup tarafından da kullanılmış olabilir. Bu durum, yalnızca teknik IOC'lere dayalı atıf yapmanın güvenilirliğini azaltır.
\end{itemize}

Bu nedenle, atıf birincil hedef olmamalıdır. Bir saldırı meydana geldiğinde, ilk ve en acil öncelik, hasarı durdurmak, sistemleri güvence altına almak ve devam eden tehditleri ortadan kaldırmaktır. Atıf, bu acil riskler ortadan kaldırıldıktan sonra, olay sonrası analiz aşamasında daha sonraki savunmaları güçlendirmek için bir araç olarak ele alınmalıdır. Atıf, bir "evet/hayır" cevabı yerine, farklı güven seviyelerine sahip (düşük, orta, yüksek) analitik bir değerlendirme süreci olarak görülmelidir. Güvenilir bir atıf için, sadece IOC'ler yerine, davranışsal kanıtlar ve TTP'lere odaklanmak daha geçerli bir yaklaşımdır.

\subsection{Diamond Model ve Kill Chain Analysis}

Siber saldırıları anlamak ve analiz etmek için kullanılan iki önemli analitik çerçeve, Lockheed Martin'in Siber Kill Chain’i ve MITRE’nin Saldırı Analizinin Elmas Modeli’dir (Diamond Model of Intrusion Analysis). Bu modeller, saldırı sürecini farklı açılardan ele alarak birbirini tamamlar.

\begin{itemize}
    \item \textbf{Siber Kill Chain (Siber Saldırı Zinciri):} Bu model, bir saldırının yedi aşamalı, doğrusal bir sürecini sunar. Her aşama, bir saldırganın başarılı bir siber saldırı gerçekleştirmek için tipik olarak izlediği adımları tanımlar.
    \begin{enumerate}
        \item \textbf{Keşif (Reconnaissance):} Saldırgan, hedefin sistemleri, ağ yapısı ve çalışanları hakkında bilgi toplar.
        \item \textbf{Silahlandırma (Weaponization):} Bir exploit ve bir arka kapı (payload) bir araya getirilerek tek bir saldırı paketi oluşturulur.
        \item \textbf{Teslimat (Delivery):} Saldırı paketi, hedef sisteme iletilir (örneğin, oltalama e-postası veya kötü amaçlı web sitesi aracılığıyla).
        \item \textbf{İstismar (Exploitation):} Bir sistemdeki zafiyet kullanılarak ilk erişim elde edilir.
        \item \textbf{Kurulum (Installation):} Saldırgan, sistemde kalıcılığı sağlamak için arka kapıyı kurar.
        \item \textbf{Komuta ve Kontrol (Command and Control):} Saldırganın ele geçirilmiş sistemle uzaktan iletişim kurarak onu kontrol etmesini sağlar.
        \item \textbf{Hedefler Üzerindeki Eylemler (Actions on Objectives):} Saldırgan, veri sızdırma, sistem bozulması veya fidye gibi nihai hedeflerine ulaşır.
    \end{enumerate}
    Kill Chain, saldırıların nasıl ilerlediğine dair adım adım bir yol haritası sunar ve özellikle olay müdahalesi ve taktiksel savunma kararları için çok değerlidir.
    \item \textbf{Diamond Model of Intrusion Analysis (Saldırı Analizinin Elmas Modeli):} Bu model, siber saldırıları dört temel bileşen arasındaki ilişkilere odaklanarak inceler. Doğrusal bir model değildir, daha çok saldırının bütüncül bir resmini sunar.
    \begin{enumerate}
        \item \textbf{Saldırgan (Adversary):} Saldırının arkasındaki tehdit aktörü.
        \item \textbf{Kabiliyet (Capability):} Saldırganın kullandığı araçlar, teknikler ve yöntemler.
        \item \textbf{Altyapı (Infrastructure):} Saldırıyı desteklemek için kullanılan ağ altyapısı (örneğin, C2 sunucuları, vekil sunucular).
        \item \textbf{Kurban (Victim):} Saldırıya uğrayan kişi, organizasyon veya sistem.
    \end{enumerate}
    Elmas Modeli, tehdit istihbaratı ve proaktif tehdit avcılığı için daha kullanışlıdır. Saldırıdan elde edilen herhangi bir bilgi parçası (örneğin, bir C2 sunucusu), diğer üç bileşenle ilişkilendirilerek saldırganın potansiyel diğer faaliyetlerini ve gelecekteki olası hedeflerini tahmin etmek için kullanılabilir.
\end{itemize}

Kill Chain, bir saldırının \textit{nasıl} ilerlediğini detaylandırırken, Diamond Model saldırının \textit{neden} ve \textit{kim tarafından} yapıldığını daha geniş bir bakış açısı sunar. Bu modeller birbirini tamamlar. Bir saldırı tespit edildiğinde, Kill Chain modeli olayın taktiksel olarak yönetilmesine yardımcı olurken, Diamond Model olayın bileşenlerini analiz ederek bu olayı bilinen diğer tehdit gruplarıyla ilişkilendirmeyi ve gelecekteki potansiyel saldırıları tahmin etmeyi mümkün kılar.

% Kill Chain ve Diamond Model Karşılaştırma Tablosu
\begin{longtable}{|p{3cm}|p{5.5cm}|p{5.5cm}|}
\hline
\textbf{Özellik} & \textbf{Siber Kill Chain} & \textbf{Diamond Model} \\
\hline
\textbf{Odak} & Saldırı aşamaları ve süreci & Aktör-kabiliyet-altyapı-kurban ilişkileri \\
\hline
\textbf{Granülarite} & Spesifik ve doğrusal & Geniş, bağlamsal ve ilişkisel \\
\hline
\textbf{Uygulama} & Olay müdahalesi, taktiksel savunma & Tehdit istihbaratı, proaktif avcılık \\
\hline
\textbf{Avantajlar} & Adım adım müdahale yol haritası & Motivasyon ve kabiliyet analizi \\
\hline
\end{longtable}