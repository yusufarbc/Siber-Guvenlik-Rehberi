\chapter{SOSYAL MÜHENDİSLİK VE İNSAN FAKTÖRÜ}

\section*{Giriş}
Sosyal mühendislik ve insan faktörü, siber güvenlikte teknolojik savunmaların ötesinde insan psikolojisine dayalı saldırılarla ilgili kritik bir alandır. Bu bölümde sosyal mühendislik teknikleri, farkındalık eğitimleri ve insan odaklı güvenlik stratejilerini ele alacağız.

\begin{longtable}{|>{\raggedright\arraybackslash}p{2.5cm}|>{\raggedright\arraybackslash}p{3.5cm}|>{\centering\arraybackslash}p{3cm}|>{\raggedright\arraybackslash}p{3.5cm}|}
\caption{OSINT Araçları ve Kullanım Alanları} \\
\hline
\rowcolor{tableheadcolor}
\textbf{Araç} & \textbf{Açıklama} & \textbf{Kullanım} & \textbf{Komut} \\
\hline
\endfirsthead
\multicolumn{4}{c}{\small\tablename\ \thetable\ -- devamı} \\
\hline
\rowcolor{tableheadcolor}
\textbf{Araç} & \textbf{Açıklama} & \textbf{Kullanım} & \textbf{Komut} \\
\hline
\endhead
\hline
\multicolumn{4}{r}{\small Devamı sonraki sayfada} \\
\endfoot
\hline
\endlastfoot
\textbf{theHarvester} & E-posta, alt alan adı ve IP bilgilerini toplar. & Keşif (footprinting) & \texttt{theharvester -d example.com -b all} \\
\hline
\textbf{Maltego} & Veri ilişkilendirme ve görselleştirme aracı. & Profil ve e-posta haritalama & \texttt{maltego} \\
\hline
\textbf{SET} & Python tabanlı sosyal müh. aracı. & Oltalama ve sahte site & \texttt{sudo apt install set} \\
\end{longtable}

\section{İnsan Psikolojisi ve Sosyal Mühendislik Temelleri}

Sosyal mühendislik, insan davranışlarının ve psikolojik eğilimlerinin, bir kişiyi normalde yapmayacağı bir eylemi gerçekleştirmesi veya gizli bilgileri ifşa etmesi için manipüle edilmesi sanatıdır. Bu, teknolojik zafiyetlerden ziyade, insan doğasının öngörülebilir tepkilerine dayanan bir saldırı vektörüdür. Sosyal mühendisler, hedeflerinin güvenini kazanmak, korku veya aciliyet hissi yaratmak veya bir otorite figürünü taklit etmek gibi çeşitli taktikler kullanırlar.

\subsection{Bilişsel Yanılgılar ve Karar Verme Zafiyetleri}

İnsan beyninin karar verme süreci, Nobel Ödüllü psikolog Daniel Kahneman'ın "Sistem 1" ve "Sistem 2" teorisi ile açıklanır. Sistem 1, hızlı, sezgisel ve otomatik tepkiler verirken, Sistem 2 daha yavaş, analitik ve mantıksal düşünme süreçlerini yönetir. Siber tehditler karşısında, özellikle stres veya bilişsel aşırı yük altında kalan çalışanlar, karmaşık dijital risklere uyum sağlayamayan Sistem 1'i aktive etme eğilimindedir. Bu durum, güvenlik eğitimi almış kişilerin bile basit hatalar yapmasına, örneğin sahte bir bağlantıya tıklamasına yol açabilir.

Saldırganlar, insan beyninin bu iki sistem arasındaki zayıflıktan yararlanarak, belirli bilişsel yanılgıları hedeflerler:

\begin{itemize}
    \item \textbf{Kötümserlikten Uzak Durma (Unrealistic Optimism):} Bu yanılgı, insanların "bu sadece başkasının başına gelir" inancını yansıtır. Çalışanlar, siber saldırı riskini kişisel olarak hafife alarak, güvenlik yönergelerini göz ardı etme veya şüpheli ekleri doğrulamama gibi pervasız davranışlar sergileyebilir.
    \item \textbf{Otorite Yanılgısı (Authority Bias):} Bireylerin, meşru olarak algılanan bir figürden gelen talimatlara sorgusuz uymasıdır. Saldırganlar, üst düzey yöneticileri veya tanınmış kurumları taklit ederek bu yanılgıdan yararlanır ve fon transferi veya hassas bilgi ifşası gibi aceleci eylemleri tetikler.
    \item \textbf{Tanıdıklık ve Alışkanlık Yanılgısı (Familiarity Bias):} İnsanların, tanıdık gelen bir şeye karşı gardını indirme eğilimidir. Yoğun bilgi akışı altında çalışan bir kişi, tanıdık bir isimden veya logodan gelen mesajın orijinalliğini doğrulamak yerine, hızlı bir yanıt vermeyi tercih edebilir.
    \item \textbf{Kullanılabilirlik Yanılgısı (Availability Bias):} Bir riskin olasılığını, yakın zamanda yaşanan deneyimlere dayanarak değerlendirme eğilimidir. Bir çalışan daha önce hiç oltalama saldırısıyla karşılaşmadıysa, bu tehdidi düşük bir tehlike olarak algılayabilir.
    \item \textbf{Olumsuz Duygular ve Bilişsel Yük:} Yorgunluk, stres ve bilişsel aşırı yük, analitik yetenekleri zayıflatarak muhakeme yeteneğini bozar. Bu durum, "acil" gibi terimler içeren e-postaları alışkanlıkla açmaya veya hızlı ve içgüdüsel kararlar vermeye neden olur.
\end{itemize}

\subsection{Güvenlik Bağlamında Sosyal Psikoloji İlkeleri}

Davranışsal psikolog Robert Cialdini'nin ikna prensipleri, sosyal mühendislik saldırganlarının manipülasyon taktiklerinin temelini oluşturur. Bu prensipler, insan davranışını yönlendiren evrensel eğilimleri kullanır ve saldırganların bu ilkelerden nasıl yararlandığı aşağıda detaylandırılmıştır:

\begin{itemize}
    \item \textbf{Karşılık Verme (Reciprocity):} İnsanlar, bir iyiliğe veya hediyeye karşılık verme eğilimindedir. Saldırganlar, kurbanlarına "ücretsiz bir araç" veya "yardımcı bir bilgi" sunarak bir borçluluk duygusu yaratır ve bu sayede daha büyük bir talep için zemin hazırlarlar.
    \item \textbf{Taahhüt ve Tutarlılık (Commitment and Consistency):} Bir kez küçük bir eylemde bulunan bireyler, daha sonra bu davranışlarıyla tutarlı kalmaya daha yatkın olurlar. Saldırganlar, masum bir taleple başlayıp, kademeli olarak daha hassas bilgi taleplerine geçerek bu prensibi kullanır.
    \item \textbf{Sosyal Kanıt (Social Proof):} Bir eylemin veya fikrin başkaları tarafından da benimsendiğine dair kanıt görmek, uyma eğilimini artırır. Saldırganlar, sahte referanslar veya "meslektaşlarınızın \%90'ı zaten şifrelerini güncelledi" gibi mesajlar kullanarak bu prensibi istismar eder.
    \item \textbf{Otorite (Authority):} İnsanlar, yetkili olarak algılanan figürlerden gelen isteklere boyun eğmeye daha yatkındır. Saldırganlar, CEO'ları, IT yöneticilerini veya tanınmış kurumları taklit ederek bu prensipten yararlanır.
    \item \textbf{Beğenme (Liking):} Bireyler, tanıdık, benzer veya kendilerinden biri olarak gördükleri kişilere daha kolay güvenir. Saldırganlar, ortak ilgi alanlarını veya sosyal bağlantıları kullanarak güven oluşturur.
    \item \textbf{Kıtlık (Scarcity):} Bir şeyin sınırlı sayıda veya sürede olduğuna inanıldığında, ona olan talep artar. "Yalnızca 5 koltuk kaldı" veya "bu özel erişim 24 saat içinde sona eriyor" gibi aciliyet ifadeleri, hızlı ve düşüncesiz kararlar almaya yol açar.
\end{itemize}

\subsection{Güven Oluşturma ve İkna Teknikleri}

Güven, bir sosyal mühendislik saldırısının temelini oluşturur ve saldırganlar, sahte bir kimlik (pretext) oluşturarak bu güveni inşa eder. Bu süreç, hedefin adını, çalıştığı departmanı, teknoloji kullanım alışkanlıklarını ve hatta kişisel/profesyonel ilişkilerini bilmek gibi detayları içerir. Bu bilgilerle, saldırganlar inandırıcı bir senaryo hazırlar ve kurbanın şüpheciliğini bypass eder.

Daha derinlemesine ikna teknikleri arasında, hedefe sempati duymak, karşılıklılık ilkesini kullanmak ve egoyu okşamak bulunur. Örneğin, bir saldırgan, bir aile acil durumuyla ilgili yardım istiyormuş gibi yaparak hedefin sempati duygusunu harekete geçirebilir veya küçük bir iyilik yaparak bir borçluluk bağı kurabilir.

\subsection{Psikolojik Profilleme ve Hedef Seçim Metotları}

Bir sosyal mühendislik saldırısının ilk ve en kritik adımı, hedef hakkında kapsamlı bilgi toplamaktır. Bu süreç, pasif ve aktif bilgi toplama olarak ikiye ayrılır. Pasif bilgi toplama, hedeften bağımsız olarak halka açık verilerin analizini içerir. Sosyal medya platformları (LinkedIn, Facebook, Twitter) bu aşamada kritik bir rol oynar, çünkü saldırganlar hedefin ilgi alanları, işi ve ailesi hakkında değerli bilgiler edinebilir ve bu bilgiler kişiselleştirilmiş bir saldırı için kullanılabilir. Aktif bilgi toplama ise hedefle doğrudan etkileşimi gerektirir, ancak bu, gündelik bir sohbet stratejisiyle yapılır ve hedefin sorgulanıyormuş gibi hissetmesi engellenir.

Sosyal mühendislik saldırılarının başarısı, yalnızca teknolojik zafiyetlere değil, aynı zamanda bilişsel yanılgıları istismar eden ve duygusal tepkileri tetikleyen ikna prensiplerine de dayanır. Saldırgan, bilişsel yanılgıları istismar eden ikna prensiplerini kullanarak, hedefin beynindeki hızlı ve sezgisel karar veren Sistem 1'ini aktive eder. Bu, hedefin mantıksal doğrulama süreçlerini (Sistem 2) atlamasını sağlar. Bu zincirleme reaksiyon, saldırganın az çabayla büyük bir başarı elde etmesini mümkün kılar. Örneğin, bir "CEO dolandırıcılığı" senaryosu, otorite yanılgısını kullanarak bir finans çalışanının normalde yapmayacağı bir fon transferini gerçekleştirmesini sağlayabilir.

İnsan beynindeki bu "özelliklerin" istismarı, savunma stratejilerinin sadece teknik kontrollere (güvenlik duvarı, anti-virüs yazılımları) odaklanamayacağını, aynı zamanda psikolojik güvenliği (PsySec) de içermesi gerektiğini gösterir. Sosyal mühendislik, sadece bilgi çalma girişimi değil, aynı zamanda bir psikolojik durum manipülasyonudur. Saldırgan, önce halka açık kaynaklardan (OSINT) bilgi toplayarak bir hedef belirler. Ardından, aciliyet duygusu yaratan bir senaryo (örneğin, bir hesabın askıya alınma tehdidi) tasarlar. Bu senaryo, bilişsel olarak aşırı yüklenmiş bir çalışanı hızlı ve düşüncesiz bir eyleme sevk eder. Bu, sosyal mühendisliğin neden bu kadar tehlikeli ve kalıcı bir tehdit olduğunu açıklar.

\textbf{Bilişsel Yanılgılar ve İkna İlkeleri Karşılaştırması:}

\begin{itemize}
    \item \textbf{Otorite Yanılgısı}
    \begin{itemize}
        \item \textbf{Tanım:} Bireylerin yetkili figürlerin talimatlarına uyma eğilimi
        \item \textbf{Sosyal Mühendislik Taktiği:} CEO, IT yöneticisi veya tanınmış kurumları taklit etme
        \item \textbf{Saldırı Senaryosu:} Bir finans çalışanına, CEO'dan geldiği iddia edilen bir e-posta ile acil fon transferi talimatı gönderilmesi
    \end{itemize}
    
    \item \textbf{Kıtlık}
    \begin{itemize}
        \item \textbf{Tanım:} Bir şeyin sınırlı olduğu algısının talebi artırması
        \item \textbf{Sosyal Mühendislik Taktiği:} "Sınırlı sayıda," "yalnızca bugün" gibi aciliyet ifadeleri kullanma
        \item \textbf{Saldırı Senaryosu:} "Hesabınızın güvenliği tehlikede, 1 saat içinde şifrenizi güncellemezseniz hesabınız askıya alınacak."
    \end{itemize}
    
    \item \textbf{Alışkanlık Yanılgısı}
    \begin{itemize}
        \item \textbf{Tanım:} Tanıdık gelen bir şeye karşı gardını indirme eğilimi
        \item \textbf{Sosyal Mühendislik Taktiği:} Sahte e-posta veya web sitelerinde tanıdık marka veya logo kullanımı
        \item \textbf{Saldırı Senaryosu:} Microsoft veya banka logosu içeren, gerçekle birebir aynı görünen bir oturum açma sayfası
    \end{itemize}
    
    \item \textbf{Karşılık Verme}
    \begin{itemize}
        \item \textbf{Tanım:} Bir iyiliğe karşılık verme yükümlülüğü hissetme
        \item \textbf{Sosyal Mühendislik Taktiği:} Ücretsiz bir hediye, indirim veya yardımcı bir araç sunma
        \item \textbf{Saldırı Senaryosu:} "Ücretsiz bir güvenlik aracı" indirme bağlantısı içeren bir e-posta gönderilmesi; bu aracın aslında kötü amaçlı yazılım olması
    \end{itemize}
\end{itemize}

\section{Dijital Sosyal Mühendislik Saldırı Teknikleri}

Dijital sosyal mühendislik, teknolojiyi kullanarak insanları manipüle etme ve kandırma sürecidir. Bu saldırılar, genellikle e-posta, sosyal medya, anlık mesajlaşma ve diğer dijital iletişim kanalları aracılığıyla gerçekleştirilir. Saldırganlar, hedeflerini kötü amaçlı yazılım indirmeye, sahte web sitelerine kişisel bilgilerini girmeye veya yetkisiz para transferleri yapmaya ikna etmek için çeşitli taktikler kullanırlar.

\subsection{Oltalama (Phishing) ve Hedef Odaklı Oltalama (Spear-phishing) Kampanya Tasarımı}

\begin{itemize}
    \item \textbf{Genel Oltalama (Phishing):} Geniş bir kitleye gönderilen, güvenilir bir kaynaktan (banka, e-posta sağlayıcısı) geliyormuş gibi görünen dolandırıcılık girişimleridir. Saldırının amacı, kurbanın kişisel bilgilerini veya kimlik bilgilerini ele geçirmek için kötü amaçlı bir bağlantıya tıklamasını veya sahte bir web sitesine yönlendirilmesini sağlamaktır. Geleneksel olarak imla hataları gibi bariz işaretlerle tanınsa da, günümüzde bu saldırılar çok daha sofistike ve inandırıcıdır.
    \item \textbf{Hedef Odaklı Oltalama (Spear-phishing):} Belirli bir kişiyi veya grubu hedef alan, çok daha kişisel ve incelikli bir oltalama türüdür. Saldırgan, hedefin sosyal medyadaki dijital ayak izini (LinkedIn, Facebook, vb.) kapsamlı bir şekilde araştırarak, hedefin bir meslektaşından veya yöneticisinden geliyormuş gibi görünen ikna edici e-postalar hazırlar. Bu kişiselleştirme, kurbanın mesajın meşruiyetine inanma olasılığını büyük ölçüde artırır.
\end{itemize}

\subsection{İş E-postası Ele Geçirme (BEC) ve CEO Dolandırıcılığı}

\begin{itemize}
    \item \textbf{BEC (Business Email Compromise):} Finansal işlemleri hedef alan, sosyal mühendislik tabanlı bir e-posta saldırısıdır. Bu saldırılar, kötü amaçlı bir bağlantı veya ek içermemesiyle diğer oltalama türlerinden ayrılır ve bu sayede geleneksel e-posta filtrelerini atlatabilir. Saldırgan, bir yöneticiyi veya üçüncü taraf bir satıcıyı taklit ederek, hedefe bir fon transferi yapması talimatını verir.
    \item \textbf{CEO Dolandırıcılığı:} BEC'nin, özellikle üst düzey bir yöneticiyi (CEO, CFO) taklit etmeye odaklanan bir türüdür. Saldırgan, yöneticinin e-posta adresini taklit edebilir veya yazışma stilini taklit ederek mesajın inandırıcılığını artırır. Talebin aciliyeti vurgulanır ve hedefin normal doğrulama süreçlerini atlaması hedeflenir.
\end{itemize}

\subsection{Vishing (Sesli Oltalama) ve Smishing (SMS Oltalama)}

\begin{itemize}
    \item \textbf{Vishing:} Telefon görüşmeleri aracılığıyla gerçekleştirilen bir oltalama türüdür. Saldırganlar, bankalar, hükümet kurumları veya teknik destek temsilcileri gibi meşru kuruluşları taklit ederek kurbandan hassas bilgileri veya kredi kartı detaylarını doğrudan elde etmeyi amaçlar.
    \item \textbf{Smishing:} SMS veya anlık mesajlaşma uygulamaları (WhatsApp, Telegram) üzerinden gerçekleştirilen oltalama türüdür. Saldırgan, kötü amaçlı bir bağlantı içeren mesajlar gönderir ve insanlar SMS'lere hızlı tepki verme eğiliminde oldukları için smishing oldukça etkilidir. Ortak senaryolar arasında banka sahtekarlığı uyarıları, kargo teslimat bildirimleri veya sahte ödül bildirimleri bulunur.
\end{itemize}

\subsection{Sosyal Medya Manipülasyonu ve Sahte Profiller}

Sosyal medya manipülasyonu, sahte profiller oluşturarak veya mevcut hesapları ele geçirerek kurbanın güvenini kazanmayı amaçlayan bir dijital kimlik hırsızlığı biçimidir. Saldırganlar, kişisel bilgileri (isim, resim, konum) çalarak gerçeğe yakın sahte profiller oluşturur ve bu profilleri, kurbanın çevresindeki kişilerle etkileşim kurarak bilgi toplamak ve spear-phishing veya başka bir saldırı için gerekli istihbaratı elde etmek için kullanırlar.

\subsection{Derin Sahte (Deepfake) ve Yapay Zeka Üretimi İçerik Kullanımı}

\begin{itemize}
    \item \textbf{Deepfake:} Yapay zeka kullanılarak oluşturulan ve gerçekçi görünen sahte video, ses veya görsel içeriklerdir. Bu teknoloji, bir kişinin sesini taklit ederek (ses klonlama) veya yüzünü başka bir videodaki bir kişinin üzerine bindirerek kullanılır.
    \item \textbf{Kullanım Alanları:} Saldırganlar, bir CEO'nun sesini klonlayarak bir finans çalışanına fon transferi talimatı vermek için kullanabilir. Bu tür saldırılar, sesin veya görüntünün anlık doğrulamasının zorluğundan yararlanır.
    \item \textbf{Teknik Adımlar (Ses Klonlama):}
    \begin{enumerate}
        \item \textbf{Veri Toplama:} Kurbanın sesinden örnekler (sosyal medya videoları, sesli mesajlar, röportajlar) toplanır. Birkaç saniyelik bir ses örneği bile yeterli olabilir.
        \item \textbf{Model Eğitimi:} Toplanan ses örnekleri, yapay zeka ses sentezi modellerine (Tacotron 2, Vall-E gibi) beslenir. Model, kurbanın sesindeki benzersiz özellikleri (tonlama, ritim, nefes alma kalıpları) öğrenir.
        \item \textbf{Sahte Ses Üretimi:} Saldırgan, istediği metni yazar ve model, bu metni kurbanın sesini kullanarak okur. Bu, önceden oluşturulmuş bir ses dosyası olarak kullanılabileceği gibi, gerçek zamanlı bir telefon görüşmesinde de uygulanabilir.
    \end{enumerate}
\end{itemize}

Geleneksel olarak, kitlesel oltalama (phishing) geniş bir kitleyi hedef alırken, hedef odaklı oltalama (spear-phishing) yüksek başarı oranı için daha fazla çaba gerektirir. Yapay zeka bu çelişkiyi ortadan kaldırmaktadır. Saldırganlar artık, açık kaynak istihbarat araçlarıyla topladıkları verileri yapay zeka dil modellerine besleyerek, binlerce kişiye aynı anda gönderilebilen, dilbilgisi hatasız ve son derece kişiselleştirilmiş saldırılar oluşturabilirler. Bu yeni yaklaşım, geleneksel savunma yöntemlerini (imla hatalarını kontrol etme gibi) etkisiz hale getirir ve savunma paradigmalarının doğrulamaya ve sıfır güvene doğru kaymasını hızlandırır.

Ayrıca, siber suç ekosistemi de giderek daha karmaşık hale gelmektedir. Oltalama (phishing) kitlerinin (hazır oltalama web siteleri) incelenmesi, bu kitleri yazan geliştiricinin, kitleri kullanan saldırganlardan gizlice çaldıkları verilerin bir kopyasını kendilerine yönlendirdiğini göstermektedir. Bu, siber suç tedarik zincirindeki karmaşıklığı ve güven eksikliğini ortaya koymaktadır.

\section{Fiziksel Sosyal Mühendislik ve OSINT}

Fiziksel sosyal mühendislik, bir saldırganın hedefine fiziksel olarak yaklaşarak bilgi toplaması veya yetkisiz erişim sağlamasıdır. Bu, genellikle bir binaya veya kısıtlı bir alana sızmayı içerir. Açık Kaynak İstihbaratı (OSINT), halka açık kaynaklardan bilgi toplama sürecidir. Sosyal mühendisler, hedefleri hakkında bilgi toplamak ve saldırılarını daha inandırıcı hale getirmek için OSINT tekniklerini sıklıkla kullanırlar.

\subsection{Arkadan Girme (Tailgating), Omuzdan Gözetleme (Shoulder Surfing) ve Fiziksel Sızma}

\begin{itemize}
    \item \textbf{Arkadan Girme (Tailgating):} Bir yetkisiz kişinin, yetkili bir kişiyi takip ederek kısıtlı bir alana girmesidir. Saldırgan, "kapıyı açık tutar mısınız?" gibi kibarlığa dayalı bir sosyal normu istismar eder. Ellerinde paketler bulunan bir kurye gibi davranmak veya sahte bir kimlik kartı gösterme taklidi yapmak, bu tekniğin yaygın örneklerindendir.
    \item \textbf{Omuzdan Gözetleme (Shoulder Surfing):} Bir kişinin şifre, PIN veya diğer hassas bilgileri girerken omzunun üzerinden gizlice izlemesi veya kaydetmesidir. Bu, kafeler veya havaalanları gibi kamusal alanlarda çok yaygın, düşük teknolojili ancak etkili bir saldırıdır.
    \item \textbf{Fiziksel Sızma:} Daha genel bir kavram olup, saldırganın sahte bir kimlikle (pretext) bir binaya veya kısıtlı alana girmesini içerir. Bir merdiven veya alet kemeri taşımak gibi basit eylemler bile, saldırganın "ait olduğu" izlenimini yaratmasına yardımcı olur.
\end{itemize}

\subsection{Çöp Karıştırma (Dumpster Diving) ve Fiziksel Bilgi Toplama}

Çöp karıştırma, atılan fiziksel veya dijital atıklardan bilgi çıkarma eylemidir. Bu, ilkel bir yöntem gibi görünse de, saldırganlara paha biçilmez veriler sağlayabilir. Bu veriler arasında, müşteri listeleri, organizasyon şemaları, şifrelerin yazılı olduğu not defterleri, hatta eski kimlik kartları bulunabilir. Elde edilen bilgiler, hedef odaklı oltalama (spear-phishing) e-postaları hazırlamak için kullanılabilir. Bir organizasyon şemasının bulunması, saldırganın doğru kişileri hedeflemesini ve inandırıcı bir senaryo oluşturmasını kolaylaştırır. Bu, düşük teknolojili bir yöntemin nasıl yüksek etkili bir siber saldırının öncüsü olabileceğini gösterir.

\subsection{Ön Metin Oluşturma (Pretexting) ve Telefon Tabanlı Sosyal Mühendislik}

Pretexting, saldırganın kurbandan bilgi almak veya eylem yaptırmak için uydurulmuş bir senaryo (pretext) oluşturmasıdır. Saldırgan, IT teknisyeni, banka temsilcisi veya meslektaş gibi güvenilir bir figürü taklit eder. Bu senaryo, hedefin şüphelenmesini engellemek için mantıklı ve acil görünmelidir. Telefon tabanlı sosyal mühendislik, pretexting'in telefon üzerinden uygulanmasıdır. Saldırgan, arayanın numarasını (caller ID) sahte gösterebilir.

\subsection{Açık Kaynak İstihbaratı (OSINT) Toplama ve Sosyal Medya Keşfi}

Açık Kaynak İstihbaratı (OSINT), halka açık, ücretsiz kaynaklardan (arama motorları, sosyal medya, DNS kayıtları) istihbarat toplama sürecidir. Bu, sosyal mühendislik saldırısının hazırlık aşamasında kritik bir adımdır.
\begin{itemize}
    \item \textbf{theHarvester:} Alan adları, alt alan adları (subdomains), e-posta adresleri ve IP'ler gibi halka açık bilgileri toplamak için kullanılan yaygın bir araçtır.
    \begin{itemize}
        \item \textbf{Kullanım Örneği:} \verb|theharvester -d example.com -b all| komutu, \verb|example.com| alan adıyla ilişkili tüm halka açık bilgileri toplar.
    \end{itemize}
    \item \textbf{Maltego:} Bilgileri görsel bir grafik üzerinde ilişkilendiren bir veri madenciliği aracıdır. Sosyal medya profillerini, e-posta adreslerini ve ilgili altyapıları haritalandırarak saldırı için hedef seçimi ve kişiselleştirme için gerekli bilgileri sağlar.
    \item \textbf{Social-Engineer Toolkit (SET):} Sosyal mühendislik saldırıları için tasarlanmış açık kaynaklı, Python tabanlı bir araçtır. Oltalama, spear-phishing ve diğer saldırı vektörleri için hazır modüller sunar.
    \begin{itemize}
        \item \textbf{Kullanım Örneği:} \verb|sudo apt install set| komutu ile kurulum yapılır ve ardından \verb|setoolkit| komutu ile çalıştırılır.
    \end{itemize}
\end{itemize}

Sosyal mühendislik, siber veya fiziksel olmak üzere tek bir kategoriye ayrılamaz. Saldırganlar her iki alanı birleştiren bir "hibrit yaklaşım" kullanır. Saldırgan, bir çalışanın sosyal medya profilinden (dijital OSINT) adını ve rolünü öğrenir. Ardından, bir çöp karıştırma eylemiyle (fiziksel) o kişinin eski bir kimlik kartını veya dahili bir organizasyon şemasını bulur. Bu bilgilerle, IT teknisyeni kılığında ofise fiziksel olarak sızar veya hedefin telefonuna "teknik bir sorunu düzeltmek" için pretexting yapar. Bu zincirleme eylemler, saldırının tek bir zafiyeti değil, bir dizi zafiyeti istismar ettiğini gösterir. Bir güvenlik sisteminin en zayıf halkası, genellikle en düşük teknolojili saldırıya maruz kalan bileşenidir. Bir çalışanın çöpe attığı şifre notu, milyarlarca dolarlık bir şirketin güvenlik duvarını atlatmak için yeterli olabilir. Bu nedenle, savunma stratejilerinin bütünsel bir bakış açısıyla tasarlanması gerekir.


\begin{longtable}{|>{\raggedright\arraybackslash}p{2.5cm}|>{\raggedright\arraybackslash}p{3.5cm}|>{\centering\arraybackslash}p{3cm}|>{\raggedright\arraybackslash}p{3.5cm}|}
\caption{Fiziksel ve Dijital Sosyal Mühendislik Teknikleri} \\
\hline
\rowcolor{tableheadcolor}
\textbf{Teknik} & \textbf{Açıklama} & \textbf{Kullanım Alanı} & \textbf{Önleme} \\
\hline
\endfirsthead
\multicolumn{4}{c}{\small\tablename\ \thetable\ -- devamı} \\
\hline
\rowcolor{tableheadcolor}
\textbf{Teknik} & \textbf{Açıklama} & \textbf{Kullanım Alanı} & \textbf{Önleme} \\
\hline
\endhead
\hline
\multicolumn{4}{r}{\small Devamı sonraki sayfada} \\
\endfoot
\hline
\endlastfoot
\textbf{Tailgating} & Yetkili kişi arkasından sızma & Fiziksel erişim & Kapı güvenliği, eğitim \\
\hline
\textbf{Pretexting} & Sahte kimlik ile bilgi edinme & Telefon/e-posta & Doğrulama prosedürleri \\
\hline
\textbf{Baiting} & USB, CD gibi tuzak cihazlar & Zararlı yazılım yükleme & Bilinmeyen cihaz politikası \\
\end{longtable}

\section{Kurumsal Sosyal Mühendislik Zafiyetleri}

Kuruluşlar, hem teknolojik hem de insan kaynaklı çeşitli sosyal mühendislik zafiyetlerine sahiptir. Bu zafiyetler, çalışanların güvenlik farkındalığı eksikliğinden, zayıf güvenlik politikalarına ve yetersiz erişim kontrollerine kadar uzanabilir. Sosyal mühendisler, bu zafiyetleri istismar ederek, bir kuruluşun ağına sızabilir, hassas verileri çalabilir veya iş süreçlerini kesintiye uğratabilir.

\subsection{Çalışan Güvenlik Farkındalığı Açık Değerlendirmesi}

Güvenlik açık değerlendirmesi, bir organizasyonun mevcut güvenlik duruşunu, belirlenmiş standartlara göre değerlendiren sistematik bir süreçtir. Bu değerlendirme, çalışanların güvenlik farkındalığındaki boşlukları belirlemek için anketler, röportajlar ve belge incelemeleri gibi yöntemleri kullanır.
\begin{itemize}
    \item \textbf{Değerlendirme Aşamaları:}
    \begin{enumerate}
        \item \textbf{Kapsam Tanımlama:} Hangi sistemlerin, departmanların veya veri türlerinin değerlendirileceği belirlenir.
        \item \textbf{Veri Toplama:} Personel ile yapılan görüşmeler ve anketler, çalışanların davranışları hakkında nitel veriler sağlar.
        \item \textbf{Risk Değerlendirmesi:} Toplanan veriler, nicel veya nitel yöntemlerle analiz edilerek potansiyel riskler belirlenir.
        \item \textbf{Açık Tespiti:} Belirlenen riskler, NIST veya CIS gibi bilinen güvenlik çerçeveleriyle karşılaştırılarak mevcut zafiyetler ortaya konur.
        \item \textbf{Eylem Planı ve Raporlama:} Tespit edilen açıklar için önceliklendirilmiş bir eylem planı oluşturulur. Bu plan, belirli adımları, zaman çizelgelerini ve sorumlu kişileri içermelidir.
    \end{enumerate}
\end{itemize}

\subsection{İç Tehdit Göstergeleri ve Davranışsal Analiz}

İç tehditler, organizasyon içinden gelen risklerdir ve iki ana kategoriye ayrılır: Kötü niyetli (malicious) ve kasıtsız (unintentional) tehditler. Kasıtsız tehditler, dikkatsizlik veya kötü muhakeme sonucunda ortaya çıkar ve aslında çoğu siber olay insan hatasından kaynaklanır (\%75-95). Bir iç tehdidin dijital ve fiziksel davranışları, potansiyel bir riskin anahtar göstergeleridir.

\begin{itemize}
    \item \textbf{Davranışsal Göstergeler:}
    \begin{itemize}
        \item \textbf{Alışılmadık Erişim Kalıpları:} Bir çalışanın normalde yapmadığı saatlerde, konumlardan veya cihazlardan oturum açması.
        \item \textbf{Veri Sızdırma Girişimleri:} Geleneksel eşiklerin altında kalacak şekilde çoklu küçük dosya transferleri gibi alışılmadık boyutta veri hareketleri.
        \item \textbf{Yetki Kötüye Kullanımı:} Bir çalışanın işi için gerekli olmayan sistemlere veya verilere erişim talepleri veya bu yönde davranışlar sergilemesi.
    \end{itemize}
    \item \textbf{Tespit Teknolojileri:}
    \begin{itemize}
        \item \textbf{Kullanıcı ve Varlık Davranış Analizi (UEBA):} Kullanıcıların normal davranışlarını izleyen ve bir temel çizgi (baseline) oluşturan ileri düzey bir güvenlik yaklaşımıdır. Bu temel çizgiden sapan her türlü anormallik, potansiyel bir tehdit olarak işaretlenir.
        \item \textbf{Veri Kaybı Önleme (DLP):} Hassas verilerin organizasyon dışına izinsiz transferini engellemeyi amaçlayan bir stratejidir.
    \end{itemize}
\end{itemize}

İç tehdit, sadece bir teknoloji sorunu değil, aynı zamanda psikolojik ve örgütsel bir sorundur. Bir çalışanın yetkilerini aşması veya olağandışı saatlerde oturum açması, finansal stres, hayal kırıklığı veya intikam arzusu gibi duygusal motivasyonlardan kaynaklanabilir. Bu nedenle, bir iç tehdit programı sadece teknolojik izleme araçlarını değil, aynı zamanda çok disiplinli bir tehdit yönetimi ekibini de içermelidir. Bu ekip, bir tehdidin sadece ne yaptığını değil, aynı zamanda neden yaptığını da analiz etmelidir.

\begin{longtable}{|>{\footnotesize}p{1.8cm}|>{\footnotesize}p{2.5cm}|>{\footnotesize}p{2cm}|>{\footnotesize}p{2.2cm}|}
    \caption{İç Tehdit Göstergeleri ve Tespit Yöntemleri} \\
    \hline
    \textbf{İç Tehdit Davranışı} & \textbf{Davranışsal Göstergeler} & \textbf{Tespit Teknolojileri} & \textbf{Tespit Mekanizması} \\
    \hline
    \endhead
    \textbf{Yetki Kötüye Kullanımı} & İşi için gerekli olmayan kritik sistemlere veya verilere erişim denemeleri. & UEBA, SIEM & Normal davranışlar için temel çizgi oluşturarak anormallikleri tespit etme. \\
    \hline
    \textbf{Veri Sızdırma} & Çok sayıda küçük dosyanın dış e-postalara veya harici depolama cihazlarına transferi. & DLP, UEBA & Hassas verilerin sınıflandırılması ve izinsiz çıkışını engelleme. \\
    \hline
    \textbf{Alışılmadık Erişim} & Normalde oturum açılmayan saatlerde veya konumlardan giriş denemeleri. & UEBA, SIEM & Erişim kalıplarındaki sapmaları tespit ve risk skorlama. \\
    \hline
\end{longtable}

\subsection{Yönetici Hedefleme (Whaling) ve Yüksek Değerli Hedeflere Yönelik Saldırılar}

Whaling, yüksek rütbeli bir yöneticiyi (CEO, CFO) hedef alan son derece kişiselleştirilmiş bir oltalama saldırısıdır. Saldırganlar, hedefin finansal hesaplara, maaş bordrosu bilgilerine ve gizli verilere erişimini kullanmayı amaçlar. Saldırganlar, bir CEO'ya odaklanarak tüm şirketin fonlarına veya gizli verilerine erişim elde edebilirler. Bu, saldırganların "en az çaba en çok etki" prensibini nasıl uyguladığının bir örneğidir.

\subsection{Üçüncü Taraf ve Tedarik Zinciri Sosyal Mühendislik}

Saldırganlar, bir organizasyonun tedarikçileri, iş ortakları veya danışmanları üzerinden bilgi sızdırmaya çalışabilir. Tedarikçi E-posta Ele Geçirme (VEC), karmaşık tedarik zincirlerine sahip sektörlerde önemli bir tehdit oluşturmaktadır. Bu saldırılar, bir üçüncü taraf satıcının ödeme talimatlarını güncelliyormuş gibi görünerek finansal dolandırıcılıkları hedefler.

\subsection{Uzaktan Çalışma Ortamı Sosyal Mühendislik Riskleri}

Hibrit veya uzaktan çalışma ortamları, sosyal mühendislik saldırıları için yeni riskler yaratır. Çalışanların ev ağlarında veya ortak alanlarda çalışması, teknik güvenlik kontrollerini aşmayı kolaylaştırır. Bu durum, kimlik doğrulama, şifre yönetimi ve cihaz güvenliği gibi konularda ek zafiyetler yaratır.

\section{Sosyal Mühendislik Savunma Stratejileri}

Sosyal mühendislik saldırılarına karşı savunma, yalnızca teknolojik kontrollere dayanmaz; aynı zamanda güçlü güvenlik politikaları, sürekli çalışan eğitimi ve proaktif bir güvenlik kültürü gerektirir. Bir kuruluş, sosyal mühendislik tehditlerine karşı çok katmanlı bir savunma stratejisi benimseyerek, insan faktöründen kaynaklanan riskleri önemli ölçüde azaltabilir.

\subsection{Güvenlik Farkındalığı Eğitim Programı Tasarımı}

Etkili bir güvenlik eğitim programı, riskleri azaltmak ve proaktif bir güvenlik yaklaşımı benimsemek için çok önemlidir.

\begin{itemize}
    \item \textbf{Tasarım Aşamaları:}
    \begin{enumerate}
        \item \textbf{Üst Yönetim Desteği (Executive Buy-in):} Program için bütçe ve kaynak sağlamak ve güvenlik farkındalığını şirket önceliği haline getirmek için üst yönetimin desteği alınmalıdır.
        \item \textbf{İhtiyaç Analizi:} Çalışanların güvenlik bilgilerindeki boşlukları (gaps) belirlemek için anketler ve başlangıç oltalama simülasyonları yapılır.
        \item \textbf{Hedeflerin Belirlenmesi:} Programın amaçları, somut ve ölçülebilir hedefler (örn. tıklama oranlarını düşürme, raporlama oranını artırma) olarak tanımlanır.
        \item \textbf{Uygulama ve Katılım:} Programın iç tanıtımı yapılır ve gamification (oyunlaştırma) gibi stratejilerle katılım teşvik edilir.
        \item \textbf{Ölçüm ve Özelleştirme:} Programın etkinliği, tamamlanma oranları ve raporlama metrikleri gibi verilerle düzenli olarak ölçülür. Yüksek riskli çalışanlar için özel eğitimler tasarlanır.
    \end{enumerate}
\end{itemize}

\subsection{Oltalama Simülasyonu ve Ölçüm Programları}

Oltalama simülasyonları, çalışanları gerçek dünya oltalama girişimlerine maruz bırakarak, güvenli bir ortamda öğrenmelerini sağlayan pratik bir eğitim bileşenidir. Bu programlar, geleneksel eğitimlerden çok daha etkilidir.

\begin{itemize}
    \item \textbf{Uygulama Adımları:}
    \begin{enumerate}
        \item \textbf{Planlama:} Simülasyonun hedefleri (eğitim, zafiyet seviyesini belirleme) ve kapsamı (hedef kitle, senaryo karmaşıklığı) belirlenir.
        \item \textbf{Senaryo Oluşturma:} Gerçekçi e-posta şablonları (sahte fatura, şifre sıfırlama talebi) hazırlanır. İçerik, hedef departmanın sorumluluklarına göre özelleştirilir.
        \item \textbf{E-postaların Gönderilmesi:} Şüphe uyandırmamak ve gerçek bir saldırıyı taklit etmek için e-postaların dağıtımı rastgele yapılır.
        \item \textbf{Yanıtların İzlenmesi:} Tıklama, indirme, kimlik bilgilerini girme veya e-postayı raporlama gibi kullanıcı etkileşimleri izlenir.
        \item \textbf{Geribildirim ve Eğitim:} Saldırıya maruz kalan çalışanlara, atladıkları ipuçlarını (red flags) gösteren anında geri bildirim ve ek eğitim materyalleri sunulur.
    \end{enumerate}
\end{itemize}

\subsection{Güvenlik Kültürü Geliştirme ve Davranış Değişikliği}

Etkili bir güvenlik programının nihai hedefi, çalışanların bilgiyi davranışa dönüştürdüğü kalıcı bir güvenlik kültürü oluşturmaktır. Bu kültürel değişim, sadece tıklama oranlarını düşürmeyi değil, aynı zamanda proaktif bir raporlama mekanizması oluşturmayı da içerir.

\begin{itemize}
    \item \textbf{Kültür Değişimi için Anahtar Faktörler:}
    \begin{itemize}
        \item \textbf{Pozitif Takviye (Positive Reinforcement):} Hata yapanları cezalandırmak yerine, şüpheli e-postaları rapor edenleri ödüllendirmek gerekir. Oyunlaştırma (gamification) ve liderlik tablosu gibi mekanizmalar, olumlu duygular ve motivasyon yaratır.
        \item \textbf{Psikolojik Güvenlik:} "Hata yapma hakkı" tanınan bir ortam yaratmak, çalışanların cezalandırma korkusu olmadan hatalarını bildirmelerini sağlar. Bu, bir saldırının tespiti ve müdahalesini geciktiren korku tabanlı bir kültürün aksine, olumlu bir etki yaratır.
        \item \textbf{Sürekli Pratik:} Öğrenmelerin pekişmesi için simülasyonların sık sık (örneğin, her 10 günde bir) tekrarlanması gerekir.
        \item \textbf{İlgililik ve Özelleştirme:} Eğitim materyallerinin her çalışan rolüne ve yaşadığı gerçek senaryolara göre uyarlanması katılımı artırır.
    \end{itemize}
\end{itemize}

\begin{longtable}{|>{\raggedright\arraybackslash}p{2.5cm}|>{\raggedright\arraybackslash}p{3.5cm}|>{\raggedright\arraybackslash}p{3.5cm}|>{\centering\arraybackslash}p{3cm}|}
\caption{Oltalama Simülasyonu Başarı Metrikleri ve Anlamları} \\
\hline
\rowcolor{tableheadcolor}
\textbf{Metrik} & \textbf{Ölçüm Yöntemi} & \textbf{Güvenlik Kültürü İçin Anlamı} & \textbf{Hedef} \\
\hline
\endfirsthead
\multicolumn{4}{c}{\small\tablename\ \thetable\ -- devamı} \\
\hline
\rowcolor{tableheadcolor}
\textbf{Metrik} & \textbf{Ölçüm Yöntemi} & \textbf{Güvenlik Kültürü İçin Anlamı} & \textbf{Hedef} \\
\hline
\endhead
\hline
\multicolumn{4}{r}{\small Devamı sonraki sayfada} \\
\endfoot
\hline
\endlastfoot
\textbf{Tıklama Oranı} & Tıklama yapan kullanıcı sayısı / Toplam kullanıcı sayısı & Çalışanların saldırılara karşı genel duyarlılığı ve farkındalık seviyesi. & Zaman içinde tutarlı bir şekilde düşüş göstermesi. \\
\hline
\textbf{Raporlama Oranı} & Şüpheli e-postayı raporlayan kullanıcı sayısı / Tıklama yapan kullanıcı sayısı & Çalışanların güvenlik sürecine aktif katılımı ve psikolojik güvenlik seviyesi. & Sürekli artış göstermesi. \\
\hline
\textbf{Olayı Bildirme Süresi} & E-postanın gönderildiği an ile ilk raporlamanın yapıldığı an arasındaki süre. & Tehdide karşı organizasyonun reaksiyon hızı. & Mümkün olduğunca kısalması. \\
\end{longtable}

\subsection{Teknik Kontroller: E-posta Filtreleme, Web Koruması}

Güvenlik farkındalığı eğitimleri tek başına yeterli değildir, teknik önlemlerle desteklenmelidir. E-posta filtreleme sistemleri, kötü amaçlı yazılım içeren ekleri veya oltalama bağlantılarını otomatik olarak engelleyerek saldırganların kullanıcıya ulaşmasını zorlaştırır. Çok Faktörlü Kimlik Doğrulama (MFA) kullanımı, çalınan kimlik bilgilerinin etkisini büyük ölçüde azaltır.

\subsection{Olay Bildirme ve Müdahale Mekanizmaları}

Etkili bir savunmanın son adımı, bir saldırı meydana geldiğinde nasıl hareket edileceğine dair net bir plana sahip olmaktır. Bu plan, olay bildirme protokollerini, müdahale ekibinin rollerini ve sorumluluklarını içermelidir. Olay müdahale planları, gerçek bir olay meydana gelmeden önce düzenli olarak test edilmelidir.

\section{İleri Seviye Sosyal Mühendislik ve Gelecek Tehditler}

Sosyal mühendislik, sürekli olarak gelişen ve daha sofistike hale gelen bir tehdittir. Gelecekteki sosyal mühendislik saldırılarının, yapay zeka (AI), makine öğrenimi (ML) ve deepfake gibi teknolojileri kullanarak daha kişiselleştirilmiş ve inandırıcı olması beklenmektedir. Bu teknolojiler, saldırganların hedeflerini daha etkili bir şekilde manipüle etmelerine ve geleneksel savunma mekanizmalarını atlatmalarına olanak tanıyabilir.

\subsection{Yapay Zeka Destekli Sosyal Mühendislik Saldırıları}

Yapay zeka (AI), geleneksel sosyal mühendislik taktiklerini çok daha etkili ve ölçeklenebilir hale getirmektedir. Generatif AI, dilbilgisi hatası olmayan, son derece kişiselleştirilmiş ve gerçekçi e-postalar, SMS mesajları ve telefon görüşmeleri oluşturabilir. Bu, geleneksel filtreleri ve insan gözünü kolayca atlatabilir.
\begin{itemize}
    \item \textbf{Deepfake ve Ses Klonlama:} Saldırganlar, bir kişinin sesini klonlayarak veya videosunu taklit ederek bir yöneticiyi veya meslektaşı ikna edici bir şekilde taklit edebilir.
    \item \textbf{AI Destekli Sohbet Botları:} Gerçek zamanlı olarak insan etkileşimini taklit eden otomatik sohbet botları, aynı anda binlerce kişiyle etkileşim kurarak hassas bilgiler elde edebilir.
\end{itemize}

Yapay zeka, sosyal mühendisliğin doğasını ve ölçeğini kökten değiştirmektedir. Daha önce beceri ve zaman gerektiren bir saldırı (örneğin, bir CEO'nun yazma stilini taklit etmek), artık birkaç saniye içinde bir AI tarafından yapılabilmektedir. Bu, savunmanın artık "bariz hataları" (imla hataları) arayamayacağını, bunun yerine kimlik doğrulamaya ve çok faktörlü doğrulamaya (MFA) daha fazla odaklanması gerektiğini gösterir. Bu, savunmanın doğrulama (verify) ve güvenmeme (zero trust) prensiplerine geçişini hızlandırır.

\subsection{Sentetik Kimlik Oluşturma ve Manipülasyonu}

Sentetik kimlik, gerçek ve uydurulmuş bilgilerin birleştirilmesiyle oluşturulan tamamen sahte bir kimliktir. Bu kimlikler, genellikle ölen veya çocukların sosyal güvenlik numaraları gibi gerçek unsurları, sahte isim ve adreslerle birleştirir. Geleneksel kimlik hırsızlığından farklı olarak, sentetik kimlikler kredi notu oluşturmak ve finansal dolandırıcılık yapmak için aylar veya yıllar boyunca kullanılabilir.

\subsection{Etki Operasyonları ve Dezenformasyon Kampanyaları}

Etki operasyonları, siber uzayda kamuoyunu manipüle etmeyi ve karar verme süreçlerini etkilemeyi amaçlayan kasıtlı faaliyetlerdir. Bu operasyonlar, yanlış veya yanıltıcı bilgileri (dezenformasyon) yayarak hedefin inançlarını ve davranışlarını değiştirmeyi hedefler. Dezenformasyon kampanyaları, sahte profiller, otomatik botlar ve yapay zeka tarafından oluşturulan inandırıcı içeriklerle yürütülür ve kamuoyu, seçimler veya şirket itibarları üzerinde yıkıcı etkilere sahip olabilir.

\subsection{Hibrit Savaş ve Ulus-Devlet Sosyal Mühendisliği}

Hibrit savaş, siber saldırıları, dezenformasyonu, ekonomik baskıyı ve fiziksel operasyonları bir araya getiren bir savaş biçimidir. Sosyal mühendislik, bu savaşın kritik bir bileşenidir, çünkü hedefin insan katmanını, yani "yumuşak hedefi" istismar eder. Ulus-devletler, entelektüel mülkiyeti, ticari sırları veya gizli hükümet materyallerini sızdırmak için siber casusluk operasyonlarında sosyal mühendisliği bir araç olarak kullanırlar.

\subsection{Karşı İstihbarat ve Savunmacı Sosyal Mühendislik}

Gelecekteki tehditlere karşı en güçlü savunma, pasif bir duruştan (saldırının gelmesini bekle) aktif bir duruşa (düşmanın ne yapacağını tahmin et ve önle al) geçmeyi gerektirir. Bu yaklaşım, sadece saldırıları engellemeye çalışmak yerine, saldırganın taktiklerini ve düşünce süreçlerini anlamayı amaçlayan "karşı-istihbarat" ve "savunmacı sosyal mühendislik" kavramlarını içerir. Bu, güvenlik uzmanlarının sadece teknoloji bilmesi değil, aynı zamanda bir saldırgan gibi düşünmesi gerektiğini gösterir.
